\documentclass[a4paper,11pt,twoside]{book}
\usepackage[english]{babel}
\usepackage[latin1]{inputenc}
\usepackage[T1]{fontenc}
\usepackage{amssymb}
\usepackage{graphics}
\usepackage{amsmath}
\usepackage{geometry}
\usepackage{amssymb}
\usepackage{graphicx}
\usepackage{threeparttable}
\usepackage{wrapfig}
%\usepackage{floatflt}
\usepackage{fancyhdr}
%\usepackage{acsarticle}
\usepackage[hang, bf]{caption}
%\usepackage{xymtex}
%\usepackage{chemist}
\usepackage{overcite}

\geometry{a4paper,tmargin=28mm,bmargin=42mm,lmargin=35mm,rmargin=40mm}
%\geometry{a4paper,tmargin=50mm,bmargin=20mm,lmargin=4cm,rmargin=4cm}
\addtolength\evensidemargin{-7mm}
\addtolength\oddsidemargin{7mm}
\pagestyle{fancy}
\renewcommand{\chaptermark}[1]{\markboth{\chaptername\ \thechapter. #1}{}}
\renewcommand{\sectionmark}[1]{\markright{\ #1 }}

\fancypagestyle{plain}{
	\fancyhead{}
	\renewcommand{\headrulewidth}{0pt}
}
\fancyhf{}
\fancyhead[le,ro]{\bfseries\thepage}
\fancyhead[re]{\bfseries\leftmark}
\fancyhead[lo]{\bfseries\rightmark}
\renewcommand{\headrulewidth}{0.1pt}
\renewcommand{\footrulewidth}{0.1pt}
\addtolength{\headheight}{0.5pt}
\headsep=10mm
\makeatletter
\@addtoreset{equation}{section}
\@addtoreset{table}{section}
\@addtoreset{figure}{section}
\makeatother
\renewcommand{\theequation}{\arabic{chapter}.\arabic{section}.\arabic{equation}}
\renewcommand{\thetable}{\arabic{chapter}.\arabic{section}.\arabic{table}}
\renewcommand{\thefigure}{\arabic{chapter}.\arabic{section}.\arabic{figure}}
\linespread{1.2}

% sommatorie 
\newcommand{\sumoneinf}[1]{\sum_{#1 = 1}^{+\infty}}
\newcommand{\sumonen}[1]{\sum_{#1 = 1}^{n}}
\newcommand{\sumonenprime}[1]{\sumonen{#1} \! {}^{\prime}}
\newcommand{\sumalpha}{\sum_{\alpha = 1}^{N}}
\newcommand{\sumidx}[1]{\sum_{#1}}
\newcommand{\sumidxprm}[1]{\sum_{#1} \! {}^{\prime}}

% alcune macro utili
\newcommand{\beq}{\begin{equation}}
\newcommand{\eeq}{\end{equation}}
\newcommand{\beqa}{\begin{eqnarray}}
\newcommand{\eeqa}{\end{eqnarray}}
\newcommand{\beqas}{\begin{eqnarray*}}
\newcommand{\eeqas}{\end{eqnarray*}}

% hamiltoniani ed altri operatori
\newcommand{\ham}{\hat{\mathcal{H}}}
\newcommand{\hamel}{\hat{\mathcal{H}}_{el}}
\newcommand{\hamtot}{\hat{\mathcal{H}}_{tot}}
\newcommand{\fock}{\hat{\mathcal{F}}}

% energia totale
\newcommand{\etot}{E_{tot}}

% psi (varie)
\newcommand{\psia}{\psi_a}
\newcommand{\psii}{\psi_i}
\newcommand{\psij}{\psi_j}
\newcommand{\psik}{\psi_k}
\newcommand{\psil}{\psi_l}
\newcommand{\psin}{\psi_n}
\newcommand{\psim}{\psi_m}
\newcommand{\psitilde}{\tilde{\psi}}
\newcommand{\psitot}{\psi_{tot}}
\newcommand{\psiixq}{\psii \left(x,\! Q\right)}
\newcommand{\psiixqp}{\psii \left(x;\! Q\right)}
\newcommand{\psitotxq}{\psitot \left(x,\! Q\right)}
\newcommand{\psitotxqp}{\psitot \left(x;\! Q\right)}

% costanti frequenti
\newcommand{\half}{\frac{1}{2}}
\newcommand{\hfrac}{\frac{\hbar^2}{2}}
\newcommand{\hfracm}{\frac{\hbar^2}{2 m} }

% derivate
\newcommand{\dpart}[1]{\frac{\partial}{\partial #1}}
\newcommand{\ddpart}[1]{\frac{\partial^2}{\partial #1^2}}
\newcommand{\dpartfrac}[2]{\frac{\partial #1}{\partial #2}}
\newcommand{\ddpartfrac}[2]{\frac{\partial^2 #1}{\partial #2^2}}
\newcommand{\nablavect}[1]{\overrightarrow{\nabla} \!\! _#1}
\newcommand{\nablaquad}[1]{\nabla^2 \! \! \! _#1}

% braket

\newcommand{\bra}[1]{\left\langle #1 \right|}
\newcommand{\ket}[1]{\left| #1 \right\rangle}
\newcommand{\vacuum}{\left| \mbox{vac} \right\rangle}
\newcommand{\braket}[3]{\left\langle #1 \left| #2 \right| #3 \right\rangle}
\newcommand{\integral}[2]{\left\langle #1 \! \!  \right.\left| #2 \right\rangle}
\newcommand{\integralmodif}[2]{\left\langle #1 \!  \left| #2
\right. \right\rangle}
\newcommand{\interact}[2]{\left\langle #1 \left| \right| #2 \right\rangle}
\newcommand{\commut}[2]{\left[ #1,  #2 \right]}

% altri

\newcommand{\adiab}{\hat{\Lambda}_{ki}}
\newcommand{\abs}[1]{\left| #1 \right|}
\newcommand{\comm}[2]{\left[ #1, #2 \right]}
\newcommand{\anticomm}[2]{\left[ #1, #2 \right]_{+}}

% orbitali e analoghi

\newcommand{\pistar}{\pi^{*}}
\newcommand{\detsl}[1]{\left\Vert #1 \right\Vert}

% simboli ricorrenti
\newcommand{\constr}[1]{a_{#1}^{+}}
\newcommand{\destr}[1]{a_{#1}}
\newcommand{\proj}[1]{\hat{\mathcal{P}}_{#1}}
\newcommand{\perturb}{\hat{\mathcal{V}}}
\newcommand{\resolvent}{\frac{\mathcal{Q}_0}{a}}

% varie sigle
\newcommand{\dalton}{\texttt{DALTON}\ }
\newcommand{\molcas}{\texttt{MOLCAS}\ }
\newcommand{\columbus}{\texttt{Columbus}\ }

% eccitazioni
\newcommand{\npi}{$n\!\!\rightarrow\!\!\pi^*$}
\newcommand{\snpi}{$^1\!\left(n\!\rightarrow\!\!\pi^*\right)$}
\newcommand{\snts}{$^1\!\left(n\!\rightarrow\!\!  3s\right)$}
\newcommand{\tnpi}{$^3\!\left(n\!\rightarrow\!\!\pi^*\right)$}
\newcommand{\pipi}{$\pi\!\!\rightarrow\!\!\pi^*$}
\newcommand{\spipi}{$^1\!\left(\pi\!\rightarrow\!\!\pi^*\right)$}
\newcommand{\tpipi}{$^3\!\left(\pi\!\rightarrow\!\!\pi^*\right)$}
\newcommand{\spi}{$\sigma\!\!\rightarrow\!\!\pi^*$}
\newcommand{\sspi}{$^1\!\left(\sigma\!\rightarrow\!\!\pi^*\right)$}
\newcommand{\tspi}{$^3\!\left(\sigma\!\rightarrow\!\!\pi^*\right)$}


\newcommand{\cit}[1]{\csname b@#1\endcsname}
\newcommand{\bbra}[1]{\left<#1\right|}
\newcommand{\kket}[1]{\left|#1\right>}
\newcommand{\bracket}[3]{\left<#1\left|#2\right|#3\right>}
\newcommand{\scal}[2]{\left<#1\left|\right.#2\right>}
\newcommand{\elm}[3]{\left< #1 \left| #2 \right| #3 \right>}
\newcommand{\hami}{\hat{\cal H}}
\newcommand{\ungap}{\!}
\newcommand{\nto}[1]{$n \rightarrow #1\!$}
\newcommand{\nyto}[1]{$\!n_y\!\!\rightarrow\!\! #1$}
\newcommand{\pipis}{$\!\pi\!\!\rightarrow\!\! \pi^*$}
\newcommand{\sipis}{$\!\sigma\!\!\rightarrow\!\! \pi^*$}
\newcommand{\psimz}{\Psi_m^{(0)}}
\newcommand{\pertu}{\Psi_{l,\mu}^{(k)}}
\newcommand{\pertus}{S_l^{(k)}}
\newcommand{\pertusp}[2]{S_{#1}^{({#2})}}
\newcommand{\pertup}[2]{\Psi_{#1}^{({#2})}}
\newcommand{\pertusb}{\bar{S}_l^{(k)}}
\newcommand{\psilmk}{\Psi_{l,\mu}^{(k)}}
\newcommand{\pertue}[2]{E_{#1}^{({#2})}}
\newcommand{\elmk}{E_{l,\mu}^{(k)}}
\newcommand{\E}[1]{E_{#1}}
\newcommand{\crea}[1]{a_{#1}^{+}}
\newcommand{\dist}[1]{a_{#1}^{\phantom{+}}}
\newcommand{\heff}[1]{h_{#1}^{\mathrm{eff}}}
\newcommand{\emp}[1]{E_m^{(#1)}}






\raggedbottom
\begin{document}
\pagestyle{empty}

UNIVERSIT\'E PAUL SABATIER
UNIVERSIT\'A DEGLI STUDI DI FERRARA
Ph.D Thesis - Th\`ese d'Universit\'e 
Exam date - Date de soutenance: 2005/03/17
Candidate - Candidat: Stefano BORINI

\begin{center}
\textbf{Theory and applications of advanced techniques in quantum chemistry
and their integration in a common infrastructure}
\end{center}

\textbf{R\'esum\'e}:

\textbf{Abstract}:
This work surveyed two innovative techniques dedicated to the ab initio
study of molecular systems:

The Freeze-and-Cut is an extension of the localization technique developed
in Toulouse, which is based on an optimization procedure that preserves the
locality of a starting guess of molecular orbitals. Freeze-and-Cut allows
the reduction of the computational cost by performing a neglection of the
molecular system not involved in the molecular phenomena under study. 

The n-electron valence state perturbation theory (NEVPT) is a
multireference perturbation theory applicable to CAS wavefunctions. 
The perturber wavefunctions are multireference in nature, and belongs to
eight excitation classes, depending on the promotion schema of the
electrons. NEVPT is available in two accuracy levels, Strongly Contracted
and Partially Contracted, and is blessed with many powerful properties, like
size consistency, absence of intruder states and computational efficiency.
NEVPT is available as a single state approach, where the perturbation is
applied to a specified electronic state, as a Quasi Degenerate
approach, where the perturbation effects are applied simultaneously to a set
of electronic states, and as a Non Canonical approach, which removes the
need to perform the evaluation on canonical orbitals, allowing the
direct application of NEVPT on a set of localized orbitals.

Finally, a collaboration with the CINECA supercomputing center lead to
accurate insights in the problems related to the deployment of a grid
infrastructure for quantum chemistry evaluations. Two problems have been
recognized and solved: the need of a library for parsing XML with the
Fortran language solved thanks to the development of the F90xml library, and
a common file format for large binary data, solved by means of the Q5Cost
library.

\textbf{Keywords/Most cl\'es}: ab initio, localization, nevpt, q5cost, f90xml, abigrid

\textbf{R\'esum\'e}:
Ce travail a regarde' deux nouvelle techniques pour l'etudie ab initio des
systemes moleculaires. La technique de localisation utilise une procedure
d'optimisation que preserve la localite' d'une groupe d'orbitals de guess
localisee. Une extension de cette technique, la technique Freeze-and-Cut,
permit la reduction du poids computationnel apres elimination de la partie
du systeme moleculaire pas interesse' a l'event etudie'. Les orbitals
localisee sont gelees a une niveau plus bas de theorie (dans notre cas, une
evaluation SCF) et leur effect est preserve'. Les autres orbitals sont
projectes sur un sousensemble de la bas atomique original, permittant une
reduction de poids computationnel pour la transformation des integrales
bielectroniques. La methode a ete' essaye' sur deux systemes moleculaires:
le (7Z)-13 ammoniotridec-7-enoate, une aminoacid cree' pour en evaluer la
reponse a des cas avec des interactions de charge, et le polyenal C13.
Une troisieme evaluation a ete' effectue' sur l'etat excite' $n \rightarrow
\pi^{*}$ de la
molecule d'acetone avec 6 molecules d'eau. Dans ce cas la dimensionalite'
plus grande de la bas atomique a montre une sensibilite' de la methode
Freeze-and-Cut a des bas diffuse. Une solution peut etre l'utilise de
orbitals non-ortogonaux entre le groupe gelee' et non-gelee'. Ce solution
donne des resultats plus correct, mais la procedure courant doit etre
adapte' avec l'assumption de non orthogonalite'.  Autres developpements
theoretiques sont necessaires pour cette amelliorement.

La deuxieme technique etudie' pendant ce travail de These est la theorie
perturbative n-electron valence state perturbation theory (NEVPT). La NEVPT
est une theorie perturbative multireference qu'il peut etre applique' a des
fonction d'onde de typ CAS. Le developpement theoretique utilise une Hamiltonian
bielectronique, l'Hamiltonian de Dyall, qu'il donne une theorie propre et
efficient. Les fonctions perturbatives sont de nature multireference, et
peut etre classifiee' en huit classes d'excitation, dependant da le scheme
de promotion des electrons. La NEVPT existe en deux niveau d'accurate , la
Strongly Contracted et la Partially contracted, dependant da le niveau de
contraction de l'espace perturbatif. La Strongly Contracted NEVPT utilise 
une espace monodimensional pour chaque classe, et la Partially Contracted 
utilise un espace a dimensionalite' plus haute, mais plus petit que l'espace
complete des determinants excitees.
L'approche Strongly Contracted donne resultats de bonne qualite' si
confrontee avec la Partially Contracted, et une difference entre les deux
est normalment une synthome d'une maivaise description a l'ordre-zero.

La NEVPT est available comme single etat, ou la perturbation est applique' a
un particulier etat electronique. La NEVPT Quasi Degenerate applique
l'effect perturbatif sur une ensemble d'etats a le meme temp. Cet approche
considere les interactions entre les functions d'onde apres la perturbation,
et permit le melangement de ce functions.
Enfin, une variation de la NEVPT appelle' Non Canonical a ete' developpe'
pour obtenir l'invariabilite' des resultats pour le melangement des orbitals
de core et virtuels. Cet approche elimine la necessite' de travailler avec
orbitals  canoniques, et permit l'utilise direct avec un ensemble d'orbitals
localisee'.

Les differents types de NEVPT ont ete' utilise' sur un ensemble de systemes
concernant le group carbonyl. Tous les evaluations effectue' donne des
resultats interessant et en particulier est interessant les proprietes de ce
theorie, comme l'efficience computational et l'absence d'etats intrus.
Les resultats de la theorie QDNEVPT peut etre amelliore' avec une procedure
iterative. Cet amelliorement peut etre une bonne solution a des problemes
dans les resultats courent.
L'integration de la theorie NEVPT avec l'approche localisee' a' ete'
effectue avec la NEVPT Non Canonique. Cette travail sera' important pour
effectuer des calcul sur systemes moleculaire de grand dimension.

Une collaboration avec le centre de supercalcul CINECA a' ete' aussi
effectue' dans le cadre d'une collaboration international entre different
instituts de recherche en chimie quantique. L'objectif est l'integration des 
differentes metodologiques ab initio dans une grid d'ordinateurs.
Deux problemes sont prioritaires dans la phase initial du project: la
necessite' d'une bibliotheque pour la lecture du format de fichier XML avec
le language Fortran, et le developpement d'un format de fichier binaire
commune pour le stockage de donnees de gran dimensions.
Pour le premiere probleme, la bibliotheque F90xml a ete' developpe. Cette
bibliotheque interface la bibliotheque libgdome2 (bibliotheque pour le
language C) avec le language Fortran. 
Pour le deuxieme probleme, on a analisee le format de fichier deja
developpe', et evaluee' leur defects dans l'implementation en grid. Les
necessites de la grid necessite de considerations additionnel, comme
independence par la platforme, simplicite d'utilise de la bibliotheque
d'access, accuratesse des messages d'erreur et extensibilite'.
La recherche effectue' a condu a le developpement d'une modele des donnees
que a ete' implemente' dans la bibliotheque Q5Cost. Cette bibliotheque
a ete' utilise' pour l'integration des differents logiciels, et ouvre
des nouvelle possibilites pour la condivision d'information quantistique et
l'integration de differentes methodes ab initio.
La bibliotheque a aussi une suite de test pour controler l'interface et
chercher problemes dans l'implementation. 
Le resultat est une bibliotheque tres solide, estensible et affidable.

Le problemes principals de Q5Cost sont le non efficient stockage des integrals en terms
d'espace occupee et l'implementation pas encore complete, mais l'implementation
courent est sufficient pour le premiere developpements et implementations.

\end{document}



