{\Huge \textbf{Preface}}
\vspace{6mm}


This Ph.D. thesis does not target a single technique: two quantum chemistry
techniques are studied separately and then analyzed in order to combine them
into a common infrastructure.

The main problem in applicability of \textit{ab initio} methods to medium
and large molecular systems is the very steep increase of the computational
cost in terms of time, memory and disk space, with respect to the size of
the molecule. Many different techniques have been developed to deal with
this problem, ranging from improving algorithms to adding levels of
approximation. In many cases changing the theoretical approach has
proved successful, like in Density Functional Theory methods, but
leading to other new interesting problems and limitations.

Since long time, the best approach to deal with this complexity has been 
the well known \textit{Divide et Impera}: splitting a very large problem in smaller,
more manageable problems. Along with this strategy goes the ``right tool for
the job'' philosophy, leading to a number of techniques created to provide
appropriate tools at different levels of theory, each one with its own
strenghts and weaknesses, collaborating together to deal with the problem in
synergic way.

The first technique presented in this work is the Freeze-and-Cut
localized optimization, developed and implemented at Paul Sabatier
University in Toulouse. The main advantage of this technique is to
provide a purely \textit{ab initio} evaluation on medium and large molecular
systems, achieving at the same time saving of computational cost and
partitioning of a large system in subsystems.

The second technique here presented is the n-Electron Valence state
Perturbation Theory (NEVPT), a perturbative treatment developed at Ferrara
University in partnership with Paul Sabatier University in Toulouse. The main advantage of this
technique is the very accurate evaluation of spectroscopical properties of
the studied chemical systems. Unfortunately NEVPT can be applied only to
small molecular systems, being limited by the feasibility of Complete Active
Space (CAS) evaluations.

Combining these two theories allows the computational chemist to perform a
first low-level analysis with Freeze-and-Cut on the less critical parts of
the molecule under study, and to perform a CASSCF plus NEVPT evaluation on the chemically interesting
part, where a more detailed and high-level treatment is needed.  This
objective presents both theoretical and implementative hurdles.  Refinements
in the NEVPT theory have been introduced, driven by the need to deal with
the localized orbitals approach provided by Freeze-and-Cut. New and
improved standards for file formats and information exchange have been
developed to satisfy the future needs of a common infrastructure, where the
potentiality of the methods can be further combined, used, and improved by
other research laboratories.

This Ph.D. thesis is organized as follows: 
\begin{enumerate}
\item The First chapter addresses the basic knowledge needed for the
subsequent chapters
\item The Second chapter is devoted to localization techniques,
focusing on theory and software implementation. A deep insight in the
\mbox{Freeze-and-Cut} technique will follows, along with obtained results
\item The Third chapter presents NEVPT theory and applications on
various carbonyl-based molecules. Also, various implementations of this
theory will be discussed, such as Quasi-Degenerate and Non-Canonical
approaches, needed in order to work with a localized set of orbitals
\item The Fourth chapter presents aspects and problems related to
integration in a common grid infrastructure, a more manageable file format
and a glance ahead toward the future of quantum chemistry computing.
Particular focus will be granted to F77/F90xml, a C/Fortran binding library to
parse the XML file format with Fortran, and Q5Cost, a purely Fortran 95
library to read and write a common file format with an intention-revealing
interface on the \textit{ab initio} quantum chemistry domain.
\end{enumerate}
