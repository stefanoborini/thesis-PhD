\section{Conclusions}

The Freeze-and-Cut technique has proved its usefulness in optimizing local
orbitals at CASSCF or quasi-CASSCF level. It combines the freeze of the
molecular orbitals that are not relevant for the phenomenon under study, and
the cut of those atoms that are not needed to describe the optimized
orbitals.

Different levels of theory can be used for the molecular framework and the
physically interesting region of the system.  The presented test cases
demonstrate the ability of the method to give essentially the same results
as a localized CASSCF calculation on the complete system, with a
significant reduction of the computational cost.

Two problems must be kept into account in order to obtain reliable CASSCF
energies: the first problem involves the tails of the unfrozen orbitals,
which can have a significant delocalization on the nearest neighbor atoms.
The projection of the unfrozen orbitals into a restricted set of atoms has
the effect of rising the absolute energy, but the relative energies seems to
be poorly affected. 

The second problem is relative to the methodological need to reorthonormalize
the frozen orbitals set after the cut is performed. This can introduce artificial
tails which can create instability when occurring near the active
orbitals. Both effects depend on the atomic basis set spatial distribution.

To face these problems, a general strategy can be devised:
\begin{itemize}
\item reduce the cut out of the unfrozen orbitals, setting the cut seam
far from the unfrozen region. Atoms which are geometrically very distant
from the unfrozen set can be usually cut out safely
\item reduce the influence of the frozen orbitals orthonormalization tails 
on the chemically active region (usually, the active orbitals).  To minimize
this effect, a sufficient spacer of unfrozen orbitals should be kept.
\end{itemize}
Further evaluations are however needed to gain detailed comprehension of
this strategy.
