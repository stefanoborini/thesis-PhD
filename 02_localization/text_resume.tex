\pagestyle{empty}
\begin{center}
{\Huge \textbf Resum\'e chapitre \ref{chp:localization} \\ Localisation }
\end{center}
{\ }\\
\vspace{-1mm}

En g\'en\'eral, la d\'elocalisation des orbitales mol\'eculaires n'est pas un
requisite physique. La description d\'elocalis\'ee des orbitales est
importante pour la connexion directe avec l'\'energie de ionisation definie
par le th\'eor\`eme de Koopmans. Elle donne aussi une bonne explication des
syst\`emes \`a haute d\'elocalisation, par exemple les syst\`emes
conjugu\'es ou le mol\'ecules aromatiques.

Si le syst\`eme se comporte comme un ensemble de liaisons localis\'ees, une
description d\'elocalis\'ee est plus complexe. Par exemple, une approche
localis\'ee explique naturellement les similarit\'es entre les liaisons C-H du
m\'ethane et de l'\'ethane. Une description d\'elocalis\'ee est plus complexe.

La d\'elocalisation est souvent une consequence des proc\'edures
math\'ematiques utilis\'ees pour l'optimisation des orbitales. Si on change
l'approche math\'ematique, on peut obtenir des orbitales optimis\'ees et
localis\'ees, qui donnent une representation plus facile pour le chimiste.
Cette approche peut \^etre aussi utilis\'ee pour r\'eduire le co\^ut
computationnel des m\'ethodes \textit{ab initio}.

l'\'evaluation de la corr\'elation statique et dynamique est centrale pour
obtenir l'\'energie d'un syst\`eme mol\'eculaire avec un haut niveau de pr\'ecision.
La corr\'elation \'electronique est un ph\'enom\`ene local, li\'ee \`a la distribution
spatiale des \'electrons. Une approche d\'elocalis\'ee oblige des \'evaluations tr\`es
ch\`eres, parce que l'effet physique de la corr\'elation est distribu\'e sur un
nombre tr\`es \'elev\'e d'integrales.

En utilisant une description locale, les interactions entre orbitales
eloign\'ees est presque negligible, et peut \^etre \'elimin\'ee. Les effets corr\'elatifs
sont concentr\'es sur un petit ensemble d'int\'egrals tr\`es importantes.
La localit\'e des orbitales mol\'eculaires est reconnue importante pour le Linear
Scaling, c'est \`a dire obtenir un algorithme d'\'evaluation avec co\^ut
computationnel lin\'eaire avec la dimension du syst\`eme mol\'eculaire. 

Des syst\`emes de Linear Scaling sont d\'ej\`a impl\'ement\'es pour les m\'ethodes
single r\'ef\'erence, mais pour l'\'evaluation des \'etats excit\'ees, de
rupture de liaison, de syst\`emes magn\'etiques ou de transfert de charge,
l'approximation single r\'ef\'erence n'est pas suffisante, et une approche
multir\'ef\'erence doit \^etre utilis\'ee.  Le Linear Scaling pour les m\'ethodes
multireference n'existe pas encore, mais la localisation des orbitales est
le premier pas pour l'obtention d'une telle m\'ethode.

La localisation des orbitales peut aider aussi \`a contr\^oler la nature de
l'espace actif sur des calculs Complete Active Space (CAS), et une reduction
de la dimension de cet espace peut \^etre obtenue.

Une m\'ethode de localisation a \'et\'e d\'evelopp\'ee \`a Toulouse. Cette m\'ethode 
est \textit{a priori}, parce que la localisation est gard\'ee \`a partir d'un
set d'orbitales localis\'ees qui sont optimis\'ees \`a niveau Complete
Active Space Self Consistent Field (CASSCF). Au contraire, les m\'ethodes
\textit{a posteriori} travaillent avec des orbitales d\'elocalis\'ees qui
sont localis\'ees apr\`es optimisation. 

Apr\`es localisation, les orbitales pas int\'eress\'ees au ph\'enom\`ene
chimique \'etiud\'e peuvent \^etre gel\'ees \`a un niveau de th\'eorie plus bas,
dans notre cas Self Consistent Field (SCF), pour r\'eduire le poids
computationnel, mais cette m\'ethode change assez peu le travail
n\'ecessaire pour la transformation des integrales bielectroniques, qui a une
complexit\'e O( $N_{\mbox{\tiny AO}}^4 N_{\mbox{\tiny MO}}$ ) avec
$N_{\mbox{\tiny AO}}$ nombre d'orbitales atomiques (AO) et
$N_{\mbox{\tiny MO}}$ nombre d'orbitales mol\'eculaires. Si on g\`ele des
orbitales mol\'eculaires, l'effet est limit\'e au facteur $N_{\mbox{\tiny
MO}}$.  R\'eduire le nombre d'orbitales atomiques et donc le facteur
$N_{\mbox{\tiny AO}}^4$ est central pour utiliser des m\'ethodes \textit{ab
initio} sur grands syst\`emes mol\'eculaires. 

Une strat\'egie possible est l'effacement de la partie de mol\'ecule
pas n\'ecessaire pour la description des orbitales pas gel\'ees. Ces
orbitales sont project\'ees sur un sous-ensemble de la bas atomique
original, et une optimisation CASSCF peut \^etre r\'ealis\'ee sur un nombre
d'atomes r\'eduit avec un poids computational mineur, mais l'effet des
orbitales gel\'ees ser\`a gard\'e pendant l'optimisation.

La technique, appell\'ee Freeze-and-Cut, a \'et\'e utilis\'ee sur deux
syst\`emes mod\`ele. Le premi\`er test est relative au (7Z)-13
ammoniotridec-7-enoate, un aminoacide express\'ement construit pour
\'evaluer la r\'eponse de la m\'ethode \`a l'interaction de charge.
L'interconversion cis-trans autour d'un double liaison a \'et\'e \'etudi\'ee
avec un espace actif minimal de deux \'electrons dans deux orbitales. 

Le deuxi\`eme test a \'et\'e appliqu\'e \`a la mol\'ecule de tridequenal,
pour \'evaluer la coh\'erence de la m\'ethode sur l'estimation de
l'\'energie d'interconversion cisoid-transoid du groupe ald\'ehydique.

Une troisi\`eme \'evaluation a \'et\'e effectu\'e sur l'etat excit\'e $n \rightarrow
\pi^{*}$ de la molecule d'acetone avec 6 molecules d'eau. Dans ce cas, la
dimensionalit\'e plus grande de la bas atomique a d\'emontr\'e une
sensibilit\'e de la m\'ethode Freeze-and-Cut \`a des bases diffuses. Une solution
peut \^etre l'utilise d'orbitales non-orthogonaux entre le groupe gel\'e et
non-gel\'e. Cette solution donne des r\'esultats plus corrects, mais la
procedure courant doit \^etre adapt\'ee avec l'assumption de non
orthogonalit\'e, et donc autres d\'eveloppements theoretiques sont
n\'ecessaires.
