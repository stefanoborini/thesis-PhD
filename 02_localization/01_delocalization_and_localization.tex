\section{Advantages of a localized approach}

In general, delocalization of molecular orbitals is not a physical
requirement. A delocalized picture of the orbitals gained mainstream
attention in connection with the ionization energy as expressed by the
Koopmans theorem. Delocalization can also explain some phenomena in those
systems that possess intrinsic physical delocalization, like for example
polyene chains or aromatic molecules.

However, when a molecule behaves like a set of localized bonds, a
delocalized description introduces a high degree of complexity. As an
example, a localized point of view clearly explains the similar behavior
of C-H bonds in methane, ethane and cyclohexane. A delocalized description
of this physical phenomenon is more difficult.

Delocalization is also a frequent consequence of the mathematical procedures
implemented for orbital optimization. Improving the mathematical asset in
order to obtain spatially localized orbitals allows the quantum chemist to
deal with a more practical representation of the charge distribution, and 
also provides possible new strategies in reducing the computational cost
for \textit{ab initio} evaluations.

A reliable evaluation of the static and dynamic correlation is the main 
limiting factor for a quantum chemical approach to medium and large
molecular system. These contributions to the energy are critical, and their
evaluation computationally demanding, but their nature is a local
phenomenon, related to the spatial distribution of the electrons. A
delocalized approach imposes very expensive evaluations: the physical effect
of correlation is spread over a large number of integrals. Each of these
integrals accounts for an almost unpredictable quantity to the energy, and
their physical meaning is difficult to understand.

Making use of a localized description, interactions between electrons
populating distant, non-overlapping orbitals can be considered negligible, and
thus candidate for being eliminated from the evaluation. As a direct
consequence, correlative effect are now concentrated on a reduced set of
highly important integrals. Neglecting these long-range interactions due to
locality of the molecular orbitals is an important strategy to reduce the
computational cost.

Linear Scaling techniques are already available for SCF calculations
\cite{rmp-71-1085-1999}, M{\o}ller-Plesset perturbative treatments (MP) at
different orders
\cite{jcp-110-3660-1999,pulay-gdesmp,jcc-19-1241-1998,cpl-290-143-1998,
tca-69-357-1986,jcp-86-914-1987,ijqc-70-167-1998}
Single and Double Configuration Interaction (SDCI) \cite{jcp-104-6286-1996},
Single and Double
Coupled Cluster (CCSD)\cite{jcp-104-6286-1996,jcp-111-8330-1999}
and CCSD(T)\cite{jcp-113-9986-2000,jcp-114-661-2001}.  All the cited methods
apply to a single reference wavefunction, and are therefore unbalanced when
two or more determinants of nearly equal weight are critical in the
description of the system. % FIXME ripete il concetto in alto ?
Although multireference Linear Scaling techniques
are still to be devised, a localized orbitals approach is recognized as an
important requisite in achieving this objective, allowing a ``\textit{divide et
impera}'' strategy on the molecule.

It is important to note that many chemically interesting problems require
the application of a multireference scheme in order to produce reliable
results.  Examples of these problems are the treatment of bond breakings,
electronically excited states, magnetic systems, and charge transfer.
Treatment of these systems is important to gather a deep insight in
biological catalysis, photochemical reactions, high atmosphere degradation
of polluting compounds, technological improvements for data storage, data
transfer and computation. 

Localization of molecular orbitals can also benefit multireference
evaluations in terms of quality and size of the active space. 
Phenomena involving electronic excitation or bond breaking usually happen in a
well localized region of the molecule. 

Working with delocalized orbitals, the active space needed to describe these
phenomena is normally defined by those MOs bringing the highest correlative
effects, regardless of the spatial or physical nature. Moreover, the user
has poor or no control of the active orbital selection. Using local
orbitals, a well defined and physically clear active space can be chosen and
maintained during the optimization. 

Finally, a reduction of the reference space can also be obtained: a
localized description clearly depicts which determinants are not important
for the description of the wavefunction, because they express highly ionic
electronic distributions where a large number of electrons are concentrated
in a certain region of space. 

\section{Existent methods of localization}

In the past, localized orbitals describing the molecule were obtained
performing an appropriate transformation of a delocalized manifold, i.e.
applying a unitary transformation to the spatial orbitals row vector
$\mathbf{\psi}$
\beq
\mathbf{\psi}_{\mbox{\tiny loc}} = \mathbf{\psi}_{\mbox{\tiny deloc}} \mathbf{U}
\eeq

Finding localized, equivalent orbitals is normally achieved choosing a
$\mathbf{U}$ matrix which maximizes a ``localization function''.
For example, the contribution from Edmiston and Ruedenberg
\cite{rmp-34-457-1963} aims at minimizing the Coulomb repulsion between
electron pairs occupying two different orbitals.  If $\phi_i$ and $\phi_j$
are two different orbitals, then satisfying the condition
\beq
\sum_{i<j}^{\mbox{\tiny occ}} \braket{\phi_{i}\phi_{j}}{g}{\phi_{i}\phi_{j}} = \mbox{minimum}
\eeq
imposes locality of the charge distribution.  Another approach from
Boys\cite{rmp-32-2-1960} is to maximize the distance between the centroids
of the transformed orbitals.  Other methods, like for example those by
Pipek\cite{jcp-90-4916-1989}, Angeli et al.\cite{cpl-233-102-1995} work
along the same principle, and can be classified as \textit{intrinsic
methods} of localizations.  \textit{Extrinsic methods} work equally well,
and they are realized performing a projection of localized MO's onto a
delocalized optimized manifold\cite{chavet-ladiqc}. 

All these methods resort on \textit{a posteriori} localization techniques,
therefore a prior evaluation of delocalized canonical MOs is needed.
Obtaining a localized set of orbitals is also possible using \textit{a
priori} methods, where a controlled optimization procedure is applied on a
localized but unoptimized guess of orbitals.  The controlled procedure
preserves the locality, rather than imposing it. Methods that address SCF
equations already exist\cite{jcp-34-89-1961,prl-4-17-1969,gilbert-moicpb}.

The main advantage of an \textit{a priori} method is the possibility to
choose and preserve the nature of the active space during a CAS evaluations.
This property is of particular importance in order to build a CAS space
using the orbitals involved in the chemical or physical process under study.

In all these methods, the local orbitals are expressed as an orthogonal set.
In general this approach is computationally more efficient compared to
Generalized Valence Bond methods\cite{jcp-57-738-1972,bobrowicz-mest}
where a set of non-orthogonal orbitals with atomic character is
variationally optimized.

