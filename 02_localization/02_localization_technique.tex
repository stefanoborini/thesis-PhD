\section{Toulouse method of localization}
\label{sec:toulouse_method}

An \textit{a priori} localization method has been recently proposed by the
working group in Toulouse, with the goal to address multireference problems. 
The method has been applied to the description of the excited
state of polyenes polyenals, polyendials and polyenones
\cite{ijqc-101-325-2005,ijqc-97-688-2004,cpl-372-22-2003,mp-101-1389-2003},
C$_{60}$ Fullerene + Sodium interaction \cite{jms_theo-681-203-2004}, nickel
and copper chains \cite{cpl-378-503-2003} in a mixed localized/delocalized
approach, Cu magnetic systems \cite{jpca-107-7581-2003}, Hydrogen chains
\cite{jms_theo-621-51-2003}, and Mo-Cl complexes \cite{cpl-349-555-2001}.

Briefly, this method preserves the local character of LMO, converging to a
CASSCF wavefunction. A localized guess is obtained through the projection
of Localized Molecular Orbitals (LMO) onto the delocalized SCF manifold.
Then, the locality is preserved avoiding a complete diagonalization of the
density matrix. 

Three versions of the optimization algorithm are currently available: 
a variational uncontracted \cite{jcp-116-10060-2002}, developed in Toulouse,
a perturbative contracted \cite{jcp-117-10525-2002} and the recently deployed
variational contracted, both developed in Ferrara.

The subsequent work is based on the variational uncontracted approach. This
approach does not converge to a rigorous CASSCF solution, because of the
uncontracted nature of the implemented CI expansion.  However, as already
shown in Ref. \citen{jcp-116-10060-2002}, this fact does not seem to
have severe practical effects.

\subsection*{Creation of the guess}

The method starts with the creation of a set of Orthogonal Atomic Orbitals
(OAO), obtained through a Schmidt orthogonalization and a $S^{-\frac{1}{2}}$
normalization of the atomic basis set. ANO type \cite{tca-77-291-1990} is
preferred, since the orbitals are described by a minimal number of coefficients.

The orthonormalization is performed against a hierarchical classification of
the orbitals: the core orbitals are orthogonalized among themselves, then
the valence orbitals against the cores and among themselves, finally the
remaining orbitals. The obtained OAOs are then used to build Localized
Molecular Orbitals (LMOs) as linear combinations of OAOs, using the
one-electron density matrix of a one-electron evaluation (Huckel or SCF):
\beq
R_{ij} = \braket{\Phi_{0}}{a_i^+ a_j}{\Phi_{0}}
\eeq

It is possible to define a projector for an atom K (built using the projector onto
the $\tilde{\chi}_{l}$ OAO functions) or for a molecular fragment F (from a sum of projectors
onto the involved atoms)
\beqa
P_{K} = \sum_{l \in K} \ket{\tilde{\chi}_{l}} \bra{\tilde{\chi}_{l}} \\
P_{F} = \sum_{K \in F} P_{K}
\eeqa

The density matrix obtained by the projected wavefunction $P_{F}
\Phi_{0}$ can be diagonalized. This step leads to a linear combination of
OAO, representing MO's localized on the fragment. Some of these MO's
exhibit occupation numbers which are close to two or zero, and they
represent a localized bond between the atoms involved in the fragment. Other
orbitals can exhibit occupation numbers close to one, and they are rejected
since they are hybrid orbitals that are representative of an ``half-bond''
between non considered atoms.

As an example, the creation of LMO's in the carbonyl group of acetone is
here presented.

{\small
\begin{verbatim}
Extracted matrix
----------------

O1 1s   1.6881  0.0000  0.0000  0.3780  0.2764  0.0000  0.0000  0.5185
O1 2px  0.0000  1.2406  0.0000  0.0000  0.0000  0.9340  0.0000  0.0000
O1 2py  0.0000  0.0000  1.8980  0.0000  0.0000  0.0000  0.2703  0.0000
O1 2pz  0.3780  0.0000  0.0000  1.4621 -0.4912  0.0000  0.0000 -0.6081
C1 1s   0.2764  0.0000  0.0000 -0.4912  0.9477  0.0000  0.0000 -0.0968
C1 2px  0.0000  0.9340  0.0000  0.0000  0.0000  0.8168  0.0000  0.0000
C1 2py  0.0000  0.0000  0.2703  0.0000  0.0000  0.0000  1.0402  0.0000
C1 2pz  0.5185  0.0000  0.0000 -0.6081 -0.0968  0.0000  0.0000  0.9164

Orbitals
--------
 
Occup.
Number  1.9975  1.9865  1.9777  1.9761  1.0201  0.9622  0.0709  0.0190
  
O1 1s  -0.5478  0.0000  0.7384  0.0000 -0.0855  0.0000  0.0000 -0.3838
O1 2px  0.0000  0.7814  0.0000  0.0000  0.0000  0.0000 -0.6240  0.0000
O1 2py  0.0000  0.0000  0.0000  0.9607  0.0000 -0.2775  0.0000  0.0000
O1 2pz  0.5405  0.0000  0.6639  0.0000 -0.0413  0.0000  0.0000  0.5151
C1 1s  -0.3477  0.0000 -0.1176  0.0000 -0.8134  0.0000  0.0000  0.4512
C1 2px  0.0000  0.6240  0.0000  0.0000  0.0000  0.0000  0.7814  0.0000
C1 2py  0.0000  0.0000  0.0000  0.2775  0.0000  0.9607  0.0000  0.0000
C1 2pz -0.5356  0.0000 -0.0089  0.0000  0.5739  0.0000  0.0000  0.6195

----------------------------------------------------------------------
    Occupied               Eliminated             Virtuals
 1.9975... 1.9761 /  /  1.0201...  0.9622 /  /  0.0709 ... 0.0190
\end{verbatim}

}

As can be seen, the extracted density matrix leads to a localized set of
orbitals. The first four, with occupation numbers close to two, are the
C-O $\sigma$ bond, the CO $\pi$ bond, and the lone pairs $n_z$ and $n_y$.
Two orbitals express the half $\sigma$ bonds directed toward the remaining
carbon atoms. These bonds will be kept into account in other selections, so
they are discarded. Finally, two orbitals with occupation number close to
zero express the C-O $\pi^{*}$ and $\sigma^{*}$.  The resulting set of LMOs
is not orthogonal, so a new hierarchical orthonormalization is performed.

The obtained orbitals are then projected onto a delocalized SCF manifold.
This step, like an extrinsic method of localization, provides a localized
SCF-quality set of orbitals.

\subsection*{Optimization of the guess}

In the classical (non Freeze-and-Cut) localization technique, orbitals
obtained from the last step are then optimized at CASSCF level.  Working on
a single reference SCF wavefunction $\Phi_{0}$, the optimization
should follow these steps:
\begin{enumerate}
\item define a single excitation configuration interaction
$\Psi_{\mbox{\tiny CIS}} = \Phi_{0} + \sum_{i,r} C_{ir} a_{r}^{+} a_i \Phi_{0}$
and obtain the $C_{ir}$ coefficients through diagonalization
\item create the one electron density matrix $R_{ir} =
\braket{\Psi_{\mbox{\tiny CIS}}}{a_{r}^{+} a_i}{\Psi_{\mbox{\tiny CIS}}}$ and
block diagonalize it
\item use the new set of orbitals to create a new $\Phi_{0}$, and iterate until
$\Psi_{\mbox{\tiny CIS}} = \Phi_{0}$, satisfying the Brillouin theorem \cite{asi-159-1934}.
\end{enumerate}

Particular care must be applied in the density matrix diagonalization step.
A complete diagonalization leads to delocalized natural orbitals. To prevent
this behavior, a controlled block diagonalization is performed
\beq
R_{\mbox{DP}} = W^{+} R W
\eeq
where $R$ is the density matrix, $R_{\mbox{\tiny DP}}$ is the block diagonalized
density matrix, and $W$ is a unitary matrix, given by
\beq
W = U_{\mbox{\tiny D}} U_{\mbox{\tiny P}}^{+}
\eeq
where $U_{\mbox{\tiny D}}$ is a matrix that fully diagonalizes $R$ and
$U_{\mbox{\tiny P}}$ is a matrix that restores the orbital locality, mixing those
orbitals inside the same class (occupied or virtual, in the case of
a single reference calculation). The $U_{\mbox{\tiny P}}$ matrix is obtained
through projection of $U_{\mbox{\tiny D}}$ onto the subspaces and a subsequent
orthonormalization.

The same working principle can be applied to multireference CAS
wavefunctions: using a CAS space, the CI energy and wavefunction are
invariant with respect to rotations inside each block of core, active
and virtual orbitals. This invariance allows us to produce localized orbitals
equivalent to the delocalized ones, and locality can be maintained
preventing as much as possible rotations inside each of the three blocks.

The requirement for convergence is to satisfy the Extended Brillouin Theorem
\beq
\braket{\Psi_{\mbox{\tiny CASSCF}}}{\ham \left[ a_{k}^{+} a_{l} -
a_{l}^{+} a_{k} \right]}{\Psi_{\mbox{\tiny CASSCF}}} = 0
\eeq
which is always fulfilled if $k$ and $l$ refer both to core, active or
virtual orbitals. The previous equation can threfore be decomposed in three
equations between different classes of indexes:
\beqa
\braket{\Psi_{\mbox{\tiny CASSCF}}}{\ham \left[ a_{a}^{+} a_{i} -
a_{i}^{+} a_{a} \right]}{\Psi_{\mbox{\tiny CASSCF}}} & = & 0 \\
\braket{\Psi_{\mbox{\tiny CASSCF}}}{\ham \left[ a_{r}^{+} a_{i} -
a_{i}^{+} a_{r} \right]}{\Psi_{\mbox{\tiny CASSCF}}} & = & 0 \\
\braket{\Psi_{\mbox{\tiny CASSCF}}}{\ham \left[ a_{r}^{+} a_{a} -
a_{a}^{+} a_{r} \right]}{\Psi_{\mbox{\tiny CASSCF}}} & = & 0
\eeqa
where $i$, $a$ and $r$ refer to core, active and virtual orbitals,
respectively. In order to minimize the energy, the non-interaction of
singly-excited contracted CAS wavefunctions must be reached iteratively
using the Super-CI approach. A prototype for this approach has been
developed in Ferrara laboratory, and has been used for the combined
Localization-NEVPT2 evaluation, presented later in this thesis.

An alternative solution, followed at Toulouse laboratories, is to consider a
decontracted scheme, exploiting the full dimensionality defined by the space
of determinants $S = \{\Phi_i\} \cup \{a^{+}_{r} a_{i} \Phi_i\}$ spanned by
the CAS space and their single excitations. The procedure is not a Super-CI
optimization and does not converge to the real CASSCF solution. However, as
pointed out in Ref. \citen{jcp-116-10060-2002}, the resulting orbitals
are very similar to the CASSCF ones.

The third possibility is to work with a perturbartive contracted approach.
The idea is to consider the $\Psi_{\mbox{\tiny CASCI}}$ at a given iteration as a
zero-order description of the $\Psi_{\mbox{\tiny CAS+S}}$. We diagonalize a
zero-order Hamiltonian inside a perturber space Q, spanned by the singly
excited contracted functions $a_{a}^{+} a_{i} \ket{\Psi_{\mbox{\tiny CASCI}}}$,
$a_{r}^{+} a_{i} \ket{\Psi_{\mbox{\tiny CASCI}}}$ and $a_{r}^{+} a_{a}
\ket{\Psi_{\mbox{\tiny CASCI}}}$. From the diagonalization, we obtain perturber
wavefunctions as linear combinations
\beq
\ket{\Psi_{\mu}} = \sum_{ai} c_{ai \mu} \constr{a} \destr{i}
\ket{\Psi_{\mbox{\tiny CASCI}}} + \sum_{ri} c_{ri \mu} \constr{r} \destr{i}
\ket{\Psi_{\mbox{\tiny CASCI}}} + \sum_{ra} c_{ra \mu} \constr{r} \destr{a}
\ket{\Psi_{\mbox{\tiny CASCI}}}
\eeq
and the first-order correction to the wavefunction is
\beq
\ket{\Psi^{(1)}} = \sum_{\mu}
\frac{\braket{\Psi_{\mu}}{\ham}{\Psi_{\mbox{\tiny CASCI}}}}{E_{\mbox{\tiny CASCI}}-E_{\mu}^{(0)}}
\ket{\Psi_{\mu}}
\eeq
From this correction, it is possible to obtain the correction to the
one-particle density matrix
\beqa
\rho_{lk}^{(1)} & = & \sum_{\mu} \left\{
\frac{\braket{\Psi_{\mbox{\tiny CASCI}}}{a_{k}^{+} a_{l}}{\Psi_{\mu}}
\braket{\Psi_{\mu}}{\ham}{\Psi_{\mbox{\tiny CASCI}}}}{E_{\mbox{\tiny CASCI}}-E_{\mu}^{(0)}}
+ \right. \nonumber \\
& & \left. + \frac{\braket{\Psi_{\mbox{\tiny CASCI}}}{\ham}{\Psi_{\mu}} \braket{\Psi_{\mu}}{a_{k}^{+}
a_{l}}{\Psi_{\mbox{\tiny CASCI}}}}{E_{\mbox{\tiny CASCI}}-E_{\mu}^{(0)}} \right\} 
\eeqa
When the convergence is reached, the correction vanishes.
For the choice of the zero-order Hamiltonian, is convenient to make use of a
particular Hamiltonian, the Dyall's Hamiltonian\cite{jcp-102-4909-1995}
\beqa
\ham^D &=& \ham^D_i + \ham^D_v + C \\
\ham^D_i &=& \sumidx{i} \epsilon_i \constr{i} \destr{i} + \sumidx{r}
\epsilon_r \constr{r} \destr{r} \\
\ham^D_v &=& \sumidx{ab} h_{ab}^{\mbox{\tiny eff}} \constr{a} \destr{b} +
\frac{1}{2} \sumidx{abcd} \integral{ab}{cd} \constr{a}\constr{b}
\destr{d}\destr{c}
\eeqa
where labels $i,j,\ldots$, $a,b,\ldots$, $r,s,\ldots$ denote core, active and
virtual orbitals respectively, and $ h_{ab}^{\mbox{\tiny eff}} =
\braket{a}{h+\sumidx{i}\left( J_i - K_i \right)}{b}$.
This Hamiltonian, which will be presented in more details in the n-Electron
Valence state Perturbation Theory approach, leads to a simplification of the
computational asset. The zero-order Hamiltonian is defined as the projection
of the Dyall inside the interested spaces
\beq
\ham_{0} = \proj{\mbox{\tiny CAS}} \ham^D \proj{\mbox{\tiny CAS}} +
\proj{\mbox{\tiny Q}} \ham^D \proj{\mbox{\tiny Q}}
\eeq
The perturbative approach is very straightforward and efficient, and
converges to the real CASSCF wavefunction. However, divergent behaviors 
have been reported when the zero-order wavefunction needs a strong
correction.


