Possiamo anche definire un operatore spin traced
\beq
E_{ij} = \left( \constr{i\alpha}\destr{j\alpha} +
\constr{i\beta}\destr{j\beta} \right)
\eeq
che si riferisce agli orbitali molecolari spaziali,
indipendentemente dal fattore di spin.

Valgono le propriet\`a
\begin{itemize}
\item $ \left[ E_{ij}, E_{kl} \right] = E_{il}\delta_{jk} -
E_{kj}\delta_{il} $
\item $ E^{+}_{ij} = E_{ji} $
\end{itemize}

\subsection{Trasformazione di base}

Al fine di rendere minima l'energia, occorre effettuare una trasformazione 
sia sui coefficienti della nostra funzione MC, sia sulla base spinorbitalica 
su cui costruiamo i vari determinanti.

Le trasformazioni della base spinorbitalica possono essere viste come rotazioni in uno spazio
orbitalico, attuate da un operatore opportuno. Dal momento che la
rotazione non cambia la dimensionalit\`a della base, il risultato
appartiene ancora allo spazio delle funzioni spinorbitaliche di
partenza, e come tale \`e esprimibile come una opportuna combinazione
lineare dei vettori della base iniziale
$$
\mbox{\boldmath $\varphi$}^{\prime} = \mbox{\boldmath $\varphi$}\mathbf{U}
$$
dove {\boldmath $\varphi$} \`e un vettore riga contenente gli spinorbitali di
partenza, {\boldmath $\varphi^{\prime}$} \`e la base ottenuta dalla trasformazione e
$\mathbf{U}$ \`e una matrice unitaria che attua la trasformazione stessa.

Dal momento che viene attuata una trasformazione sulla base, anche gli
operatori di creazione e distruzione andranno incontro ad una trasformazione.
\`E possibile esprimere tale trasformazione attraverso
la relazione
\beq
\constr{i}{}^\prime = e^{\left(-\hat{T}\right)} \constr{i}
e^{\left(\hat{T}\right)}
\eeq
e analogamente per il distruttore. L'operatore $\hat{T}$ \`e
antihermitiano, che pu\`o essere espanso, come operatore
monoelettronico, nella forma
\beq
\hat{T}=\sumidx{i,j}T_{ij}\constr{i}\destr{j}
\eeq
dove i coefficienti $T_{ij}$ definiscono una matrice $\mathbf{T}$ antihermitiana (ovvero
\mbox{$\mathbf{T}^{+} = - \mathbf{T}$})

Si dimostra che \`e possibile trasformare un generico determinante di 
Slater applicando l'operatore $e^{-\hat{T}}$:
\beqas
% 
\ket{m'}= a{'}_i^+ a{'}_j^+ \ldots\vacuum & = &
e^{\left(-\hat{T}\right)} \constr{i} e^{\left(\hat{T}\right)}
e^{\left(-\hat{T}\right)}\constr{j} e^{\left(\hat{T}\right)}\ldots\vacuum
\\
% 
& = & e^{\left(-\hat{T}\right)}\constr{i} \constr{j}\ldots\vacuum \\
%
& = & e^{\left(-\hat{T}\right)}\ket{m}
\eeqas

Dal momento che la trasformazione attuata sugli orbitali
avviene esclusivamente sulla parte spaziale, la matrice di
trasformazione $\mathbf{T}$ \`e partizionata in 4 sottomatrici
\beq
\mathbf{T} = \left(
\begin{array}{cc}
\mathbf{T}_{\alpha\alpha} & \mathbf{T}_{\alpha\beta} \\
\mathbf{T}_{\beta\alpha} & \mathbf{T}_{\beta\beta}
\end{array}
\right)
\eeq

Se, in seguito al cambiamento di base, non si effettua mescolamento tra
orbitali con occupazione $\alpha$ e orbitali con occupazione $\beta$, i
termini delle sottomatrici fuori  diagonale $\mathbf{T}_{\alpha\beta}$ e
$\mathbf{T}_{\beta\alpha}$ sono nulli. Inoltre, dato che le trasformazioni
sono le stesse per lo stesso set spaziale, $\mathbf{T}_{\alpha\alpha} = \mathbf{T}_{\beta\beta}$.
Con queste relazioni, l'operatore $\hat{T}$ risulta
\beqa
%
\hat{T} & = & \sumidx{i,j} \left( T_{ij}^{\alpha\alpha} \constr{i\alpha}
\destr{j\alpha} 
+ T_{ij}^{\alpha\beta} \constr{i\alpha} \destr{j\beta}
+ T_{ij}^{\beta\alpha} \constr{i\beta} \destr{j\alpha}
+ T_{ij}^{\beta\beta} \constr{i\beta} \destr{j\beta} \right) \nonumber \\
%
& = & \sumidx{i,j} T_{ij} \left( \constr{i\alpha} \destr{j\alpha} +
\constr{i\beta} \destr{j\beta} \right) \nonumber \\
%
& = & \sumidx{i,j} T_{ij}E_{ij}
\eeqa
supponendo la matrice antihermitiana $\mathbf{T}$ reale, si avr\`a $T_{ij} = -T_{ji}$ e
quindi
\beq
\hat{T} = \sumidx{i>j} T_{ij}\left(E_{ij} - E_{ji}\right) 
= \sumidx{i>j} T_{ij} E_{ij}^-
\eeq
dove con l'ultimo termine si esprime una scrittura abbreviata.


\subsection{Ottimizzazione dei parametri e degli orbitali}

L'ottimizzazione dei parametri e degli orbitali passa attraverso l'uso
di operatori unitari. \`E gi\`a stato presentato l'operatore $\hat{T}$, 
che attua una rotazione dello spazio orbitalico. Analogamente,
\`e possibile definire un operatore unitario $\hat{S}$ che attua una
trasformazione sui coefficienti della combinazione lineare.

L'energia dello stato MC $\ket{0}$ sar\`a quindi definita
\beq
E(\mathbf{T},\mathbf{S}) =
\braket{0}{e^{\left(-\hat{S}\right)}e^{\left(-\hat{T}\right)}\ham e^{\left(\hat{T}\right)}e^{\left(\hat{S}\right)}}{0}
\eeq
Espandendo con l'identit\`a di Glauber\footnote{Ricordiamo l'identit\`a di Glauber 
\beqas
e^{\hat{A}}\hat{B}e^{-\hat{A}} & = & \hat{B} + \left[ \hat{A} , \hat{B}
\right] + \frac{1}{2!} \left[ \hat{A} , \left[ \hat{A} , \hat{B} \right]
\right] + \frac{1}{3!} \left[ \hat{A} , \left[ \hat{A} , \left[ \hat{A} ,
\hat{B} \right] \right] \right] + \ldots
\eeqas
} tale espressione, si ottiene
\beqas
E(\mathbf{T},\mathbf{S}) & = & \left\langle 0 \left| \ham + \left[\ham,\hat{T}\right] 
+ \left[\ham,\hat{S}\right] 
+ \half \left[ \left[ \ham,\hat{T}\right],\hat{T}\right] \right. \right. \\
& & \left. \left. + \half \left[ \left[ \ham,\hat{S}\right],\hat{S}\right]
+ \left[ \left[ \ham,\hat{T}\right],\hat{S}\right]
+ \ldots \right| 0 \right\rangle
\eeqas

Il primo termine rende conto dell'energia di ordine zero
$E(\mathbf{0},\mathbf{0})$. Il successivo pu\`o essere sviluppato
\beqas
\braket{0}{\left[\ham,\hat{T}\right]}{0} & = & \sumidx{i,j} T_{ij}
\braket{0}{\left[\ham,E_{ij}\right]}{0} \\
%
& = & \sumidx{i,j} T_{ij} g_{ij}^{(o)}
\eeqas
dove abbiamo posto
\beq
g_{ij}^{(o)} = \braket{0}{\left[\ham,E_{ij}\right]}{0}
\eeq
Il simbolo $(o)$ indica che la trasformazione \`e a carico degli
orbitali.

Ora \`e necessario ottenere l'operatore per la trasformazione a
carico dei coefficienti. Sia innanzitutto definito l'operatore di rotazione
sui coefficienti $\hat{S}$ come
\beq
\hat{S} = \sumidx{K\neq0} S_{K0} \left(\ket{K}\bra{0} - \ket{0}\bra{K}
\right)
\eeq
dove $\ket{K}$ sono generici stati, espansi sulla medesima base
spinorbitalica, tali da essere ortogonali tra loro.

Svolgendo opportunamente il commutatore
\beq
\braket{0}{\left[\ham,\hat{S}\right]}{0} =
\sumidx{K\neq0}S_{K0} \left( \braket{0}{\ham}{K} + \braket{K}{\ham}{0}
\right)
\eeq
che con autofunzioni reali, per hermitianit\`a, conduce ad ottenere
\beq
g_K^{(c)} = 2 \braket{0}{\ham}{K}
\eeq
vettore che esprime la derivata prima effettuata su una
variazione dei coefficienti della funzione MC.

\`E possibile procedere in modo analogo ricavando la matrice
\textit{hessiana}, responsabile della trasformazione al secondo ordine.
Si otterranno 3 contributi: un contributo (oo) di trasformazione sugli 
orbitali, un contributo (cc) di trasformazione sui coefficienti e 
un contributo (co) = (oc) di variazione accoppiata orbitali-coefficienti.

Ora, riprendendo e riadattando
\beqas
\mathbf{g} + \mathbf{Hp} = \mathbf{0} \\
\mathbf{Hp} = - \mathbf{g}
\eeqas
si pu\`o scrivere, in definitiva
\beq
\left(
\begin{array}{cc}
\half \mathbf{H}^{(cc)} & \half \mathbf{H}^{(co)} \\
\left( \half \mathbf{H}^{(co)} \right)^+ & \half \mathbf{H}^{(oo)} \\
\end{array}
\right) 
\left(
\begin{array}{c}
\mathbf{S} \\
\mathbf{T} \\
\end{array}
\right) = -
\left(
\begin{array}{c}
\half \mathbf{g}^{c} \\
\half \mathbf{g}^{o} \\
\end{array}
\right)
\eeq

Il metodo Newton-Raphson prevede la risoluzione iterativa di questo
sistema o analoghi (al fine di semplificare il metodo computazionale),
tuttavia richiede che il vettore di guess per il processo iterativo
sia sufficientemente vicino al minimo locale verso cui si vuole convergere. In
caso contrario, il metodo pu\`o convergere molto lentamente o addirittura
divergere. Sono quindi necessarie procedure per identificare una zona
di minimo, in cui definire in modo approssimativo un vettore di guess.

