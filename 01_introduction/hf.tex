\section{Hartree-Fock approximation and SCF method}

A wavefunction can be defined in a space of Slater determinants, each one
describing a possible electronic distribution, by creating a linear
expansion over this basis
\beqa
\label{eqn:hf_1}
\Psi(1,2,\ldots,N) &=& \sumidx{i_1 < i_2 < \ldots < i_N } c_{i_1,i_2,\ldots,i_N} \detsl{\psi_{{i}_1} \psi_{{i}_2} \ldots \psi_{{i}_N}} \\
&=& \sumidx{I} c_I \Phi_I
\eeqa
This wavefunction satisfies the requisite of antisymmetry, and spans the full
space of determinants of the Fock space defined by $N$ electrons and an
infinite number of spinorbitals. For practical reasons in daily
computations, the number of spinorbitals in the expansion is finite, and so
is the expansion.

In Sec. \ref{sec:variational} we reported the possibility to obtain the best
description for the ground state of a molecular system, making use of the
variational theorem and optimizing a given set of parameters of the wavefunction.
Two degrees of freedom are available: the coefficients $c_I$ of the linear
expansion (a linear optimization) and the spinorbitals $\psi_i$ (a non-linear
optimization).

One possible strategy for the optimization is to work
with the complete space of determinants, providing the \textit{Full
Configuration Interaction} (Full-CI) description. This approach is often
unfeasible due to the very high number of coefficients to optimize. Although
special algorithms have been developed to reduce the computational cost,
Full-CI scales factorially with respect to the size of the system, and its
applicability is limited to small molecules. 

On the opposite side, a tough but effective approximation is to consider a
single determinant to describe the system.
The multideterminant expansion 
\beqa
\Psi(1,2,\ldots,N) &=& \sumidx{i_1 < i_2 < \ldots < i_N }
c_{i_1,i_2,\ldots,i_N} \detsl{\psi_{{i}_1} \psi_{{i}_2} \ldots \psi_{{i}_N}} \\
&=& \sumidx{I} c_I \Phi_I
\eeqa
is forced to
\beqas
c_I &=& 0  \qquad \forall I \neq 0 \\
c_0 &=& 1
\eeqas
leading to 
\beq
\Psi(1,2,\ldots,N) = \Phi_0
\eeq
This approach drops any degree of freedom provided by the linear expansion,
and focuses on the optimization of the one-electron spinorbitals.  

It is possible to show that the best spinorbitals for the single determinant
description of a molecular system are the eigenfunctions of a one-electron
operator, named \textit{Fock operator}. The procedure goes under the name of
\textit{Self Consistent Field}, since the Fock operator itself depends on the
spinorbitals, and the solution is built iteratively until autoconsistency is
reached. The \textit{Hartree-Fock} equation
\beq
\label{eqn:hf_2}
\fock \psi_i = \epsilon_i \psi_i
\eeq
gives a set of one-electron spinorbitals which describe at best (in terms of
energy minimization) the electronic situation represented by the single
determinant approximation. These spinorbitals are named \textit{canonical}.

The corresponding eigenvalues $\epsilon_i$ are named \textit{orbital
energies}, and possess an intrinsic physical meaning, being an approximation
of the energy needed for the ionization of the electron occupying the
spinorbital $\psi_i$ (Koopmans theorem\cite{tp-1-104-1933}). Spinorbitals
are normally ordered by their respective orbital energies, from the lowest
one to the highest one. 

On an infinite space of spinorbitals, the Hartree-Fock equation has infinite
solutions. In a real case, the number of solutions is limited by the size of
the atomic basis set used to describe the spinorbitals, but in any case the
lowest total energy is obtained when the first $N$ spinorbitals (where $N$
is the number of electrons) are occupied by electrons. These spinorbitals
are therefore named \textit{occupied}, and the remaining empty spinorbitals
are named \textit{virtual}.

The Hartree-Fock approximation provides good results when the molecule is
evaluated at the equilibrium geometry and electronic ground state, for
closed-shell systems.  Hartree-Fock theory keeps into account only averaged
interactions between electrons. The one-electron nature of the Fock operator
aims at optimizing the interactions of each electron with an averaged charge
cloud described by the remaining electrons, completely discarding
instantaneous interactions.

\section{Brillouin theorem}

The Hartree-Fock equation gives a set of optimized spinorbitals which can be
occupied by electrons to build the single determinant wavefunction $\Psi =
\Phi_0$ with the lowest energy. As a vector of the Fock space, this
wavefunction is not the only electronic distribution which can be built with
the given set of
spinorbitals. Each different electronic distribution can be described as an
excitation of a given number of electrons from the occupied orbitals to the
virtual ones.  We can now reformulate Eqn. \ref{eqn:hf_1} with a different
grouping strategy, honoring the excitation level (referred to the $\Phi_0$)
of each determinant
\beqa
\label{eqn:mr}
\Psi &=& c_0 \ket{\Phi_0} + \sumidx{ir} c_i^r \Phi_i^r + \sumidx{ijrs}
c_{ij}^{rs} \Phi_{ij}^{rs} + \ldots
\eeqa
where $r$ and $i$ denote the excitation of an electron from the
$\psi_i$ orbital to $\psi_r$.

To improve the single determinant description, one would be tempted to
enrich the description including the singly excited contributions, and
to perform a diagonalization of the Hamiltonian inside this subspace of
determinants $\{ \Phi_0 ,  \Phi_i^r  \}$ defined by the Hartree-Fock
determinant and its single excitations. This produces
\beq
\left(
\begin{array}{cc}
\braket{\Phi_0}{\ham}{\Phi_0} & \braket{\Phi_0}{\ham}{\Phi^r_i} \\
& \\
\braket{\Phi^r_i}{\ham}{\Phi_0} & \braket{\Phi^r_i}{\ham}{\Phi^{r^{\prime}}_{i^{\prime}}} \\
\end{array}
\right)
\left(
\begin{array}{c}
c_0 \\
c^r_a \\
\end{array}
\right) = E_0
\left(
\begin{array}{c}
c_0 \\
c^r_a \\
\end{array}
\right)
\eeq
Brillouin's theorem \cite{asi-71-1933, asi-159-1934} guarantees that when
spinorbitals satisfying the Hartree-Fock equation are used to build the
determinants, all the interactions $\braket{\Phi_0}{\ham}{\Phi^r_i}$ are
zero. Thanks to the Slater rules, this relationship can be proved to be
equivalent to a spinorbital one
\beq
\braket{\psia}{\fock}{\psii} = 0
\eeq
Each $\fock\psii$ therefore belongs to a space which is totally orthogonal
to the virtual space.
