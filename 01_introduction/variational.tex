\section{Variational principle}
\label{sec:variational}

Let $\ham$ be the true Hamiltonian, and $\tilde{\Psi}$ an arbitrary
approximated wavefunction of the true wavefunction $\Psi_0$ for the ground
state of a non-degenerate case. The energy $\epsilon$ associated to this
function is given by the equation
\beq
\label{eqn:variational_1}
\epsilon = \frac{\braket{\tilde{\Psi}}{\ham}{\tilde{\Psi}}}{\integral{\tilde{\Psi}}{\tilde{\Psi}}}
\eeq
Under the variational theorem, the energy $E_0$ (eigenvalue of the true ground
state $\Psi_0$) is a lower bound for the energy $\epsilon$ associated to the
approximated wavefunction $\tilde{\Psi}$.
\beq
\epsilon \ge E_0 \quad \forall \tilde{\Psi} \quad \mbox{,} \quad \epsilon = E_0 \Leftrightarrow
\tilde{\Psi} = \Psi_{0}
\eeq
Proof of this theorem can be found in quantum chemistry books (for example
Ref.~\citen{szabo-mqc,mcweeny-momqm}). The theorem guarantees that given
a generic function $\tilde{\Psi}$, the energy evaluated by means of
Eqn. \ref{eqn:variational_1} will be greater, or at best
equal than the true ground state energy, with the equality holding if the function $\tilde{\Psi}$
is the exact function for the electronic state. It is possible to
parametrize $\tilde{\Psi}$ through a set of parameters, and to optimize them
by minimizing the difference between $\epsilon$ and the true energy $E_0$.

Given the linear expansion of $\tilde{\Psi}$ as a linear combination
of a set of functions $\Phi_i$
\beq
\tilde{\Psi} = \sumidx{i}c_i\Phi_i
\eeq
the minimization condition to satisfy is
\beq
\dpartfrac{\epsilon}{c_i} = 0 \quad \forall i
\eeq
supposing real $c_i$ coefficients being part of a $\mathbf{c}$ array we
obtain
\beqa
\epsilon &=& \frac{\sumidx{i,j} c_i c_j \braket{\Phi_i}{\ham}{\Phi_j}}{\sumidx{i,j} c_i c_j \integral{\Phi_i}{\Phi_j}} \nonumber \\
&=& \frac{ \mathbf{c}^{+} \mathbf{H} \mathbf{c}}{\mathbf{c}^{+} \mathbf{S} \mathbf{c}}
\eeqa
Performing a differentiation of the energy with respect to the $\mathbf{c}$
vector we obtain the final minimization condition
\beq
\label{eqn:variational_3}
\mathbf{H}\mathbf{c} = \epsilon \mathbf{S} \mathbf{c}
\eeq
The $\mathbf{c}$ vector represents the linear combination of the $\Phi_i$
set producing the lowest possible value for the associated energy. An
extension of the variational theorem also asserts that each
solution $\epsilon_i$ obtained from Eqn. \ref{eqn:variational_3} satisfies the
same rule for the energy $E_i$ (MacDonald's theorem). 
