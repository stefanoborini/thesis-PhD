\pagestyle{empty}
\begin{center}
{\Huge \textbf Resum\'e chapitre \ref{chp:introduction} \\ Introduction}
\end{center}
\vspace{10mm}

Ce premier chapitre concerne les notions de base pour les d\'eveloppements
suivants. 

La premi\`ere section parle du th\'eor\`eme variational. Ce th\'eor\`eme dit
que si $\epsilon$
\beq
\epsilon =
\frac{\braket{\tilde{\Psi}}{\ham}{\tilde{\Psi}}}{\integral{\tilde{\Psi}}{\tilde{\Psi}}}
\nonumber
\eeq
est l'\'energie associ\'ee \`a une fonction d'onde approch\'ee
$\tilde{\Psi}$, et $E_0$ est l'\'energie valeur propre de l'\'etat
fondamental $\Psi_0$, $E_0$ est une limite inf\'erieure pour $\epsilon$.
Pour un \'etat fondamental non d\'eg\'en\'er\'ee, chaque fonction d'onde
diff\`erente de la fonction d'onde $\Psi_0$ donne une \'energie plus haute
que l'\'energie $E_0$.

La deuxi\`eme section parle du d\'eveloppement perturbatif de
Rayleigh-Schr{\o}dinger dans le cas g\'en\'eral. Ce d\'eveloppement 
obtient l'\'energie avec la correction d'une fonction d'onde bien connue, la
fonction d'onde al'ordre z\'ero en perturbation. Cette fonction est obtenue
comme solution d'un Hamiltonien mod\`ele $\ham_{0} = \ham - \hat{V}$
\beq
\ham_{0} \Psi_n^{(0)} = E_n^{(0)} \Psi_n^{(0)} \quad n=0,1,2,\ldots
\nonumber
\eeq
pour donner la correction \`a l'\'energie au deuxi\`eme ordre
\beq
E_n^{(2)} = \sum_{k} \frac{\left|
\braket{\Psi_n^{(0)}}{\hat{V}}{\Psi_k^{(0)}}
\right|^2}{E_n^{(0)} - E_k^{(0)}}
\nonumber
\eeq
La th\'eorie perturbative M{\o}ller-Plesset au deuxi\`eme ordre (MP2) est le
d\'eveloppement plus connu de cette approche.

En suite, on parler\`a de l'importante n\'ecessit\'e pour la fonction d'onde
d'\^etre antisym\'etrique pour l'echange d'un couple d'\'electrons. Par
exemple, dans le cas d'une fonction d'onde \`a deux electrons, on aura
\beq
\Psi(2,1) = - \Psi(1,2) \nonumber
\eeq
Cette loi naturelle est absolument fondamentale pour la validit\'e physique
d'une fonction d'onde. 

Une nouvelle entit\'e, le d\'eterminant de Slater $\Phi$, a
\'et\'e rationalis\'ee pour travailler avec une fonction d'onde
antisym\'etrique par construction.  Les d\'eterminants de Slater
repr\'esentent l'occupation des orbitales mono\'electroniques, et ils peuvent
\^etre utilis\'es pour d\'ecrire la fonction d'onde vrai avec une
combinaison lin\'eaire des tous les d\'eterminants de Slater qui peuvent
\^etre cr\'e\'es avec un certain nombre d'orbitales monoelectroniques et
d'\'electrons.  Si la combinaison lin\'eaire comprend tous les
d\'eterminants de Slater, une fonction d'onde Full-CI a \'et\'e cr\'e\'ee
\beq
\Psi = \sumidx{I} c_I \Phi_I
\nonumber
\eeq
qui d\'ecrit le mieux r\'esultat quel'on peut obtenir avec une base
d'orbitales monoelectroniques. Le co\^ut computationnel n\'ecessaire pour
travailler avec une fonction d'onde Full-CI est tr\`es \'elev\'e, et donc
des approximations doivent \^etre appliqu\'ees.

Une premi\`ere approximation est le choix d'un seul d\'eterminant de Slater,
$\Psi = \Phi_0$, et on utilise une proc\'edure d'optimisation qui va am\'eliorer les
orbitales monoelectroniques de $\Phi_0$.  Cette approche, bien connue sous
le nom de technique Hartree-Fock (HF) ou Self Consistent Field (SCF), marche
bien seulement dans les cas des mol\'ecules \`a la g\'eom\'etrie
d'\'equilibre avec une couche ferm\'e d'\'electrons. 

Si on veut travailler avec des \'etats excit\'es, ou des distorsions de
g\'eom\'etrie, l'approximation d'Hartree-Fock est insuffisante, et plusieurs
d\'eterminants de Slater doivent \^etre consid\'er\'es dans l'expansion lineaire.
Dans ce cas, une fonction multir\'ef\'erence est d\'evelopp\'ee. La limite est
l'expansion Full-CI, mais des m\'ethodes alternatives de s\'election des
d\'eterminants de Slater existent pour r\'eduire le co\^ut computationnel et avoir
particulier propri\'et\'e sur la fonction finale. 

Par exemple, la s\'election Complete Active Space (CAS) partage les
orbitales dans trois groupes: core, actif et virtuelles. Les orbitales core
sont toujour occup\'ees en toutes les d\'eterminants de Slater
s\'electionn\'es, les orbitales virtuelles sont toujour vides, et les
orbitales actifs ont toutes les occupations possibles pour un certain nombre
d'\'electrons actifs dans les orbitales actifs. 

Une alg\`ebre particuli\'ere a \'et\'e d\'evelopp\'ee pour travailler avec
les d\'eterminants de Slater, appell\'ee deuxi\`eme quantification. Les r\`egles
principales sont pr\'esent\'ees dans cette th\`ese: deux operateurs
particuli\'ers ont \'et\'e d\'ecrits: l'operateur de cr\'eation $\constr{i}$
cr\'ee un \'electron dans l'orbitale $i$ du d\'eterminant de Slater, et
l'operateur de destruction $\destr{i}$ d\'etruit un \'electron dans
l'orbitale.

Enfin, des approches d'optimisation de la fonction d'onde sont
pr\'esent\'ees: l'approche Newton-Rhapson (NR) et l'approche Super-CI. 
L'approche Newton-Rhapson se base sur le d\'eveloppement de Taylor de
l'\'energie. 

L'approche Super-CI, utilis\'ee dans cette th\`ese pour optimiser une
fonction d'onde localis\'ee, se base sur le th\'eor\`eme de Brillouin pour
les cases multir\'ef\'erence: l'interaction entre une fonction d'onde
optimis\'ee et ses excitations mono\'electroniques via l'hamiltonien sont
nuls. L'advantage de l'approche Super-CI est qui converge toujour \`a un
minimum. Au contraire, l'approche Newton-Rhapson peut converger vers un
point de selle.
