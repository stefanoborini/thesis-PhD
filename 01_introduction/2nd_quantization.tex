\section{Second quantization}

\subsection*{Creation and annihilation operators}

A frequently used representation for Slater determinants is by
spinorbital occupation. An occupation vector $\ket{\mathbf{n}}$ can be defined with elements
$n_i$ which evaluate 1 if a given spinorbital is occupied by an electron, 0
if not occupied. Of course, the sum $\sum_{i} n_i$ evaluates the total
number $N$ of electrons in the molecular system. 

The occupation vector
represents a particular distribution of $N$ electrons in $m$ spinorbitals
and belongs to a so-called Fock space $F(m)$, an abstract linear vector
space defined only by the occupation number of the spinorbitals.

A particular vector of the Fock space is the vacuum state
\beq
\vacuum = \ket{0_{1},0_{2},0_{3},\ldots,0_{m}}
\eeq
and it represents the case where no spinorbital is occupied by electrons. The
vacuum vector defines a subspace $F(0,m)$ of the Fock space, spanned by the
single vector given above. In general the subspace $F(N,m)$ of the
Fock space has a higher dimensionality, being defined by all those vectors
obtained by distributing $N$ electrons in $m$ spinorbitals.

For a basis of orthogonal spinorbitals, the product between two occupation
vectors $\ket{n}$ and $\ket{k}$ is defined as
\beq
\integral{n}{k} = \prod_{i=1}^{m} \delta_{n_{i}, k_{i}}
\eeq
This product represents the overlap between Slater determinants, evaluating to
one when the electronic distribution is the same, otherwise zero.

Two operators, named creation and annihilation operators, can now be defined.
The creation operator $\constr{i}$ is defined as 
\beqa
\constr{i} \ket{n_{1}, n_{2}, \ldots, 0_{i}, \ldots, n_{m}} & = &
\Gamma\left(n\right)_i \ket{n_{1}, n_{2}, \ldots, 1_{i}, \ldots, n_{m}} \\
\constr{i} \ket{n_{1}, n_{2}, \ldots, 1_{i}, \ldots, n_{m}} & = & 0
\eeqa
This operator imposes the occupation of a previously unoccupied spinorbital
$i$ by changing its occupation number from 0 to 1. If the selected
spinorbital $i$ is already occupied, the result is zero.  This can be
explained knowing that a Slater determinant with the same spinorbital
included twice is zero, due to the Pauli exclusion principle.  The
$\Gamma\left(n\right)$ value is called \textit{phase factor} and is
defined as
\beq
\Gamma\left(n\right)_i = (-1)^{\sum_{j=1}^{i-1} n_j}
\eeq
The effect of the phase factor is to change the sign of the resulting
determinant if the number of occupied spinorbitals before $i$ is odd, or
leaving it unchanged if even.

The annihilation operator $\destr{i}$ removes an electron
from an occupied spinorbital, or produces zero if the spinorbital is
already unoccupied
\beqa
\destr{i} \ket{n_{1}, n_{2}, \ldots, 1_{i}, \ldots, n_{m}} & = &
\Gamma\left(n\right)_i \ket{n_{1}, n_{2}, \ldots, 0_{i}, \ldots, n_{m}} \\
\destr{i} \ket{n_{1}, n_{2}, \ldots, 0_{i}, \ldots, n_{m}} & = & 0
\eeqa
The creation operator can be proved to be the adjoint of the annihilation
operator, and viceversa.

Important considerations can be derived from these definitions:
every vector of the Fock space can be generated from the vacuum state by
application of an ordered product of creation operators
\beq
\ket{\mathbf{n}} = \prod_{i=1}^{N} \left(\constr{i}\right)^{n_{i}} \vacuum
\eeq
Another important fact is represented by anticommutation
relationships between the operators. It can be demonstrated that
\beqa
\anticomm{\constr{i}}{\constr{j}} = & \constr{i}\constr{j} +
\constr{j}\constr{i} & = 0 \\
\anticomm{\destr{i}}{\destr{j}} = & \destr{i} \destr{j} + \destr{j}
\destr{i} & = 0 \\
\anticomm{\constr{i}}{\destr{j}} = & \constr{i} \destr{j} + \destr{j}
\constr{i} & = \delta_{ij} 
\eeqa
These relationships allow the definition of an efficient algebra for
handling determinants, operators and interactions between them.

\subsection*{Operators in second quantization}

First quantization operators such as the kinetic operator or the
Hamiltonian operator can be described using the second quantization formalism. The main difference
between first and second quantization expression resides in the different structure of the second quantization approach,
which incorporates the effect of the spinorbital basis into the operator
itself, and not into the Slater determinant, which holds only information
about the occupation.

Operators in second quantization are expressed as a combination of products
of elementary creation and annihilation operators. For example, the kinetic
operator in first quantization is written as
\beq
\hat{f}_{1q} = \sum_{i=1}^{N} \hat{f}(x_i)
\eeq
where $x_i$ represents the spatial and spin coordinates of the electron $i$,
and the sum runs over the number of electrons. 
To express the same operator in second quantization, we should keep in mind
that no change in the number of electrons is performed and only one electron
at a time is involved. Therefore, the kinetic operator in second quantization
is represented as a linear combination of excitation operators 
\beq
\hat{f}_{2q} = \sum_{rs} f_{rs} \constr{r} \destr{s}
\eeq 
where the sum runs over all spinorbitals. The operator is invariant with
respect to the number of electrons, since this information is contained into
the Fock space vector. The $\constr{r} \destr{s}$ operator
effectively moves one electron from spinorbital $s$ to spinorbital $r$.

Following the rules for orthogonality and application of the operators given
above, the application of the second quantization
$\braket{n}{\hat{f}}{m}$ operator produces three cases
\begin{enumerate}
\item the $\ket{n}$ and $\ket{m}$ vectors represent the
same occupation 
\beq
\braket{n}{\hat{f}}{m} = \sum_{r} n_{r} f_{rr}
\eeq
\item the $\ket{n}$ and $\ket{m}$ differ in a single
occupation pair
\beqa
\ket{n} &=& \ket{n_1, \ldots, 1_i, \ldots, 0_j, \ldots, n_m} \\
\ket{m} &=& \ket{n_1, \ldots, 0_i, \ldots, 1_j, \ldots, n_m} \\
\braket{n}{\hat{f}}{m} &=& \Gamma(n_1)_j \Gamma(n_1)_i f_{ij} 
\eeqa
\item the $\ket{n}$ and $\ket{m}$ differ in more than one
occupation pairs
\beq
\braket{n}{\hat{f}}{m} = 0 
\eeq
\end{enumerate}
which reproduce the Slater rules between Slater determinants if we choose
\beq
f_{rs} = \int \psi_{r}^{*}(x) \hat{f}(x) \psi_{s}(x) dx
\eeq
Therefore, the second quantization representation of a one-electron operator
is a linear combination of single excitation operators, where the
coefficients are the values of the integrals given above.

The same strategy can be arranged for two-electron operators. A
two-electron operator gives non-zero matrix elements between two Slater
determinants which differ at most for two pairs of occupation values.
Considering again that no change in the number of electrons is performed by
the operator, it could be expanded as a linear combination
\beq
\hat{g} = \frac{1}{2} \sum_{ijkl} g_{ijkl} \constr{i} \constr{j} \destr{l} \destr{k}
\eeq
where, as usual, the $g_{ijkl}$ elements can be calculated evaluating the
matrix element
\beq
g_{ijkl} = \int \psi_i^{*}(x) \psi_k^{*}(x^{\prime}) \hat{g}(x,x^{\prime}) =
\braket{ij}{\hat{g}}{kl}
\psi_{j}(x) \psi_{l}(x^{\prime})
\eeq
From these results, a representation of the Hamiltonian operator in second
quantization can be devised
\beq
\label{eqn:hamil}
\ham = \sum_{ij} h_{ij} \constr{i} \destr{j} + \frac{1}{2} \sum_{ijkl}
g_{ijkl} \constr{i} \constr{j} \destr{l} \destr{k} 
\eeq
where $h_{ij}$ is the integral of kinetic energy plus the electron-nuclear
operators, and $g_{ijkl}$ is the electron-electron repulsion term.

Summarizing, the differences between first and second quantization operators
are: 
\begin{itemize}
\item In first quantization the Slater determinants depend on the orbital
basis, and the operators are independent of this information. In second
quantization the operators contain this information, and the Slater
determinants are represented only as occupation vectors in the Fock space,
without references to the basis set
\item In first quantization, the operators depend on the number of
electrons. In second quantization, they are independent of this parameter.
\item First quantization operators are exact operators, while second
quantization operators are projected operators.
\end{itemize}

\subsection*{Density matrix}

As we know, the expectation value of the energy for a normalized
wavefunction can be obtained by
\beq
E = \braket{\Psi}{\ham}{\Psi}
\eeq
As we saw in Sec. \ref{sec:variational}, we can define the wavefunction as
a linear combination of functions, in our case Slater determinants
\beq
\Psi = \sum_i c_i \Phi_i
\eeq
If we substitute we obtain
\beq
E = \braket{\sum_i c_i \Phi_i}{\ham}{\sum_j c_j \Phi_j} = \sum_{ij} c_i c_j
H_{ij}
\eeq
Where $H_{ij}$ is the matrix element of the Hamiltonian operator between two
Slater determinants. We can expand further this expression keeping into
account Eqn. \ref{eqn:hamil}
\beq
E = \sum_{ij} c_i c_j \left( \sum_{ab} \gamma_{ab}^{ij} h_{ab} + \frac{1}{2}
\sum_{abcd} \Gamma_{abcd}^{ij} g_{abcd} \right)
\eeq
where $\gamma_{ab}^{ij} = \braket{\Phi_i}{\constr{a} \destr{b}}{\Phi_j}$ and
$\Gamma_{abcd}^{ij} = \braket{\Phi_i}{\constr{a} \constr{b} \destr{d}
\destr{c}}{\Phi_j}$.
We can remove the dependency on indexes $i$ and $j$ by including the
coefficients and defining two entities: the one-particle density matrix
\beq
\gamma_{ab} = \sum_{ij} c_i c_j \gamma_{ab}^{ij}
\eeq
and the two-particles density matrix
\beq
\gamma_{abcd} = \sum_{ij} c_i c_j \Gamma_{abcd}^{ij}
\eeq
The energy can now be written as  
\beq
E = \sum_{ab} \gamma_{ab} h_{ab} + \frac{1}{2} \sum_{abcd} \gamma_{abcd} g_{abcd} 
\eeq

These matrixes are central in modern quantum chemistry. An interesting
property of the one-particle density matrix can be found in the sum of the
diagonal elements (trace), which adds to the total number of electrons
\beq
\mbox{Tr}(\gamma_{ij}) = N
\eeq
The eigenvectors of the density matrix are called \textit{natural orbitals}.
The corresponding eigenvalues are called \textit{occupation numbers}, and
express the electronic occupation of the corresponding natural orbital.

