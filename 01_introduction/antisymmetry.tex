\section{Antisymmetry}

A natural law imposes that fermionic particles like electrons must be
described by a wavefunction which is antisymmetric with respect to the
exchange of two particles. For a simple two-electron wavefunction the
relationship
\beq
\Psi(2,1) = - \Psi(1,2)
\eeq
must be satisfied.

It is always possible to express a polyelectronic wavefunction as a
combination of products of one-electron spinorbitals.
If $\Psi(1,2,\ldots,N)$ is a N-electrons
wavefunction, the spatial position of particles $(2,\ldots,N)$ can be
fixed. The resulting function can be expanded on a single-particle
basis set $\left\{ \psi_i \right\} $
\beq
\Psi(1,\underbrace{2,\ldots,N}_{\mbox{fixed}}) = \sumidx{i} c_i(2,\ldots,N) \psi_i(1)
\eeq
where the $c_i$ coefficients hold the dependency against the fixed
particles, therefore being functions themselves. The procedure can 
be repeated for $c_i(2,\ldots,N)$, fixing $(3,\ldots,N)$ and expanding the
function on the same basis set
\beq
c_i(2,\underbrace{3,\ldots,N}_{\mbox{fixed}}) = \sumidx{j}
c^{\prime}_{ij}(3,\ldots,N) \psi_j(2)
\eeq
Repeating for the coordinates of $N$ particles, we obtain
\beq
\label{eqn:antisym_1}
\Psi(1,2,\ldots,N) = \sumidx{i,j,k,\ldots,l} c_{ijk\ldots l} \psii(1)
\psij(2) \psik(3) \ldots \psi_l(N)
\eeq
where $c_{ijk\ldots l}$ is a purely numeric factor. Applying Eqn.
\ref{eqn:antisym_1} to a two-particles case, we obtain
\beqa
\Psi(1,2) &=& \sumidx{i,j} c_{ij} \psii(1) \psij(2) \nonumber \\
          &=& c_{11} \psi_1(1) \psi_1(2) + c_{12} \psi_1(1) \psi_2(2) + c_{21} \psi_2(1) \psi_1(2) \nonumber \\
	  &&  + c_{22} \psi_2(1) \psi_2(2) + \ldots
\eeqa
Describing a system with the electrons swapped
\beqa
\Psi(2,1) &=& c_{11} \psi_1(2) \psi_1(1) + c_{12} \psi_1(2) \psi_2(1) + c_{21} \psi_2(2) \psi_1(1) \nonumber \\
	  &&  + c_{22} \psi_2(2) \psi_2(1) + \ldots
\eeqa
in order to satisfy the antisymmetry rule the following conditions must hold
\beqa
c_{ii} &=& - c_{ii} \rightarrow c_{ii} = 0 \nonumber \\
c_{ij} &=& - c_{ji} \nonumber
\eeqa
Therefore, the sum can be reduced to
\beqa
\Psi(1,2) &=& c_{12} \left( \psi_1(1) \psi_2(2) - \psi_2(1) \psi_1(2) \right) \nonumber \\
	  && + c_{13} \left( \psi_1(1) \psi_3(2) - \psi_3(1) \psi_1(2) \right) + \ldots \nonumber \\
	  &=& \sumidx{i<j} c_{ij} \left( \psi_i(1) \psi_j(2) - \psi_j(1) \psi_i(2) \right) \nonumber \\
	  &\rightarrow& \sumidx{i<j} c_{ij} \detsl{\psii \psij}
\eeqa
where
\beq
\detsl{\psii \psij} = 2^{-\frac{1}{2}} \left|
\begin{array}{cc}
\psii(1) & \psij(1) \\
\psii(2) & \psij(2) \\
\end{array}
\right|
\eeq
is the normalized \textit{Slater determinant}. 
The presented scheme can be generalized for a $N$ electron case
\beqa
\Psi(1,2,\ldots,N) &=& \sumidx{i_1 < i_2 < \ldots < i_N }
c_{i_1,i_2,\ldots,i_N} \detsl{\psi_{{i}_1} \psi_{{i}_2} \ldots \psi_{{i}_N}} \\
&=& \sumidx{I} c_I \Phi_I
\eeqa
where 
\beqa
\detsl{\psi_{{i}_1} \psi_{{i}_2} \ldots \psi_{{i}_N}} &=&
(N!)^{-\frac{1}{2}} \left|
\begin{array}{cccc}
\psi_{{i}_1}(1) & \psi_{{i}_2}(1) & \ldots & \psi_{{i}_N}(1) \\
\psi_{{i}_1}(2) & \psi_{{i}_2}(2) & \ldots & \psi_{{i}_N}(2) \\
\vdots          &  \vdots         & \ddots &  \vdots          \\
\psi_{{i}_1}(N) & \psi_{{i}_2}(N) & \ldots & \psi_{{i}_N}(N) \\
\end{array}
\right|
\eeqa

The wavefunction can be expressed as a linear combination of Slater
determinants, each one describing a possible electronic distribution.
Slater determinants satisfy by construction the
antisymmetry requisite: exchanging two particles is expressed as a column
swap, which changes the sign of the resulting determinant.
Another interesting property, Pauli's rule, is a direct consequence of the
antisymmetry requisite: if two particles are forced to occupy the same
one-electron function, the representative determinant is 
\beqa
\detsl{\psi_{{i}_1} \psi_{{i}_1} \ldots \psi_{{i}_N}} =
(N!)^{-\frac{1}{2}}
\left|
\begin{array}{cccc}
\psi_{{i}_1} (1) & \psi_{{i}_1} (1) & \ldots & \psi_{{i}_N} (1) \\
\psi_{{i}_1} (2) & \psi_{{i}_1} (2) & \ldots & \psi_{{i}_N} (2) \\
\vdots           &   \vdots         & \ddots &  \vdots          \\
\psi_{{i}_1} (N) & \psi_{{i}_1} (N) & \ldots & \psi_{{i}_N} (N) \\
\end{array}
\right|
\eeqa
which evaluates as zero (two columns are equal).
