\section{Perturbation theory}

Let $\ham$ be the true Hamiltonian and let us suppose it can be expressed in
the form 
\beq
\ham = \ham_{0} + \lambda \hat{V}
\eeq
where $\ham_{0}$ is a model Hamiltonian, $\lambda$ is a parametric value
expressing the intensity of the perturbative effect and $\hat{V}$ is a
perturbative operator. 

Supposing a non-degenerate case and that all eigenvalues and eigenvectors of
the model Hamiltonian $\ham_{0}$ are known (therefore knowing its spectral
decomposition) we have, at zero order of perturbation
\beq
\ham_{0} \Psi_n^{(0)} = E_n^{(0)} \Psi_n^{(0)} \quad n=0,1,2,\ldots
\eeq
the $\Psi_n^{(0)}$ eigenfunctions ($\ket{n}$ in shorter notation) define a
complete basis set of wavefunctions. If no perturbative effects are present,
the descriptive wavefunction for the state $n$ will be $\Psi_n^{(0)}$ with
$E_n^{(0)}$ as associated energy.

Imposing a perturbation, the descriptive wavefunction and the associated
energy will change as a function of the perturbative parameter $\lambda$.
We can perform a Taylor's expansion to obtain
\beqa
\Psi_n &=& \ket{n} + \lambda \Psi_n^{(1)} + \lambda^2 \Psi_n^{(2)} +
\ldots \\
E_n &=& E_n^{(0)} + \lambda E_n^{(1)} + \lambda^2 E_n^{(2)} + \ldots
\eeqa
We suppose that the eigenfunctions of $\ham_{0}$ are normalized
($\integral{n}{n} = 1$) and we choose the
normalization of $\Psi_n$  such that $\integral{n}{\Psi_n} = 1$
(\textit{intermediate normalization}). This normalization can always be performed
when $\ket{n}$ and $\Psi_n$ are not orthogonal, simplifying the
treatment and leading to 
\beq
\integralmodif{n}{\Psi_n^{(k)}} = 0 \quad k = 1,2,3,\ldots
\eeq
Given the final relationship to solve
\beq
\ham \Psi_n = E_n \Psi_n
\eeq
a substitution can be performed to obtain
\beqa
& & \lambda^0 \left( \ham_{0} \ket{n} - E_n^{(0)} \ket{n} \right) \nonumber \\ 
&+& \lambda^1 \left( \ham_{0} \Psi_n^{(1)} + \hat{V} \ket{n} - E_n^{(0)} \Psi_n^{(1)} - E_n^{(1)} \ket{n} \right) \nonumber \\
&+& \lambda^2 \left( \ham_{0}\Psi_n^{(2)} + \hat{V}\Psi_n^{(1)} - E_n^{(0)} \Psi_n^{(2)} - E_n^{(1)} \Psi_n^{(1)} - E_n^{(2)} \ket{n} \right) \nonumber \\
&+& \ldots = 0 \label{eqn:perturb_1}
\eeqa
In order to satisfy \ref{eqn:perturb_1}, terms inside parentheses must be zero
\beqa
\ham_{0} \ket{n} &=& E_n^{(0)} \ket{n} \\
\label{eqn:perturb_1_2}
\left( \ham_{0} - E_n^{(0)} \right) \Psi_n^{(1)} &=& \left( E_n^{(1)} -
\hat{V} \right) \ket{n} \\
\left( \ham_{0} - E_n^{(0)} \right) \Psi_n^{(2)} &=& \! E_n^{(2)} \ket{n} \! + \left( E_n^{(1)} - \hat{V} \right) \Psi_n^{(1)} \\
& \vdots & \nonumber
\eeqa
which also lead to
\beqa
E_n^{(0)} &=& \braket{n}{\ham_{0}}{n} \\ 
E_n^{(1)} &=& \braket{n}{\hat{V}}{n} \\
\label{eqn:perturb_vbgh}
E_n^{(2)} &=& \braket{n}{\hat{V}}{\Psi_n^{(1)}} 
\eeqa

We can find the first-order correction to the wavefunction by expressing the
$\Psi_n^{(1)}$ on the complete basis set defined by the solutions of the zero-order
Hamiltonian 
\beq
\label{eqn:perturb_2}
\Psi_n^{(1)} = \sumidx{k}c_k^{(1)} \ket{k}
\eeq
It is clear that, since the eigenfunctions of $\ham_{0}$ are orthonormal,
we can write
\beq
\integralmodif{k}{\Psi_n^{(1)}} = c_k^{(1)}
\eeq
and then
\beqa
\Psi_n^{(1)} = \sumidx{k}c_k^{(1)} \ket{k} \\
\label{eqn:perturb_bbb}
\Psi_n^{(1)} = \sum_{k} \ket{k} \integralmodif{k}{\Psi_n^{(1)}}
\eeqa
We can now multiply \ref{eqn:perturb_1_2} by $\bra{k}$ 
\beq
\label{eqn:perturb_xyb}
\left( E_n^{(0)} - E_k^{(0)} \right) \integralmodif{k}{\Psi_n^{(1)}} =
\braket{k}{\hat{V}}{n} \\
\eeq
and combining \ref{eqn:perturb_bbb} with \ref{eqn:perturb_vbgh}
\beq
E_n^{(2)} = \braket{n}{\hat{V}}{\Psi_n^{(1)}} = \sum_{k}
\braket{n}{\hat{V}}{k} \integralmodif{k}{\Psi_n^{(1)}}
\eeq
combining with \ref{eqn:perturb_xyb} leads to the second-order correction to
the energy
\beq
E_n^{(2)} = \sum_{k} \frac{\left| \braket{n}{\hat{V}}{k}
\right|^2}{E_n^{(0)} - E_k^{(0)}}
\eeq

