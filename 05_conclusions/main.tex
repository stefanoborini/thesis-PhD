\thispagestyle{empty}
{ \Huge \textbf{Conclusions} } 
\vspace{1mm} \\

This work surveyed two innovative techniques dedicated to the
\textit{ab initio} study of molecular systems.

The localization technique makes use of an optimization procedure that
preserves the locality of a starting guess.  An extension of this technique,
named Freeze-and-Cut, allows the reduction of the computational cost by
performing a neglection of those parts of the molecular system not involved
in the phenomenon under study. The localized molecular orbitals are
frozen at a lower level of theory (SCF in this case) to keep into account
their effect. The remaining orbitals are projected onto a subset of the
original basis set, allowing a reduced AO/MO two-electron integrals
transformation. 

The method proved its efficiency on two molecular system: the aminoacid
zwitterion (7Z)-13 ammoniotridec-7-enoate, specifically designed to
test the response of the method to charge interactions, and the highly
conjugated C$_{13}$ polyenal.  A third test on the acetone plus 6 H$_2$O was
attempted to evaluate excited states. In this case, a higher dimensionality
of the basis set is needed to produce effective results, but this attempt
reported a sensivity of the Freeze-and-Cut method to diffuse basis sets. 

A possible solution to this problem is the use of non-orthogonal orbitals between the frozen
and non-frozen sets, and although it seems to produce stable results, the
current procedure have to be reconsidered with the assumption of
non-orthogonality. Further developments are needed to face this
improvement.

The second technique studied during this Ph.D. is the n-Electron Valence
state Perturbation Theory (NEVPT). NEVPT is a multireference perturbation
theory applicable to CAS wavefunctions. It is based on a two-electron
Hamiltonian, the Dyall Hamiltonian, which provides a straightforward and
efficient development for both theoretical and computational implementation.
The perturber wavefunctions are multireference in nature, and belongs to
eight excitation classes, depending on the promotion scheme of the
electrons. 

NEVPT is available in two accuracy levels, depending on the degree of
contraction of the perturber space: the Strongly Contracted NEVPT uses a
monodimensional space for each class, while the Partially Contracted uses a
space of higher dimensionality but smaller than the complete space. The
Strongly Contracted approach produces results of good accuracy when compared
to the Partially Contracted, and a difference between the two approaches has
been noted as a symptom of a poor zero-order wavefunction.

NEVPT is available as a Single-State approach, where the perturbation is
applied to a specified electronic state, and also as a Quasi-Degenerate
approach, where the perturbation effects are applied simultaneously to a set
of states. The Quasi-Degenerate approach keeps into account the
interactions between wavefunctions after the perturbative corrections,
allowing mixing of the states.

Finally, a variation of NEVPT named Non-Canonical has been deployed to
obtain invariability for intraclass orbital rotation for core and virtual
orbitals. This approach removes the need to perform the evaluation on
canonical orbitals, allowing the direct application of NEVPT on a set of
localized orbitals.

The various aspects of NEVPT has been applied to a large set of cases, all
involving the carbonyl system. Every performed evaluation highlighted the
strengths of this method, in particular its computational efficiency, its
accuracy and the absence of intruder states.  In the Quasi-Degenerate NEVPT
further developments can be applied to open the possibility toward an
iterative treatment. An attempt has been successfully made to deploy an
integrated Localized+NEVPT Non-Canonical evaluation. This will open the
possibility to perform highly accurate evaluations on large molecular
systems.

Finally, a collaboration with the CINECA supercomputing center lead to
accurate insights in the problems related to the deployment of a grid
infrastructure for quantum chemistry evaluations. Two problems have been
recognized as rather central in the first phase of the project: the need of
a library for parsing XML with the Fortran language, and a common file
format for large binary data.

The first problem has been addressed with the development of the F90xml library.
This library provides a DOM compliant interface to the Fortran environment,
acting as a wrapper of the gdome2 library.

The second problem involved an accurate analysis of previous common formats
in the field, keeping into account a very large set of new requisites, such
as platform independence, ease of use, debugging accuracy, and
extensibility.  This research lead to the deployment of a data model which
has been implemented in the high-level library Q5Cost. This library proved
its effectiveness allowing various codes to share informations, thus opening
new possibilities in the integration of methods and the development of a
grid metasystem. The library also comes with a proficient testsuite, aimed
at stressing the interface for unexpected error conditions. The result is a
very accurate, extensible and solid library. 

The main drawbacks of this library can be found in the followings: the
envisioned data model at the moment is not completely implemented in the
interface, and the data storage for integrals can be made more efficient in
terms of space occupation. In any case, the current implementation is
optimal for a first-time experimentation in a new panorama for the
computational quantum chemistry, and the detailed informations about the
indexes of the stored integrals can be removed, providing files of reduced
size.

\clearpage
\thispagestyle{empty}
{ \Huge \textbf{Conclusions} } 
\vspace{1mm} \\
\thispagestyle{empty}

Ce travail a regard\'e deux nouvelles techniques pour l'\'etude \textit{ab
initio} des syst\`emes mol\'eculaires. La technique de localisation utilise
une procedure d'optimisation qui preserve la localit\'e d'un groupe
d'orbitales de guess localis\'ee. Une extension de cette technique, la
technique Freeze-and-Cut, permet la r\'eduction du co\^ut computationnel
apr\`es \'elimination de la partie du syst\`eme  mol\'eculaire pas
int\'eress\'ee au ph\'enom\`ene \'etudi\'e.
Les orbitales localis\'ees sont gel\'ees \`a un niveau plus bas de th\'eorie
(dans notre cas, une \'evaluation SCF) et leur effect est preserv\'e. 
Pour r\'eduire le poids computationnel pour la transformation des integrales
bielectroniques, les autres orbitales sont project\'ees sur un sous-ensemble
de la bas atomique original. Cette m\'ethode a \'et\'e essay\'e sur deux
syst\`emes mol\'eculaires: le (7Z)-13 ammoniotridec-7-enoate, un aminoacide
construit pour \'evaluer la r\'eponse de la m\'ethode al'interaction de
charge, et le polyenal C13.
Une troisi\`eme \'evaluation a \'et\'e effectu\'e sur l'etat excit\'e $n \rightarrow
\pi^{*}$ de la molecule d'acetone avec 6 molecules d'eau. Dans ce cas, la
dimensionalit\'e plus grande de la bas atomique a d\'emontr\'e une
sensibilit\'e de la m\'ethode Freeze-and-Cut \`a des bases diffuses. Une solution
peut \^etre l'utilise de orbitales non-ortogonaux entre le groupe gel\'e et
non-gel\'e. Cette solution donne des r\'esultats plus corrects, mais la
procedure courant doit \^etre adapt\'ee avec l'assumption de non
orthogonalit\'e, et donc autres d\'eveloppements theoretiques sont
n\'ecessaires.

La deuxi\'eme technique etudi\'ee pendant ce travail de These est la th\'eorie
perturbative ``n-Electron Valence state Perturbation Theory'' (NEVPT). La NEVPT
est une th\'eorie perturbative multir\'ef\'erence qui peut \^etre
appliqu\'ee
\`a des fonction d'onde de type CAS. Le d\'eveloppement theoretique utilise
un Hamiltonien bielectronique, l'Hamiltonien de Dyall, qui donne une
th\'eorie propre et efficiente. Les fonctions perturbatives sont de nature
multir\'ef\'erencielle, et peuvent \^etre classifi\'ee dans huit classes d'excitation,
selon le sch\'ema de promotion des electrons. La NEVPT existe dans
deux versions, la Strongly Contracted et la Partially
contracted, selon le niveau de contraction de l'espace perturbatif. La
Strongly Contracted NEVPT utilise un espace monodimensionnel pour chaque
classe, et la Partially Contracted utilise un espace \`a dimensionalit\'e plus
haute, mais plus petite que l'espace complet des determinants excit\'es.
L'approche Strongly Contracted donne des r\'esultats de bonne qualit\'e si
compar\'ee avec la Partially Contracted. Une diff\'erence entre eux est
normalement un sympt\^ome d'une mauvaise d\'escription \`a l'ordre-zero.

La NEVPT est disponible comme single \'etat, o\'u la perturbation est
appliqu\'ee \`a un particulier \'etat \'electronique. La NEVPT Quasi Degenerate
applique l'effet perturbatif sur un ensemble d'\'etats \`a la fois. Cette
approche consid\`ere les interactions entre les fonctions d'onde apr\`es la
perturbation, et permit le m\'elange de ce fonctions.
Enfin, une variation de la NEVPT appell\'ee Non Canonical a \'et\'e
d\'evelopp\'ee pour obtenir l'invariabilit\'e des r\'esultats pour le
m\'elange des orbitales de core et virtuelles. Cette approche \'elimine la
n\'ecessit\'e de travailler avec des orbitales canoniques, et permit
l'utilisation directe avec un ensemble d'orbitales localis\'ees.

Les diff\'erents types de NEVPT ont \'et\'e utilis\'es sur un ensemble de
syst\'emes concernant le group carbonyl. Toutes les \'evaluations effectu\'ees
donnent des r\'esultats int\'eressants, et sont aussi inter\'essantes les
propri\'et\'es de la NEVPT, comme l'efficacit\'e computationnelle et l'absence
d'\'etats intrus.
Les r\'esultats de la th\'eorie QD-NEVPT peuvent \^etre amelior\'ees avec une
procedure iterative, qui peut \^etre une bonne solution \`a des
probl\'emes dans les r\'esultats obtenus.
L'integration de la th\'eorie NEVPT avec l'approche localis\'ee a \'et\'e
effectu\'ee avec la NEVPT Non Canonique. Cette travail sera important pour
effectuer des calculs sur syst\'emes mol\'eculaires de grande taille.

Une collaboration avec le centre de supercalcul CINECA a \'et\'e aussi
effectu\'ee dans le cadre d'un projet international entre diff\'erent
instituts de recherche en chimie quantique. L'objectif est l'integration des 
diff\'erentes m\'ethodologies \textit{ab initio} dans une r\'eseau d'ordinateurs.
Deux probl\`emes sont prioritaires dans la phase initial du projet: la
n\'ecessit\'e d'une bibliotheque pour la lecture du format de fichier XML avec
le langage Fortran, et le d\'eveloppement d'un format de fichier binaire
commun pour le stockage de donnees de grand taille.
Pour le premier probl\`eme, la biblioth\`eque F90xml a \'et\'e
d\'evelopp\'ee. Cette biblioth\`eque interface libgdome2 (biblioth\`eque
pour le langage C) avec Fortran. 

Pour le deuxi\`eme probl\`eme, on a analis\'e le format de fichier dej\`a
d\'evelopp\'e et evalu\'e leur d\'efects pour l'ex\'ecution en grid, qui
n\'ecessite de consid\'erations additionnelles, comme
ind\'ependance de plateforme, simplicit\'e d'utilise de la biblioth\`eque 
d'acces, exactitude des messages d'erreur et extensibilit\'e.
La recherche effectu\'ee a conduit au d\'eveloppement d'un mod\`ele des
donn\'ees qui a \'et\'e impl\'ement\'e dans la biblioth\`eque Q5Cost. Cette
biblioth\`eque a \'et\'e utilis\'ee pour l'integration des diff\'erents logiciels,
et ouvre des nouvelles possibilit\'es pour le partage d'information
quantistique et l'integration des m\'ethodes \textit{ab initio}.
La biblioth\`eque a aussi une testsuite pour contr\^oler l'interface et
chercher les probl\`emes dans l'implementation. 
Le r\'esultat est une biblioth\`eque tr\`es solide et extensible.  Le
probl\`emes principaux qui restent \`a resoudre dans Q5Cost sont: le
stockage des integrals n'est pas encore efficient en terms d'espace
occup\'e, et l'implementation du mod\`ele des donn\'ees n'est pas complete,
mais l'impl\'ementation courante est suffisante pour le premiers
d\'eveloppements et implementations.
