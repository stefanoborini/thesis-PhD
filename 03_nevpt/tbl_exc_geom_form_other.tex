\begin{center}
\begin{threeparttable}
\footnotesize
\begin{tabular*}{\textwidth}{l@{\hspace*{43mm}}cccc}
\hline
Method        &R$_{\rm CO}$ & R$_{\rm CH}$ & $\angle$HCH & $\theta$\tnote{a} \\
\hline
\multicolumn{5}{c}{\small $S_2$ excited state}\\
CASSCF(CAS(A))\tnote{b} (I) &  1.564  &  1.082  &   116.2   &  48.3 \\
CASSCF(CAS(C))\tnote{b} (I) &  1.535  &  1.112  &   113.2   &  44.6 \\
CASSCF(CAS(A))\tnote{b} (II) &  1.356  &  1.140  &    91.0   &  76.5 \\
CASSCF(CAS(C))\tnote{b} (II) &  1.257  &  1.264  &    55.7   &  66.8 \\
MRD-CI\cite{jcsft-90-683-1994} &  1.528  &  1.080  &   111.3   &  $\sim$44  \\
EOM-CCSD\cite{jpca-106-4192-2002} &  1.583  &  1.095  &   119.6   &   0.0 \\
CASPT2\cite{tca-92-227-1995}  &  1.492  &  1.094  &   118.4   &  46.2 \\
CIS\cite{jpc-96-135-1992}  &  1.460  &  1.073  &   124.4   &   0.0 \\
MR-CISD\cite{jcp-114-746-2001} &  1.505  &  1.082  &   117.5   &  45.8 \\
MR-AQCC\cite{jcp-114-746-2001} &  1.495  &  1.088  &   118.9   &  45.4 \\
\multicolumn{5}{c}{\small \tpipi\ excited state}\\
CASSCF(CAS(A))\tnote{b} &  1.474  &  1.077  &   120.0   &  35.2 \\
CASSCF(CAS(C))\tnote{b} &  1.465  &  1.102  &   119.0   &  32.2 \\
MRD-CI\cite{jpc-99-16576-1995} &  1.476  &  1.075  &   119.9   &  38.1 \\
MP2\cite{jpc-99-16576-1995} &  1.451  &  1.082  &   121.3   &  28.9 \\
CIS\cite{jpc-96-135-1992}&  1.408  &  1.072  &   119.2   &  39.4 \\
Experimental\tnote{c} &  1.423  &         &           &       \\
\multicolumn{5}{c}{\small \sspi\ excited state}\\
CIS\cite{jpc-97-4293-1993}  &  1.489  &  1.079  &  121.9   &  46   \\
MRD-CI\cite{jms-176-375-1996} &  1.529  &  1.080  &  115.1   &  42.1  \\
\multicolumn{5}{c}{\small \tspi\ excited state}\\
CASSCF(CAS(A))\tnote{b} &  1.499  &  1.073  &  129.4   &  46.9 \\
CASSCF(CAS(C))\tnote{b} &  1.483  &  1.101  &  128.7   &  45.2 \\
MRD-CI\cite{jpc-99-16576-1995} &  1.455  &  1.090  &  133.5    & 35.5 \\
\hline
\end{tabular*}
\caption{\footnotesize Equilibrium geometries for the $S_2$ (the \spipi\ and \sspi\ configurations are mixed),
\tpipi\ and \tspi\ states of formaldehyde.
Distances in \AA, angles in degrees.}\label{tbl:exc_geom_form_other}
\begin{tablenotes}
\footnotesize
\item[a] Out-of-plane bending angle.
\item[b] This work. I and II indicate two different minima found on the
$S_2$ energy surface.
\item[c] Electron-impact spectroscopy, Ref. \citen{cp-70-291-1982} as cited
in Ref. \citen{jpc-99-16576-1995}.
\end{tablenotes}
\end{threeparttable}
\end{center}
