\section{Quasi-Degenerate NEVPT2}

Single-State NEVPT2, presented in the previous sections, performs the
perturbative treatment on a given state, disregarding possible interaction
between multiple states which are nearly degenerate before or after the
perturbation. This is due to the definition of a state-specific zero-order
Hamiltonian. To address the needs for quasi-degeneracy problems, frequently
encountered when analyzing avoided crossings, the Quasi-Degenerate NEVPT2
(QD-NEVPT) approach has been implemented\cite{jcp-121-4043-2004}. 

The approach used for NEVPT follows the description made by
Lindgren\cite{jpb-7-2441-1974}.  We define a model space, as generated by a
set of $n$ eigenfunctions, in our case of CASSCF type $\left\{ \Psi_1^{(0)},
\Psi_2^{(0)},\ldots, \Psi_n^{(0)} \right\}$ which undergo altogether the
perturbative correction.  These functions define a space $\mathcal{P}$, and
have associated zero-order energies $E_i^{(0)}$.  We can also define the
complementary space $\mathcal{Q} = 1 - \mathcal{P}$ with
the set of functions $\left\{ \Psi_{n+1}^{(0)}, \Psi_{n+2}^{(0)}, \ldots
\right\}$. Projection operators can be defined for each space
\beqa
\hat{\mathcal{P}} &=& \sum_{i=1}^{n} \ket{\Psi_{i}^{(0)}} \bra{\Psi_{i}^{(0)}} \\
\hat{\mathcal{Q}} &=& \sum_{a > n} \ket{\Psi_{a}^{(0)}} \bra{\Psi_{a}^{(0)}} 
\eeqa
It can be demonstrated that the following relationships hold
\beqa
\label{eqn:qdpt_4}
\hat{\mathcal{P}}^2 = \hat{\mathcal{P}}^{+} = \hat{\mathcal{P}} \\
\hat{\mathcal{P}} \hat{\mathcal{Q}} = \hat{\mathcal{Q}} \hat{\mathcal{P}} = 0 
\eeqa
%\commut{\hat{\mathcal{P}}}{\ham_0} = \commut{\hat{\mathcal{Q}}}{\ham_0} = 0
Defining the true eigenfunctions of the Schr\"odinger equation as
\beq
\ham \Psi_i = E_i \Psi_i
\eeq
we can now write the projection of the true eigenfunctions inside the
spaces defined above
\beqa
\tilde{\Psi}_i &=& \hat{\mathcal{P}} \Psi_i \qquad \forall 1 \le i \le n \\ 
\tilde{\Psi}_a &=& \hat{\mathcal{Q}} \Psi_a \qquad \forall a > n 
\eeqa
and define a ``wave operator'' $\hat{\Omega}$ which produces the true
eigenfunctions $\Psi_i$ from the projected ones
\beq
\hat{\Omega}\tilde{\Psi}_i = \Psi_i 
\eeq
but produces zero when applied to the complementary space
\beq
\hat{\Omega}\tilde{\Psi}_a = 0
\eeq
It can be proved that the following relationships hold
\beqa
\hat{\mathcal{P}} \hat{\Omega} &=& \hat{\mathcal{P}} \\
\hat{\Omega} \hat{\mathcal{P}} &=& \hat{\Omega} \\
\hat{\Omega} \hat{\mathcal{Q}} &=& 0
\eeqa
It is possible to rewrite the Schr\"odinger equation in the form
\beq
\label{eqn:qdpt_2}
\ham \hat{\Omega} \tilde{\Psi}_i = E_i \hat{\Omega} \tilde{\Psi}_i
\eeq
and applying $\hat{\mathcal{P}}$ on both sides
\beqa
\hat{\mathcal{P}} \ham \hat{\Omega} \tilde{\Psi}_i &=& E_i \tilde{\Psi}_i \\
\label{eqn:qdpt_1}
\ham_{\mbox{\tiny eff}} \tilde{\Psi}_i &=& E_i \tilde{\Psi}_i
\eeqa
where an effective Hamiltonian $\ham_{\mbox{\tiny eff}} = \hat{\mathcal{P}} \ham
\hat{\Omega}$ has been defined.  This Hamiltonian produces the eigenvalues
$E_i$ of the true Hamiltonian, but operates in the model space.


The final goal of this technique is to obtain the matrix
$\braket{\Psi_n^{(0)}}{\ham_{\mbox{\tiny eff}}}{\Psi_m^{(0)}}$ and diagonalize it.
This matrix has the dimensionality of the number of states considered in the
Quasi-Degenerate scheme. Once diagonalized, it provides the corrected
energies and the coefficients mixing the perturbed wavefunctions.

To accomplish this task, further elaborations are needed: making use of
Eqn.~\ref{eqn:qdpt_1} by applying on both sides
$\hat{\Omega}$ and subtracting \ref{eqn:qdpt_2} we obtain
\beq
\left( \hat{\Omega} \ham_{\mbox{\tiny eff}} - \ham \hat{\Omega} \right)
\tilde{\Psi}_i = 0
\eeq
which is valid also for $\tilde{\Psi}_a$, therefore the operator itself is
zero
\beq
\hat{\Omega} \ham_{\mbox{\tiny eff}} - \ham \hat{\Omega} = 0
\eeq
Replacing the definition of $\ham_{\mbox{eff}}$ we obtain the
generalized Bloch equation
\beq
\hat{\Omega} \hat{\mathcal{P}} \ham \hat{\Omega} - \ham \hat{\Omega} = 0
\eeq

Introducing the expression of the Hamiltonian as $\ham = \ham_0 + \hat{V}$
we can now perform a substitution inside the Bloch equation to obtain 
\beq
\commut{\hat{\Omega}}{\ham_0} = \hat{\mathcal{Q}} \hat{V} \hat{\Omega} -
\hat{\mathcal{Q}} \hat{\Omega} \hat{\mathcal{P}} \hat{V} \hat{\Omega}
\eeq
To proceed further in the development, a perturbative expansion for the wave
operator and the effective Hamiltonian is performed
\beq
\hat{\Omega} = \hat{\mathcal{P}} + \hat{\Omega}^{(1)} + \hat{\Omega}^{(2)} + \dots
\eeq
and the following relationships can be obtained
\beqa
\commut{\hat{\Omega}^{(1)}}{\ham_0} &=& \hat{\mathcal{Q}} \hat{V}
\hat{\mathcal{P}} \label{eqn:qdpt_6}\\
\commut{\hat{\Omega}^{(2)}}{\ham_0} &=& \hat{\mathcal{Q}} \hat{V}
\hat{\Omega}^{(1)} - \hat{\mathcal{Q}} \hat{\Omega}^{(1)} \hat{\mathcal{P}}
\hat{V} \hat{\mathcal{P}} \\
&\vdots& \nonumber
\eeqa 

Applying the commutator in \ref{eqn:qdpt_6} to $\ket{\Psi_i}$ and
keeping into account that $\hat{\mathcal{Q}} \hat{\Omega}^{(1)} =
\hat{\Omega}^{(1)}$ we obtain
\beq
\left( E_i^{(0)} - \hat{\mathcal{Q}} \ham_0 \hat{\mathcal{Q}} \right)
\hat{\Omega}^{(1)} \ket{\Psi_i} = \hat{\mathcal{Q}} \hat{V} \ket{\Psi_i}
\eeq
which can be rearranged to produce
\beq
\hat{\Omega}^{(1)} \ket{\Psi_i^{(0)}} = \left( E_i^{(0)} - \hat{\mathcal{Q}}
\ham_0 \hat{\mathcal{Q}} \right)^{-1} \hat{\mathcal{Q}} \hat{V}
\ket{\Psi_i^{(0)}}
\eeq
and knowing that
\beq
\hat{\mathcal{Q}} \ham_0 \hat{\mathcal{Q}} = \sum_{a} \ket{\Psi_a^{(0)}}
E_a^{(0)} \bra{\Psi_a^{(0)}}
\eeq
we obtain
\beq
\hat{\Omega}^{(1)} \ket{\Psi_i^{(0)}} = \sum_{a > n} \ket{\Psi_a^{(0)}}
\frac{\braket{\Psi_a^{(0)}}{\hat{V}}{\Psi_i^{(0)}}}{E_i^{(0)} - E_a^{(0)}}
\eeq

We can now apply the perturbative expansion also to $\ham_{\mbox{\tiny eff}}$
\beq
\ham_{\mbox{\tiny eff}} =  \ham_{\mbox{\tiny eff}}^{(0)} + \ham_{\mbox{\tiny
eff}}^{(1)} + \ham_{\mbox{\tiny eff}}^{(2)} + \dots
\eeq
to obtain the following terms
\beqa
\ham_{\mbox{\tiny eff}}^{(0)} &=& \hat{\mathcal{P}} \ham^{(0)} \hat{\mathcal{P}} \\
\ham_{\mbox{\tiny eff}}^{(1)} &=& \hat{\mathcal{P}} \hat{V} \hat{\mathcal{P}} = 0 \\
\ham_{\mbox{\tiny eff}}^{(2)} &=& \hat{\mathcal{P}} \hat{V} \hat{\Omega}^{(1)}  \\
&\vdots&
\eeqa
hence, the $\mathbf{H}_{\mbox{\tiny eff}}$ matrix up to the second-order
can be obtained from the formulation of the $\ham_{\mbox{\tiny eff}}^{(2)}$ operator. 

When applying the above formulations to the NEVPT case, we have to consider
that the model space is generated by the CAS wavefunctions $\left\{
\Psi_m^{(0)} \right\}$, and the zero-order Hamiltonian is defined through its
spectral decomposition
\beqa
\ham_0(m) &=& \ket{\Psi_m^{(0)}} E_m^{(0)} \bra{\Psi_m^{(0)}} \\
&&+ \sum_{m^{\prime} \neq m} \ket{\Psi_{m^{\prime}}^{(0)}} E_{m^{\prime}}^{(0)}
\bra{\Psi_{m^{\prime}}^{(0)}} \\
&&+ \sum_{l,k,\mu} \ket{\Psi_{l,\mu}^{k}(m)} E_{l,\mu}^{k}(m)
\bra{\Psi_{l,\mu}^{k}(m)}
\eeqa
where $\Psi_{l,\mu}^{k}(m)$ are the perturbers generated by the
application of excitation operators to the state $m$. As a consequence, the
zero-order Hamiltonian is state dependen. A partition of the Hamiltonian in
a state-specific way is needed using the multipartitioning
approach\cite{cpl-233-597-1995}. For each state $m$ we define a different
partitioning scheme of the Hamiltonian
\beq
\ham = \ham_0(m) + \hat{V}(m) 
\eeq
It must be noted however that both spaces $\mathcal{P}$ and $\mathcal{Q}$
(and consequently the projection operators) are not state-specific, since
$\mathcal{P}$ is defined by the CAS space, and $\mathcal{Q} = 1 -
\mathcal{P}$. We obtain for $\hat{\Omega}^{(1)}$
\beq
\hat{\Omega}^{(1)} \ket{\Psi_m^{(0)}} = \sum_{k,l,\mu}
\frac{\ket{\Psi_{l,\mu}^{k}(m)}
\braket{\Psi_{l,\mu}^{k}(m)}{\ham}{\Psi_{m}^{(0)}}}{E_m^{(0)} -
E_{l,\mu}^{k}(m)}
\eeq
and the final $\ham_{\mbox{\tiny eff}}$ matrix element is therefore
\beq
\braket{\Psi_n^{(0)}}{\ham_{\mbox{\tiny eff}}}{\Psi_m^{(0)}} = \delta_{nm}
E_m^{(0)} + \sum_{l,k,\mu}
\frac{\braket{\Psi_n^{(0)}}{\ham}{\Psi_{l,\mu}^{k}(m)}
\braket{\Psi_{l,\mu}^{k}(m)}{\ham}{\Psi_m^{(0)}}}{E_m^{(0)} -
E_{l,\mu}^{k}(m)}
\eeq

It can be noted that the diagonal elements of the obtained matrix are the
single state contributions. In a Single-State approach, the non-diagonal
elements of the effective matrix are assumed as zero. The Quasi-Degenerate
approach provides these non-diagonal elements, whose role is to mix the
perturbed wavefunctions.

The $\mathbf{H}_{\mbox{\tiny eff}}$ matrix in general is not hermitian, but a
hermitian variant can be obtained\cite{np-20-321-1960}:
\beq
\mathbf{H}_{\mbox{\tiny eff}}^{\prime} = \mathbf{S}^{-\frac{1}{2}}
\mathbf{H}_{\mbox{\tiny eff}} \mathbf{S}^{\frac{1}{2}}
\eeq
with $S_{ij} = \integral{\tilde{\Psi}_i}{\tilde{\Psi}_j}$.

