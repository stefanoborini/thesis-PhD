\begin{center}
\begin{threeparttable}
\footnotesize
\begin{tabular*}{0.80\textwidth}{lccccc}
\hline
Method 						& $R_{\mbox{\tiny CO}}$	& $R_{\mbox{\tiny
CH}}$\tnote{a} & $R_{\mbox{\tiny CC}}$ & $\angle \mbox{XCY}$\tnote{b} 	& $\angle \mbox{OCC}$ \\	
\hline
\multicolumn{6}{c}{\small Formaldehyde} \\
CASSCF[CAS(A)]\tnote{c} 	& 1.220		& 1.090				& 			& 117.3					&       	 \\
CASSCF[CAS(C)]\tnote{c}		& 1.210		& 1.118				& 			& 115.1					&       	 \\
Exp.\cite{jpsj-18-1174-1963}& 1.208		& 1.116				& 			& 116.5					&       	 \\
\multicolumn{6}{c}{\small Acetaldehyde} \\                                                     
CASSCF[CAS(B)]\tnote{c}		& 1.222		& 1.092				& 1.501 	& 116.3					& 124.2   	 \\
Exp.\cite{jpc-8-619-1979}	& 1.213		& 1.106				& 1.504		& 114.9					& 124.0    	 \\
\multicolumn{6}{c}{\small Acetone} \\                                                          
CASSCF[CAS(B)]\tnote{c}		& 1.224		& 					& 1.509 	& 117.2					& 121.4   	 \\
Exp.\cite{jms-18-344-1965}	& 1.222		&      				& 1.507		& 117.1					& 121.4    	 \\
\hline
\end{tabular*}
\caption{\footnotesize Formaldehyde, acetaldehyde and acetone ground state
equilibrium geometries.  Distances in \AA, angles in
degrees.\label{tbl:gs_geometries}}
\begin{tablenotes}
\footnotesize
\item[a] In acetaldehyde, H is the aldehydic hydrogen.
\item[b] Formaldehyde X=H, Y=H; acetaldehyde X=H, Y=C; acetone X=C, Y=C.
\item[c] This work.
\end{tablenotes}
\end{threeparttable}
\end{center}

% jms-120-118-1986 ?
