\begin{center}
\begin{threeparttable}
\footnotesize
\begin{tabular*}{0.80\textwidth}{l@{\hspace{27mm}}cc}
\hline
Method                  &   \snpi & \tnpi \\
\hline
CASSCF  (CAS(A))\tnote{a} &    4.86 & 4.54  \\
NEVPT2 SC (CAS(A))\tnote{a} &    4.49 & 4.14  \\
NEVPT2 PC (CAS(A))\tnote{a} &    4.47 & 4.11  \\
NEVPT2 SC\cite{tca-111-352-2004} &    4.42 &       \\
NEVPT2 PC\cite{tca-111-352-2004} &    4.22 &       \\
CASPT2 \cite{jcp-104-1791-1996} &    4.18 & 3.90  \\
EOM-CCSD\tnote{b} &    4.47 &       \\
TDDFT\tnote{c} &    4.46 &       \\
MRCI \cite{jcp-111-205-1999} &    4.44 &       \\
TDDFT\tnote{d} &    4.40 &       \\
CAS/MP2\tnote{e} &    4.04 &       \\
MR-MP\tnote{f} &    4.27 & 3.98  \\
TDDFT \cite{mp-97-859-1999} &    4.29 &       \\
TDDFT(B3LYP) \cite{cpl-297-60-1998} &    4.44 & 3.86  \\
EOM-CCSD \cite{cpl-241-26-1995} &    4.48 &       \\
TDDFT/BLYP \cite{jcp-108-4060-1998} &    3.93 &       \\
Experimental\tnote{g} &    4.5  &  4.3  \\
Experimental            &  4.38\tnote{h},4.43\tnote{i} & 4.16\tnote{j},4.15\tnote{k} \\
\hline
\end{tabular*}
\caption{\footnotesize Vertical transition energies (eV) for the \npi\ singlet and triplet
excited states of acetone.  CAS(A) is defined by six active electrons and
five active orbitals.\label{tbl:vertexc_aceto_npi}}
\begin{tablenotes}
\footnotesize
\item[a] This work.
\item[b] Computed using the 6-311(2+,2+)G** basis set,
Ref. \citen{jpca-106-4192-2002}. The reported values
correspond to the 1$^1A_2$, 4$^1A_1$ and 2$^1B_1$ states.
\item[c] cc-pVDZ+aug(CO) basis set with the PBE0 functional, Ref. \citen{mp-101-1945-2003}.
\item[d] 6-31+G(d) basis set with the B3LYP functional, Ref. \citen{cpc-3-57-2002}.
\item[e] State-average CASSCF plus MR MP2 calculation using the 6-311+G(d,p) basis set, 
Ref. \citen{cpc-3-57-2002}.
\item[f] CASSCF with 8 electrons in 7 orbitals using a C[$3s2p1d$]/H[$2s1p$] basis set,
Ref. \citen{jms_theo-461-145-1999}.
\item[g] Electron-energy-loss spectroscopy in condensed phase at 18 K, Ref.
\citen{jcp-112-6707-2000}.
\item[h] Electron-impact data from Ref. \citen{jcp-87-3796-1987}.
\item[i] Ref. \citen{jpc-96-10756-1992}.
\item[j] Electron-impact data from Refs. \citen{jcp-87-3796-1987} and \citen{jcp-61-763-1974}.
\item[k] Trapped electron spectrum, Ref. \citen{cpl-36-589-1975}.
\end{tablenotes}
\end{threeparttable}
\end{center}
