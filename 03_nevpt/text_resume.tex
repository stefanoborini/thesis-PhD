\pagestyle{empty}
\begin{center}
{\Huge \textbf Resum\'e chapitre \ref{chp:nevpt} \\ NEVPT }
\end{center}

Les th\'eories perturbatives multir\'ef\'erence sont des instruments importants
pour l'\'evaluation de l'\'energie de corr\'elation sur mol\'ecules petites et
moyennes. L'avantage de l'approche perturbative est la r\'eduction du co\^ut
computationnel: \`a parit\'e de qualit\'e des r\'esultats, une approche
variationnelle est plus lourde. Seulement en temps r\'ecents les approches
perturbatives multireference sont utilis\'ees intensivement dans la recherche
en chimie quantique.

L'objectif des th\'eories perturbatives multir\'ef\'erence est la cr\'eation
d'une approche vite et propre, correspondante \`a l'approche \`a single
d\'eterminant M{\o}ller-Plesset (MP). Quand on travaille sur des syst\`emes
bien d\'ecrits avec un seul d\'eterminant, la th\'eorie M{\o}ller-Plesset au
deuxi\`eme ordre donne plus de 90 \% de l'\'energie de corr\'elation.

Quand le syst\`eme \'etudi\'e n\'ecessite d'une approche multir\'ef\'erence, et donc la
fonction d'onde d'ordre z\'ero est multir\'ef\'erence, le d\'eveloppement d'une
th\'eorie perturbative est plus complexe. Le probl\`eme est la d\'efinition
d'un Hamiltonien d'ordre z\'ero appropri\'e. Si ce choix n'est pas bon, on
obtient une th\'eorie perturbative que
\begin{itemize}
\item \textbf{n'est pas size consistent}: l'\'energie de deux syst\`emes
qui n'interagissent pas entre eux, AB, doit \^etre \'egale al'\'energie
qui on obtient en faisant le calcul sur chacune et en faisant l'addition
(E$_{AB} = $E$_A + $E$_B$)
\item \textbf{\`a des \'etats intrus}, quand les fonctions perturbatives
ont des \'energies presque \'egales \`a l'\'energie de la fonction d'ordre z\'ero.
Ce probl\`eme donne lieu \`a un comportement divergent.
\end{itemize}

La th\'eorie n-Electron Valence state Perturbation Theory est une
th\'eorie perturbative multir\'ef\'erence qui travaille sur une fonction d'onde de
type Complete Active Space (CAS). Elle a \'et\'e d\'evelopp\'ee \`a Ferrara en collaboration avec l'universit\'e
Paul Sabatier de Toulouse.  La m\'ethode adresse \`a la fois le probl\`eme
de size consistency et le probl\`eme des \'etats intrus, et donne l'approche
M{\o}ller-Plesset si on effectue le calcul sur une fonction d'onde \`a un
seul d\'eterminant.

Le d\'eveloppement theoretique utilise un Hamiltonien bielectronique,
l'Hamiltonien de Dyall, qui donne une th\'eorie propre et efficiente. Les
fonctions perturbatives sont de nature multir\'ef\'erencielle, et peuvent
\^etre classifi\'ee dans huit classes d'excitation, selon le sch\'ema de
promotion des electrons. 

La th\'eorie a deux niveaux de pr\'ecision, ``Strongly Contracted'' et ``Partially
Contracted'', selon la contraction de l'espace perturbatif.
La Strongly Contracted NEVPT utilise un espace monodimensionnel pour chaque
classe, et la Partially Contracted utilise un espace \`a dimensionalit\'e
plus haute, mais plus petite que l'espace complet des determinants
excit\'es.  L'approche Strongly Contracted donne des r\'esultats de bonne
qualit\'e si compar\'ee avec la Partially Contracted. Une diff\'erence entre
eux est normalement un sympt\^ome d'une mauvaise d\'escription \`a
l'ordre-zero.

Dans cette th\`ese, diff\`erent types de NEVPT ont \'et\'e d\'evelopp\'e
et utilis\'e:
\begin{itemize}
\item Single State NEVPT, qui fait la correction perturbative \`a un seul
\'etat \'electronique
\item Quasi Degenerate NEVPT, qui fait cette correction \`a un
ensemble d'\'etats \'electroniques simultan\'ement.  Cette approche
consid\`ere les interactions entre les fonctions d'onde apr\`es la
perturbation, et permit le m\'elange de ce fonctions.
\item Non Canonical NEVPT, est une extension de l'approche single \'etat qui
travaille avec des orbitales qui ne sont pas canoniques (elles ne diagonalisent
pas la matrice de Fock g\'en\'eralis\'ee). Cette approche est tr\`es utile
pour l'integration entre Localisation et NEVPT, parce que les resultats sont
invariables pour le m\'elange des orbitales de core et virtuelles, et donc
il permit l'utilisation directe avec un ensemble d'orbitales localis\'ees.
\end{itemize}

L'approche Single-State a \'et\'e utilis\'ee pour effectuer des calculs sur les
mol\'ecules de formaldehyde, acetaldehyde et ac\'etone. Tous ces mol\'ecules ont
un groupe carbonyle, et diff\'erents strat\'egies ont \'et\'e appliqu\'ees pour avoir
une compr\'ehension directe des \'etats excit\'es de ce groupe fondamental. En
particulier, on a effectu\'e l'\'evaluation de l'\'energi\'e des \'etats
excit\'es de valence, soit adiabatiques que verticales, et des \'etats
Rydberg. Pour les \'etats adiabatiques, on a aussi \'evalu\'e l'\'energie
de point z\'ero (ZPE) vibrationnel, les g\'eom\'etries et le comportement
vibrationnel des \'etats excit\'es de valence.

Avec l'approche Quasi-Degenerate on a \'etudi\'e la courbe de potentiel du
ground state, deux \'etats rydberg et du $\pi \rightarrow \pi^{*}$ contre le
stretching C-O, dans la mol\'ecule de formaldehyde. Le courbes obtenues \`a niveau
CASSCF ont des croisements \'evit\'es, et l'application de la NEVPT
Quasi-Degenerate donne des courbes plut\^ot diff\`erentes sur la position e la
forme des ces croisements.  Les r\'esultats de la th\'eorie QD-NEVPT peuvent
\^etre amelior\'ees avec une procedure iterative, qui peut \^etre une bonne
solution \`a des probl\'emes dans les r\'esultats obtenus.

Enfin, une \'evaluation pr\'eliminaire a \'et\'e effectu\'e avec la NEVPT
Non-Canonique sur un syst\`eme localis\'e. Les valeurs obtenues d\'emontrent la validit\'e
de cette approche, qui sera fondamentale pour l'int\'egration directe de la
th\'eorie de localisation des orbitals mol\'eculaires et la th\'eorie NEVPT.

