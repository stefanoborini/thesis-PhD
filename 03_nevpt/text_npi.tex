%\documentclass[global,twocolumn,footinfo]{svjour}
\documentclass[global,referee]{svjour}
\usepackage[dvips]{graphics}
\usepackage{epsfig}
%\usepackage{times}
%
% Macro definitions
%
\newcommand{\cit}[1]{\csname b@#1\endcsname}
\newcommand{\bbra}[1]{\left<#1\right|}
\newcommand{\kket}[1]{\left|#1\right>}
\newcommand{\bracket}[3]{\left<#1\left|#2\right|#3\right>}
\newcommand{\scal}[2]{\left<#1\left|\right.#2\right>}
\newcommand{\elm}[3]{\left< #1 \left| #2 \right| #3 \right>}
\newcommand{\hami}{\hat{\cal H}}
\newcommand{\ungap}{\!}
\newcommand{\beq}{\begin{equation}}
\newcommand{\feq}{\end{equation}}
\newcommand{\nto}[1]{$\!#1\!$}
\newcommand{\nyto}[1]{$\!n_y\!\!\rightarrow\!\! #1$}
\newcommand{\pipis}{$\!\pi\!\!\rightarrow\!\! \pi^*$}
\newcommand{\sipis}{$\!\sigma\!\!\rightarrow\!\! \pi^*$}
\newcommand{\psim}{\Psi_m}
\newcommand{\psimz}{\Psi_m^{(0)}}
\newcommand{\pertu}{\Psi_{l,\mu}^{(k)}}
\newcommand{\pertus}{S_l^{(k)}}
\newcommand{\pertusp}[2]{S_{#1}^{({#2})}}
\newcommand{\pertup}[2]{\Psi_{#1}^{({#2})}}
\newcommand{\pertusb}{\bar{S}_l^{(k)}}
\newcommand{\psilmk}{\Psi_{l,\mu}^{(k)}}
\newcommand{\pertue}[2]{E_{#1}^{({#2})}}
\newcommand{\elmk}{E_{l,\mu}^{(k)}}
\newcommand{\E}[1]{E_{#1}}
\newcommand{\crea}[1]{a_{#1}^{+}}
\newcommand{\dist}[1]{a_{#1}^{\phantom{+}}}
\newcommand{\heff}[1]{h_{#1}^{\mathrm{eff}}}
\newcommand{\half}{\frac{1}{2}}
\newcommand{\emp}[1]{E_m^{(#1)}}
%
% End of macro definitions
%
\setlength{\voffset}{.1in}
\journalname{Theor Chem Acc}
\renewcommand{\abstractname}{Abstract.}

\setlength{\textwidth}{16.5cm}
\setlength{\textheight}{24cm}
\begin{document}
 
\title{An application of second-order n-electron valence state perturbation
theory (NEVPT2) to the calculation of excited states.}

 
\author{Celestino Angeli, Stefano Borini, Renzo Cimiraglia}

\institute{
Dipartimento di Chimica, Universit\`a di Ferrara, \\
Via Borsari 46, 
I-44100 Ferrara, Italy} 

\titlerunning{An application of second-order ...}
\authorrunning{Angeli et al.}

\date{Received:  .... / Accepted: .....}
\mail{R. Cimiraglia}
\dedication{Dedicated to Jacopo Tomasi, mentor and friend}
\maketitle


\begin{abstract}
$n$--electron valence state perturbation theory (NEVPT) is a form of
multireference perturbation theory (MRPT) where all the zero--order
wave functions are of multireference nature, being generated as
eigenfunctions of a two--electron model Hamiltonian. The absence of
intruder states makes NEVPT an interesting choice for the calculation
of electronically excited states. Test calculations have been
performed on several valence and Rydberg transitions for the
formaldehyde and acetone molecules; the results are in good accordance
with the best calculations and with the existing experimental data.
\end{abstract}
\keywords{Multireference CI -- Perturbation CI -- NEV-PT -- Excited states}
{\smallskip}
{\smallskip}
%\strich

\section{Introduction\label{sect1}}
Most molecules in their ground state near the equilibrium geometry
have a closed shell nature and can be successfully described by a
single Slater determinant at the Hartree--Fock level of theory. A good
portion of the correlation energy can then be effectively gained at a
low computational cost by applying M\o{}ller--Plesset perturbation
theory \cite{Moller34}, usually at the second order (MP2). In a great deal
of chemically interesting circumstances, though, the description based
on a single reference determinant becomes defective, as, for instance,
in molecular geometries far from equilibrium during the course of a
chemical reaction or in excited electronic states. In such cases it is
essential to provide a zero--order description where all the
determinants (configurations) that play an important r\^ole in the
electronic wave function be explicitly taken into
account. Multireference perturbation theory (MRPT), where the zero
order wave-function is variationally built upon a configuration
interaction (CI) involving the most important determinants is a
natural extension of the single--determinant M\o{}ller--Plesset PT. We
shall not review here the more than thirty year old history of MRPT,
confining ourselves to recalling that the most successful treatments
are based on a zero--order reference of CAS--SCF type (Complete Active
Space Self--Consistent Field) \cite{Ander90,Ander92}, where the
first--order perturbation correction to the wave function is built in
terms of contracted excitations applied to the zero--order
wave function \cite{Werner82}. The CASPT2 technique of Roos and collaborators stands
out as a particularly efficient method for the treatment of
correlation energy in a multireference--based description of molecular
systems. Recently a variant of CAS--based MRPT, called n--electron
valence state perturbation theory (NEVPT), has been
proposed \cite{nev1,nev2,nev3} where particular care has been addressed
to complying with some important formal requirements among which we
quote: a)~strict separability (size consistence), ensuring that the
energy of a system made of non--interacting parts be equal to the sum
of the energies of the isolated parts calculated with the same method,
b)~absence of ``intruder states'', requiring that the energies of the
zero--order wave functions belonging to the outer space be well
separated from the energy of the reference state, thus averting
divergences in the perturbation summation. The latter requirement 
assumes particular relevance in the treatment of electronically
excited states, where the appearance of intruders can most easily
manifest itself. 

This paper is addressed to the description of electronically excited
states with the NEVPT at the second order (NEVPT2)
technique in a couple of test molecules, formaldehyde and acetone.
The principal aim of this work is not to solve the still open questions of the interpretation of the
electronic spectra of the molecules studied but, rather, to verify 
the ability of NEVPT2 to provide a viable means of calculation of excited states.
The rest of
the paper is organized as follows: in Sect. \ref{sect2} a brief summary of
the NEVPT2 method will be provided; Sect. \ref{sect3}  will be
dedicated to the description of various electronic transitions in the
formaldehyde and acetone molecules; Sect. \ref{sect4} will
contain concluding remarks.


\section{Summary of the NEVPT2 method\label{sect2}}
NEVPT2 has been formulated in different variants, according to the
type of zero--order Hamiltonian adopted and to the degree of
contraction employed in the definition of the zero--order
wave functions, referred to as the ``perturbers'' in the following. In
all the NEVPT variants the perturbers exhibit a multireference nature
and are obtained as eigenfunctions of a two--electron model
Hamiltonian. The state of interest $\psim$ is approximated at the zero
order by a CAS--CI wave function $\psimz$ (usually CAS--SCF) obtained
by diagonalizing the Hamiltonian matrix built on a given CAS
space. The spatial orbitals utilized are subdivided into the three
usual CAS classes: core, with occupation numbers always equal to two,
active, with all possible occupations (0,1,2) and virtual, with
occupation always zero in all the determinants of the CAS. The
zero--order wave functions other than the variational $\psimz$ (the
perturbers) also belong to CAS--CI spaces characterized by well
defined occupation patterns of the inactive (core + virtual)
orbitals. A generic perturber is designated as $\pertu$ and the
CAS--CI space to which it belongs as $\pertus$ where $k$ is the number
of electrons promoted to (if positive) or removed (if negative) from
the active space ($-2\le k\le 2$), $l$ denotes the fixed occupation
pattern of the inactive orbitals and $\mu$ enumerates the various
perturbers in $\pertus$. The number of types of CAS--CI spaces which
play a r\^ole in second--order perturbation theory is restricted to
eight: there are two possible types with $k=0$ according to whether
two or one core electrons are transferred to the virtual orbital
space; for $k=+1$ ($-1$) two types are also possible, the first
implying a transfer from two core electrons to one virtual orbital and
to the active space (from one core electron and one active space
electron to two virtual orbitals), the second involving the passage of
only one electron to (from) the active space; lastly, the case $k=+2$
($-2$) generates only one type of space with two electrons passing
from the core to the active space (from the active to the virtual
space). The different variants of NEVPT2 are defined according to the
number of perturbers that are chosen from the $\pertus$ spaces. In the
``strongly contracted'' approach only one function is chosen, by
projecting the action of the electronic Hamiltonian to the reference
function onto the $\pertus$ space: $  \pertup{l}{k} =
P_{\pertus}H\psimz$, whereas in the ``partially contracted'' approach
the $ \psilmk$ perturbers belong to the subspace $\pertusb$ of
$\pertus$ generated by the double excitation operators which map
$\psimz$ onto $\pertus$. The energies of the perturbers are evaluated
through the use of a model Hamiltonian, $H^D$, by the following
prescription: 
\begin{itemize}
\item[a)]for the strongly contracted approach 
  \begin{displaymath}
  \pertue{l}{k} =
  \frac{\bracket{\pertup{l}{k}}{H^D}{\pertup{l}{k}}}
  {\scal{\pertup{l}{k}}{\pertup{l}{k}}};
  \end{displaymath}
\item[b)]for the partially
  contracted variant the perturbers and their energies are provided by
  the diagonalization of the $H^D$ operator in the $\pertusb$
  subspaces:
  \begin{displaymath}
   P_{\pertusb}H^DP_{\pertusb} \psilmk = \elmk\psilmk.
  \end{displaymath}
\end{itemize}

From the computational point of view, a particularly convenient choice
of the model Hamiltonian is provided by Dyall's operator \cite{Dyall95},
\begin{equation}
  \label{eq:dyall}
  H^D = H_i + H_v
\end{equation}
\begin{equation}
  \label{eq:dyall2}
H_i = \sum_{i}^{core}\epsilon_{i}\E{ii} +
\sum_{r}^{virt}\epsilon_{r}\E{rr} +C
\end{equation}
\begin{equation}
  \label{eq:dyall3}
  H_v = \sum_{ab}^{act}\heff{ab}\E{ab} +\half\sum_{abcd}^{act}
  \scal{ab}{cd} (\E{ac}\E{bd}-\delta_{bc}\E{ad})
\end{equation}
where  $\epsilon_i$ and $\epsilon_r$ are suitable orbital energies \cite{Angeli2000},
$\E{xy} = \crea{x\alpha}\dist{y\alpha} +
\crea{x\beta}\dist{y\beta}$ is the spin--traced excitation
operator \cite{Mcweeny},  $ \heff{ab}$ is a modified one--electron
matrix taking into account the interaction with the core electrons
$\heff{ab} = h_{ab} + \sum_{j}^{core}( 2\scal{aj}{bj} -
  \scal{aj}{jb})$ and $C$ is a constant ensuring the equivalence of
  $H^D$ with $H$ within the CAS space.

The form of the operator $H^D$ in Eqs.~(\ref{eq:dyall}--\ref{eq:dyall3}) only ensures
that the resulting first order perturbation correction to the
wave function is invariant under unitary transformations of orbitals
belonging to the active space. It is an easy task to extend the $H^D$
definition so that invariance be guaranteed under unitary
transformations within each of the three orbital classes (core,
active, virtual) by modifying the inactive one--electron component
$H_i$ into the following:
\begin{equation}
  \label{eq:dyall4}
H_i = \sum_{i,j}^{core}f_{ij}\E{ij} +
\sum_{r,s}^{virt}f_{rs}\E{rs} +C
\end{equation}
with 
\begin{eqnarray}
  \label{eq:dyall5}
  f_{ij} &=& - \bracket{\dist{i}\psimz}{H}{\dist{j}\psimz}
  +\delta_{ij}\emp{0}\\
  \label{eq:dyall6}
  f_{rs} &=& \bracket{\crea{r}\psimz}{H}{\crea{s}\psimz} -\delta_{rs}\emp{0}
\end{eqnarray}
Diagonalization of the two $\mathbf{f}$ matrices leads then to the
adoption of so called canonical core and virtual orbitals, a practice that we
shall tacitly assume in the calculations of the next sections.

As has been shown in Ref. \cite{nev3}, all the relevant quantities
which are necessary for the second order correction to the energy,
i.e. the interaction of the reference $\psimz$ function with the
perturbers and the energy denominators, can be easily calculated with
the help of auxiliary quantities which need the knowledge of the
zero--order density matrices of particle rank not higher than four,
with indices only spanning the active orbital space. In the case of
the partially contracted approach, the use of Dyall's Hamiltonian is
particularly beneficial because each of the eight typologies of
$\pertusb$ subspaces only necessitates one diagonalization which is
then valid for all the various $\pertusb$ instances. Thus the
partially contracted approach, albeit involving many more perturber functions
than the strongly contracted case, only needs a small computational
overhead when compared with the strongly contracted case. The two
forms of NEVPT2, though, show very similar results in the
second--order correction to the energy, as has been shown in
Ref. \cite{nev3}, demonstrating that the strongly contracted approach
is built on an effective averaging process.

A final remark concerns the issue of the occurrence of possible small
denominators (intruder states) in the perturbation summation: the most
critical case can present itself in the $\pertusp{r}{-1}$ spaces,
which involve the passage of one electron from the active space to a
virtual orbital ($r$), accompanied by an excitation within the active
space. In such case the energy denominators can be expressed as a
difference between the energy of a virtual orbital ($\epsilon_r$) and
the one associated to a ionization potential in the valence shell. If
the virtual orbital is very diffuse, as happens when introducing
Rydberg orbitals in the basis set, the energy $\epsilon_r$ is close to
zero and the energy denominator reduces to the valence ionization
which is anyway never zero. So NEVPT2 should be applicable to the
study of valence excited states the energies of which are well
separated from the first ionization potential.



\section{Applications\label{sect3}}

\subsection{Formaldehyde}
\label{formsec}

The vertical spectrum of the formaldehyde molecule was computed at the ground state 
experimental geometry \cite{Nelson65} ($R$(CO)=1.208 \AA, $R$(CH)=1.116 \AA\ 
and $\theta$(HCH)=116.5 degree). The molecule belongs to the C$_{\rm 2v}$ point group symmetry and
lies in the $yz$ plane with the C and O atoms on the $z$ axis.  The atomic natural orbitals (ANO) basis set
of Widmark and coworkers \cite{Widma90} has been used with two different contraction schemes:
the smaller (indicated with ANO(S)) is C,O [$4s3p1d$]/H[$2s1p$] and the larger (ANO(L)) is 
C,O [$6s5p3d2f$]/H[$4s3p2d$]. 
These valence basis sets have been augmented with diffuse functions in order to properly describe
the diffuse orbitals involved in the Rydberg states. These basis functions are obtained by contraction
of a set of $8s8p8d$ gaussian primitives whose exponents are generated using the procedure proposed by 
Kaufmann {\it et al} \cite{Kaufm89}. The contraction coefficients are computed following the procedure
developed by  Roos {\it et al} \cite{Roos95} and two contraction schemes are considered: 
[$1s1p1d$], (Ryd(S)) and [$3s3p3d$], (Ryd(L)).
In order to directly compare our
results with the CASPT2 and the Size-Consistent Self-Consistent CI 
((SC)$^2$-CI) calculations of Merch\'an {\it et al} \cite{Merch95} and 
of Pit\-arch-Ruiz {\it et al} \cite{Ruiz03}, respectively,  we have used in the calculations the 
combinations ANO(S)-Ryd(S) and ANO(L)-Ryd(L).

The molecular orbitals are obtained from average CAS-SCF calculations which involve the lowest states
of a given symmetry. The active spaces, together with the number and the nature of the states 
considered in the average procedure are reported in Table \ref{Tabforact} and are taken from 
Ref. \cite{Merch95}.
The number of orbitals for each irreducible representation was chosen by the authors
so that all the states of interest could be correctly described.
In some cases the active space was enlarged in order to minimize the effect
of the intruder states problem in the CASPT2 calculations. 
Given that our perturbation treatment is not affected by the intruder states problem, 
in our calculations a reduction of the active space should be possible but the 
comparison of our results 
with those of Refs. \cite{Merch95} and \cite{Ruiz03} would be in this case less clear.




The energies of the states are computed in a state specific multireference perturbation scheme.
The zero order description of each state is obtained from a CAS-CI calculation using
the average CAS--SCF active orbitals. The inactive orbitals have been transformed 
in order to diagonalize the state specific Fock matrices defined in Eqs. (\ref{eq:dyall5})
and (\ref{eq:dyall6}).
A second order correction to the energy is computed using the strongly 
contracted (SC) and partially contracted (PC) variants of the NEV-PT method. All orbitals and electrons 
are included in the perturbation treatment. The excitation energies are computed with respect 
to the same ground state energy which is evaluated as the second order correction to the
energy with the reference energy and wave function obtained from a state specific 
CAS--SCF calculation with four electrons in two $b_1$ ($\pi$ + $\pi^*$) and two $b_2$
($n_y$ + virtual) orbitals.
This approach for the calculation of the excitation energies differs from the
one used in the CASPT2  \cite{Merch95} calculation, where a 
different ground state energy was used for each irreducible representation.
The vertical excitation energies obtained in our calculations are 
reported in Table \ref{Tabforeery}
for the Rydberg states and in Table \ref{Tabforeeval} for the valence states, together with 
the results of previous theoretical calculations and with some experimental results.





In Tables \ref{Tabforeemaes1} and  \ref{Tabforeemaeb1} we show the comparison between our results and
those of Pitarch-Ruiz {\it et al} \cite{Ruiz03}, which can be considered a good reference since they 
involve the whole single plus double excitations space on top of a CAS at a variational level with a size 
consistence correction. We remark that the mean absolute error (MAE) of our results is always 
small with the worst case being represented by the SC-NEVPT in the ANO(L) + Ryd(L) basis (0.15 eV).
The small errors appearing in Tables \ref{Tabforeemaes1} and  \ref{Tabforeemaeb1} bear out the
reliability of NEVPT2 which can yield results of good accuracy, comparable with much more refined calculations, but at a much
lesser computational cost.

For the case of the smaller basis (ANO(S) + Ryd(S)) the CASPT2 results are also available 
\cite{Merch95} and have been reported in Table \ref{Tabforeemaes1} for comparison.
It can be remarked that NEVPT2 and CASPT2 appear to be of the same quality, with the former showing in
all cases a small squared norm of the wave function perturbation correction (see Tables 
\ref{Tabforeery} and \ref{Tabforeeval}) thus getting over the intruder state problem.

As to the comparison with the experimental data, beyond a satisfactory general agreement with our
theoretical results we can make the following observations:
\begin{itemize}
\item[$\bullet$] in accordance with most theoretical calculations our vertical 2 $^1$A$_1$  and 1 $^1$B$_2$
Rydberg transitions appear in inverted order with respect to experimental adiabatic transitions
\cite{Robin85}: a more stringent comparison would require the calculation of adiabatic transition
with due allowance for the zero point energy correction;
\item[$\bullet$] the calculation of the $^1$A$_1$ Rydberg states with the larger basis set introduces one more
Rydberg state ($n_y\rightarrow\pi^*$) below the valence $\pi\rightarrow\pi^*$; such state has been ignored
in the average CAS--SCF since we were 
interested in transitions involving Rydberg orbitals not exceeding
the quantum number $n=3$;
\item[$\bullet$] mixing between Rydberg and valence character may occur in both the $^1$A$_1$ and $^1$B$_1$
transitions \cite{Merch95,Hachey95}. For a correct treatment of such states a quasi--degenerate treatment
would be required \cite{Durand87,Finley98}. We count to introduce such corrections in NEVPT2 in the near future.
\end{itemize}




\subsection{Acetone}
\label{acetsec}

The computational strategy used for acetone closely follows the one applied to formaldehyde.
The vertical spectrum has been computed at the ground state 
experimental geometry \cite{Nelson65}. The molecule belongs to the C$_{\rm 2v}$ point group symmetry 
with the OCCC skeleton in the $yz$ plane (C and O atoms on the $z$ axis and with an orientation of the CH$_3$ 
groups that place the two H atoms lying in the $yz$ plane as far as possible). 
For acetone, we only consider the C,O[$4s3p1d$]/H[$2s1p$] contraction of the ANO basis set
of Widmark and coworkers \cite{Widma90}.
%\clearpage



The Rydberg states are described using a set of $8s8p8d$ diffuse functions \cite{Kaufm89}
contracted to [$1s1p1d$] following the procedure described in Ref. \cite{Roos95}.


As in formaldehyde, average CAS-SCF calculations provide the molecular orbitals: 
the active spaces and the number and the nature of the states 
considered in the average procedure are reported in Table \ref{Tabaceact}
and are taken from Ref. \cite{Merch96}.
With respect to formaldehyde, the active space has been 
modified adding the two CO $\sigma$ and $\sigma^*$ orbitals and the two CO $\sigma$ electrons, except
for the A$_1$ symmetry where three virtual orbitals (one of $b_1$ and two of $b_2$ symmetry)
have been removed. In Ref. \cite{Merch96} the  CO $\sigma$ and $\sigma^*$ orbitals have been
added to the active space in order to correctly describe the adiabatic electronic transitions 
for the valence states, in which an elongation of the CO bond is observed 
due to the promotion of an electron to the $\pi^*$ orbital.

Given that we present here only results on the vertical transitions, also in the case
of acetone a reduction of the active space used in Ref. \cite{Merch96} would have been 
possible, but we have chosen to maintain the same active space in order to have a
meaningful comparison with the CASPT2 data.

The energies of the states are computed following the strategy outlined for formaldehyde
and the transition energies are reported in Table \ref{Tabaceeery} 
for the Rydberg states and in Table
\ref{Tabaceeeval} for the valence states, 
together with the results of other theoretical calculations and with some experimental results.


We note that our PC-NEVPT2 results compare very well with the CASPT2 ones.
We also remark that in two of the valence transitions (4 $^1$A$_1$, $\pi\rightarrow\pi^*$ and
1 $^1$A$_2$, $n_y\rightarrow\pi^*$) the difference between strongly and partially contracted 
NEVPT appears to be unusually sizable (0.59 eV and 0.20 eV, respectively); 
we think that this is symptom for the zero order wave function to necessitate significant improvement.



\section{Concluding remarks\label{sect4}}
Among the formal requirements satisfied by NEVPT (see section
\ref{sect2}), the absence of intruder states appears particularly
interesting for the application to the calculation of electronically
excited states. The results shown in the preceding section for the
vertical transitions of formaldehyde and acetone are of good quality
and exhibit good agreement with the best calculations so far performed
as well as with the existing experimental data. Our
calculations have been carried out starting from rather modest size
CAS--SCF wave functions. The computational overhead involved by the
two forms of NEVPT (strongly and partially contracted) amounts to only
a small fraction of the CAS--SCF calculation for such small active
orbital spaces and this favorable situation is not expected to
drastically change when passing on to larger molecules, provided that
the active space can be kept within manageable dimensions ($\le$ 10,
say).

No evidence of divergences or misbehaviours in the perturbation
summation has been found in the calculation of the Rydberg states,
which are particularly prone to exhibiting the appearance of intruder
states. We are confident that NEVPT2 can be successfully adopted as a
standard tool for the exploration of electronically excited states of
medium--size molecules.





%\section*{Acknowledgment}


\begin{thebibliography}{References}
%
%TCA bibliography style
%
\newcommand{\uau}[2]{#2 #1 }
\newcommand{\au}[2]{#2 #1, }
\newcommand{\lau}[2]{#2 #1 }
\newcommand{\jo}[4]{(#4) #1 #2: #3}
%

\bibitem{Moller34}
\au{C}{M\o ller}  \lau{MS}{Plesset}
\jo{Phys Rev}{46}{618}{1934}

\bibitem{Ander90}
\au{K}{Andersson} \au{P}{Malmqvist} \au{BO}{Roos} \au{AJ}{Sadlej} \lau{K}{Wolinski}
\jo{J Phys Chem}{94}{5483}{1990}

\bibitem{Ander92}
\au{K}{Andersson} \au{P}{Malmqvist} \au{BO}{Roos}
\jo{J Chem Phys}{96}{1218}{1992}

\bibitem{Werner82}
\au{H-J}{Werner} \lau{EA}{Reinsh}
\jo{J Chem Phys}{76}{3144}{1982}

\bibitem{nev1}
\au{C}{Angeli} \au{R}{Cimiraglia} \au{S}{Evangelisti} \au{T}{Leininger} \lau{JP}{Malrieu}
\jo{J Chem Phys}{114}{10252}{2001}

\bibitem{nev2}
\au{C}{Angeli} \au{R}{Cimiraglia} \lau{JP}{Malrieu}
\jo{Chem Phys Lett}{350}{297}{2001}

\bibitem{nev3}
\au{C}{Angeli} \au{R}{Cimiraglia} \lau{JP}{Malrieu}
\jo{J Chem Phys}{117}{9138}{2002}

\bibitem{Dyall95}
\uau{K G}{Dyall}
\jo{J Chem Phys}{102}{4909}{1995}

\bibitem{Angeli2000}
\au{C}{Angeli} \au{R}{Cimiraglia} \lau{JP}{Malrieu}
\jo{Chem Phys Lett}{317}{472}{2000}

\bibitem{Mcweeny}
\uau{R}{McWeeny} ``Methods of Molecular Quantum Mechanics'', Academic
Press, London, 1989.

\bibitem{Nelson65}
\au{R}{Nelson} \lau{L}{Pierce}
\jo{J Mol Spectr}{18}{344}{1965}

\bibitem{Widma90}
\au{P-O}{Widmark} \au{P-A}{Malmqvist} \lau{BO}{Roos}
\jo{Theor Chim Acta}{77}{291}{1990}

\bibitem{Kaufm89}
\au{K}{Kaufmann} \au{W}{Baumeister} \lau{M}{Jungen}
\jo{J Phys B: At Mol Opt Phys}{22}{2223}{1989}

\bibitem{Roos95}
\au{BO}{Roos} \au{MP}{F\"ulsher} \au{P\AA}{Malmqvist}
\au{M}{Merch\'an} \lau{L}{Serrano-Andr\'es}
in: Langhoff S R (Ed.), {\it Quantum Mechanical Eletronic Structure
Calculation with Chemical Accuracy}, p. 357-438, Kluwer Academic Publishers, the Netherlands,
1995.

\bibitem{Merch95}
\au{M}{Merch\'an} \lau{BO}{Roos}
\jo{Theor Chem Acc}{92}{227}{1995}

\bibitem{Ruiz03}
\au{J}{Pitarch-Ruiz} \au{J}{S\'anchez-Mar\'\i n} \au{A}{S\'anchez de Mer\'as} \lau{D}{Maynau}
\jo{Mol Phys}{101}{483}{2003}

\bibitem{Paris96}
\au{O}{Parisel} \lau{Y}{Ellinger}
\jo{Chem Phys}{205}{323}{1996}

\bibitem{Hachey95}
\au{MR Hachey} \au{PJ}{Bruna} \lau{F}{Grein}
\jo{J Phys Chem}{99}{8050}{1995}

\bibitem{Muller01}
\au{T}{M\"uller} \lau{H}{Lischka}
\jo{Theor Chem Acc}{106}{369}{2001}

\bibitem{gwalt95}
\au{SR}{Gwaltney} \lau{RJ}{Bartlett}
\jo{Chem Phys Lett}{241}{26}{1995}

\bibitem{Wiberg02}
\au{KB}{Wiberg} \au{AE}{de Oliveira} \lau{G}{Trucks}
\jo{J Phys Chem A}{106}{4192}{2002}

\bibitem{Robin85}
\uau{MB}{Robin}
in: {\it Higher excited states of polyatomic molecules}, Vol. 3, p. 256, Academic Press
New York, 1985.

\bibitem{Taylor82}
\au{S}{Taylor} \au{DG}{Wilden} \lau{J}{Corner}
\jo{Chem Phys}{70}{291}{1982}

\bibitem{Brint85}
\au{P}{Brint} \au{J-P}{Connerade} \au{C}{Mayhew} \lau{K}{Sommer}
\jo{J Chem Soc Faraday Trans 2}{81}{1643}{1985}

\bibitem{Suto86}
\au{M}{Suto} \au{X}{Wang} \lau{LC}{Lee}
\jo{J Chem Phys}{85}{4228}{1986}

\bibitem{Walzl87}
\au{KN}{Walzl} \au{CF}{Koerting} \lau{A}{Kuppermann}
\jo{J Chem Phys}{87}{3797}{1987}

\bibitem{Durand87}
\au{P}{Durand} \lau{JP}{Malrieu}
\jo{Adv Chem Phys}{67}{1}{1987}

\bibitem{Finley98}
\au{J}{Finley} \au{PA}{Malmqvist} \au{BO}{Roos} \lau{L}{Serrano-Andr\'es}
\jo{Chem Phys Lett}{288}{299}{1998}

\bibitem{Merch96}
\au{M}{Merch\'an} \au{B}{Roos} \au{R}{McDiarmid} \lau{X}{Xing}
\jo{J Chem Phys}{104}{1791}{1996}

\bibitem{Philis93}
\au{JG}{Philis} \lau{L}{Goodman}
\jo{J Chem Phys}{98}{3795}{1993}

\bibitem{Mcdia88}
\au{R}{McDiarmid} \lau{A}{Sablji\'c}
\jo{J Chem Phys}{89}{6086}{1988}

\end{thebibliography}
\newpage
\begin{table}[b]
\caption{Active spaces and number of states used in the average CAS-SCF calculations
for the formaldehyde molecule (in all cases 4 active electrons)}
\label{Tabforact}       
\begin{tabular}{clc}
\hline\noalign{\smallskip}
\# MOs$^a$ & Symmetry and nature of states    & \# states$^b$ \\
\noalign{\smallskip}\hline\noalign{\smallskip}
    (0340)   & $^1$A$_1$ (GS;$n_y\!\rightarrow\!3p_y$,$3d_{yz}$;$\pi\!\rightarrow\!\pi^*$) & 4 \\
    (2200)   & $^1$B$_1$ ($\sigma\!\rightarrow\!\pi^*$) & 1\\
    (0211)   & $^1$B$_1$ ($n_y\!\rightarrow\! 3d_{xy}$) & 1 \\
    (4210)   & $^1$B$_2$ ($n_y\!\rightarrow\!3s$,$3p_z$,$3d_{x^2\!-\!y^2}$,$3d_{z^2}$) & 4 \\
    (0410)   & $^1$A$_2$ ($n_y\!\rightarrow\! \pi^*$, $3p_x$, $3d_{xz}$) & 3 \\
\noalign{\smallskip}\hline
\end{tabular}\\
{\smallskip}

$^a$ number of molecular orbitals in the active space for 
the four irreducible representations ($a_1$,$b_1$,$b_2$,$a_2$)\\
$^b$ number of states used in the average procedure
% Or use
\end{table}

%\begin{minipage}{\textwidth}
\begin{table*}[h]
\caption{Vertical excitation energies (eV) for the Rydberg states of
the formaldehyde molecule.
The numbers in parenthesis are the squared norms of the first order corrections to
the wave function. The
squared norm for the ground state is 0.075 (NEV-PT SC$^b$), 0.076 (NEV-PT PC$^b$), 
0.091 (NEV-PT SC$^c$) and 0.097 (NEV-PT PC$^c$).}
\label{Tabforeery}
\begin{tabular}{lccccccccc}
\hline\noalign{\smallskip}
Method &2 A$_1$ & 3 A$_1$ & 1 B$_2$ & 2 B$_2$ & 3 B$_2$ & 4 B$_2$ & 2 A$_2$ & 3 A$_2$ &  2 B$_1$ \\
 &(\nto{3p_y}) & (\nto{3d_{yz}}) & (\nto{3s}) & (\nto{3p_z}) & 
(\nto{3d_{x^2\!-\!y^2}}) & (\nto{3d_{z2}}) & (\nto{3p_x}) &
(\nto{3d_{xz}}) & (\nto{3d_{xy}})     \\
\noalign{\smallskip}\hline\noalign{\smallskip}
%CAS--SCF$^{a,b}$  &   7.72 & 8.83  & 7.02  & 7.80  & 8.73  & 8.86  & 8.49  & 9.43  & 8.81 \\
CAS--SCF$^{a,b}$  &   8.07 & 9.18  & 7.37  & 8.15  & 9.08  & 9.21  & 8.84  & 9.78  & 9.16 \\
%CAS--SCF$^{a,c}$  &   7.66 & 8.75  & 6.91  & 7.70  & 8.62  & 8.75  & 8.43  & 9.34  & 8.75 \\
CAS--SCF$^{a,c}$  &   8.04 & 9.12  & 7.29  & 8.08  & 8.99  & 9.12  & 8.81  & 9.72  & 9.12 \\
%SC-NEVPT$^{a,b}$&   8.30 & 9.45  & 7.31  & 8.14  & 9.16  & 9.33  & 8.36  & 9.37  & 9.29 \\
SC-NEVPT$^{a,b}$&   8.27 & 9.42  & 7.28  & 8.11  & 9.13  & 9.30  & 8.33  & 9.34  & 9.26 \\
                & (0.077)&(0.076)&(0.092)&(0.090)&(0.083)&(0.082)&(0.080)&(0.079)&(0.079)\\
%SC-NEVPT$^{a,c}$&   8.46 & 9.64  & 7.40  & 8.24  & 9.24  & 9.44  & 8.53  & 9.55  & 9.46 \\
SC-NEVPT$^{a,c}$&   8.39 & 9.56  & 7.32  & 8.16  & 9.17  & 9.37  & 8.46  & 9.48  & 9.39 \\
                & (0.084)&(0.083)&(0.090)&(0.090)&(0.087)&(0.086)&(0.099)&(0.097)&(0.086)\\
%PC-NEVPT$^{a,b}$&   8.28 & 9.43  & 7.37  & 8.20  & 9.22  & 9.39  & 8.41  & 9.42  & 9.35 \\
PC-NEVPT$^{a,b}$&   8.20 & 9.34  & 7.28  & 8.12  & 9.14  & 9.31  & 8.33  & 9.34  & 9.27 \\
                & (0.084)&(0.084)&(0.095)&(0.093)&(0.085)&(0.084)&(0.082)&(0.081)&(0.081)\\
%PC-NEVPT$^{a,c}$&   8.44 & 9.61  & 7.45  & 8.29  & 9.30  & 9.51  & 8.58  & 9.61  & 9.52 \\
PC-NEVPT$^{a,c}$&   8.31 & 9.49  & 7.33  & 8.17  & 9.17  & 9.38  & 8.45  & 9.48  & 9.39 \\
                & (0.092)&(0.090)&(0.092)&(0.092)&(0.089)&(0.088)&(0.103)&(0.100)&(0.087)\\
CASPT2 \cite{Merch95}
                &   8.12 & 9.24  & 7.30  & 8.09  & 9.13  & 9.31  & 8.32  & 9.31  & 9.23 \\
MC/BMP \cite{Paris96}
                &   7.95 & 9.11  & 6.90  & 7.77  & 8.95  & 9.11  & 8.46  & 8.82  & 9.06 \\
(SC)$^2$ CAS+SD$^b$ \cite{Ruiz03}
                &   8.14 & 9.26  & 7.17  & 7.96  & 9.00  & 9.19  & 8.30  & 9.28  & 9.12 \\
(SC)$^2$ MR+SD$^c$ \cite{Ruiz03}
                &   8.27 & 9.31  & 7.12  & 7.95  & 8.96  & 9.18  & 8.36  & 9.34  & 9.36 \\
CCR(3)$^b$ \cite{Ruiz03}
               &    8.01 & 9.16  & 7.11  & 7.91  & 8.99  & 9.21  & 8.25  & 9.26  & 9.12 \\
CCR(3)$^c$ \cite{Ruiz03}
               &    8.14 & 9.27  & 7.16  & 7.99  & 9.04  & 9.27  & 8.38  & 9.40  & 9.25 \\
MRD-CI \cite{Hachey95}
                &   8.10 & 9.25  & 7.15  & 8.05  & 9.05  & 9.25  & 8.32  & 9.34  & 9.32 \\
MR-CISD + Q \cite{Muller01}
                &   8.13 & 9.28  & 7.27  & 8.10  & 9.15  & 9.30  & 8.34  & 9.36  & 9.26 \\
MR-AQCC  \cite{Muller01}
                &   8.24 & 9.38  & 7.21  & 8.03  & 9.09  & 9.24  & 8.46  & 9.49  & 9.37 \\
EOM-CCSD \cite{gwalt95}
                &   7.99 &10.16  & 6.99  & 7.93  & 9.25  & 9.98  & 8.45  &10.67  & 9.84 \\
EOM-CCSD \cite{Wiberg02}
                &   7.98 & 9.13  & 7.04  & 7.88  & 8.94  & 9.12  & 8.21  & 9.29  &10.89 \\
Exp \cite{Robin85}
                &   7.97 &       & 7.11  & 8.14  & 8.88  &       & 8.37  &       &      \\
Exp \cite{Taylor82}
                &        &       &       &       &       &       &       &       & 9.22 \\
Exp \cite{Brint85,Suto86}
                &        &       & 7.09  &       &       &       &       &       &      \\
\noalign{\smallskip}\hline
\end{tabular}\\
{\smallskip}

$^a$  This work\\
$^b$  ANO basis set with contraction [$4s3p1d$/$2s1p$] + $1s1p1d$, ANO(S)+Ryd(S)\\
$^c$  ANO basis set with contraction [$6s5p3d2f$/$4s3p2d$] + $3s3p3d$, ANO(L)+Ryd(L)
\end{table*}

\clearpage
\newpage

\begin{table}[h]
\caption{Vertical excitation energies (eV) for the valence states of
the formaldehyde molecule.
The numbers in parenthesis are the squared norms of the first order corrections 
to the wave function. The
squared norm for the ground state is 0.075 (NEV-PT SC$^b$), 0.076 (NEV-PT PC$^b$), 
0.091 (NEV-PT SC$^c$) and 0.097 (NEV-PT PC$^c$).}
\label{Tabforeeval}
\begin{tabular}{lccc}
\hline\noalign{\smallskip}
Method &4 A$_1$ & 1 A$_2$ & 1 B$_1$ \\
 & (\pipis) & ($n_y\rightarrow\pi^*$) & (\sipis) \\
\noalign{\smallskip}\hline\noalign{\smallskip}
%CAS--SCF$^{a,b}$          & 10.24  & 4.94  & 9.54  \\ Rispessto al GS mediato
CAS--SCF$^{a,b}$          & 10.59  & 5.28  & 9.89  \\
%CAS--SCF$^{a,c}$          & 10.10  & 4.89  & 9.45  \\ Rispessto al GS mediato
CAS--SCF$^{a,c}$          & 10.47  & 5.27  & 9.82  \\
%SC-NEVPT$^{a,b}$        & 10.12  & 4.07  & 9.56  \\ Rispessto al GS mediato
SC-NEVPT$^{a,b}$        & 10.09  & 4.04  & 9.53  \\
                        & (0.087)&(0.105)&(0.081)\\
%SC-NEVPT$^{a,c}$        & 10.04  & 4.00  & 9.44  \\ Rispessto al GS mediato
SC-NEVPT$^{a,c}$        &  9.97  & 3.93  & 9.37  \\
                        & (0.095)&(0.113)&(0.091)\\
%PC-NEVPT$^{a,b}$        & 10.02  & 4.11  & 9.53  \\ Rispessto al GS mediato
PC-NEVPT$^{a,b}$        &  9.94  & 4.03  & 9.45  \\
                        & (0.100)&(0.109)&(0.088)\\
%PC-NEVPT$^{a,c}$        &  9.93  & 4.04  & 9.41  \\ Rispessto al GS mediato
PC-NEVPT$^{a,c}$        &  9.80  & 3.91  & 9.28  \\
                        & (0.112)&(0.118)&(0.097)\\
CASPT2 \cite{Merch95}   &  9.77  & 3.91  & 9.09  \\
MC/BMP \cite{Paris96}
                        & 10.37  & 3.83  &13.69  \\ 
(SC)$^2$ CAS+SD$^b$ \cite{Ruiz03}
                        &  9.89  & 4.15  & 9.35  \\
(SC)$^2$ MR+SD$^c$ \cite{Ruiz03}
                        &  9.74  & 4.04  & 9.33  \\
CCR(3)$^b$ \cite{Ruiz03}
                        &  9.80  & 4.01  & 9.29  \\
CCR(3)$^c$ \cite{Ruiz03}
                        &  9.64  & 3.97  & 9.25  \\
MRD-CI \cite{Hachey95}
                        &  9.60  & 4.05  & 9.35  \\
MR-CISD + Q \cite{Muller01}
                        &  9.80  & 4.07  & 9.40  \\
MR-AQCC  \cite{Muller01}
                        &  9.84  & 4.04  & 9.37  \\
EOM-CCSD \cite{gwalt95} &  9.47  & 3.98  & 9.33  \\
EOM-CCSD \cite{Wiberg02}&  9.37  & 4.04  & 9.43  \\
Exp \cite{Robin85}      &        & 4.07  &       \\
Exp \cite{Walzl87}      &        & 3.79  &       \\
\noalign{\smallskip}\hline
\end{tabular}\\
{\smallskip}

$^a$  This work\\
$^b$  ANO(S)+Ryd(S) basis (see text)\\
$^c$  ANO(L)+Ryd(L) basis (see text)\\
\end{table}
\begin{table}[h]
\caption{Energy differences (eV) between the perturbation and the (SC)$^2$ CAS+SD results 
of Ref. \cite{Ruiz03}}
\label{Tabforeemaes1}
\begin{tabular}{lccc}
\hline\noalign{\smallskip}
         & PC-NEVPT$^a$& SC-NEVPT$^a$ & CASPT2$^b$    \\
2 A$_1$  & $~$0.06 & $~$0.13  &    -0.02  \\
3 A$_1$  & $~$0.08 & $~$0.16  &    -0.02  \\
4 A$_1$  & $~$0.05 & $~$0.20  &    -0.12  \\
1 B$_2$  & $~$0.11 & $~$0.11  &  $~$0.13  \\
2 B$_2$  & $~$0.16 & $~$0.15  &  $~$0.13  \\
3 B$_2$  & $~$0.14 & $~$0.13  &  $~$0.13  \\
4 B$_2$  & $~$0.12 & $~$0.11  &  $~$0.12  \\
1 A$_2$  &   -0.12 &   -0.11  &    -0.24  \\
2 A$_2$  & $~$0.03 & $~$0.03  &  $~$0.02  \\
3 A$_2$  & $~$0.06 & $~$0.06  &  $~$0.03  \\
1 B$_1$  & $~$0.10 & $~$0.18  &    -0.26  \\
2 B$_1$  & $~$0.15 & $~$0.14  &  $~$0.11  \\
\noalign{\smallskip}\hline
MAE$^c$      & $~$0.10 & $~$0.13  &  $~$0.11  \\
\noalign{\smallskip}\hline
\end{tabular}\\
{\smallskip}

$^a$ This work\\
$^b$ Ref. \cite{Merch95}\\
$^c$ Mean Absolute Error
\end{table}
\begin{table}[h]
\caption{Energy differences (eV) between the perturbation and the (SC)$^2$ MR+SD results 
of Ref. \cite{Ruiz03}}
\label{Tabforeemaeb1}
\begin{tabular}{lcccc}
\hline\noalign{\smallskip}
         & PC-NEVPT$^a$ &  SC-NEVPT$^a$   \\
2 A$_1$  &   0.04   &    0.12     \\
3 A$_1$  &   0.18   &    0.25     \\
4 A$_1$  &   0.06   &    0.23     \\
1 B$_2$  &   0.21   &    0.20     \\
2 B$_2$  &   0.22   &    0.21     \\
3 B$_2$  &   0.21   &    0.21     \\
4 B$_2$  &   0.20   &    0.19     \\
1 A$_2$  &  -0.13   &   -0.11     \\
2 A$_2$  &   0.09   &    0.10     \\
3 A$_2$  &   0.14   &    0.14     \\
1 B$_1$  &  -0.05   &    0.04     \\
2 B$_1$  &   0.03   &    0.03     \\
\noalign{\smallskip}\hline
MAE$^b$      &   0.13   &    0.15    \\
\noalign{\smallskip}\hline
\end{tabular}\\
{\smallskip}

$^a$ This work\\
$^b$ Mean Absolute Error
\end{table}

\begin{table}[h]
\caption{Active spaces and number of states used in the average CAS-SCF calculations
for the acetone molecule (always 6 active electrons except in the case of the 
$^1$B$_1$ ($\sigma\rightarrow\pi^*$) which is computed with 4 active electrons}
\label{Tabaceact}       
\begin{tabular}{clc}
\hline\noalign{\smallskip}
\# MOs$^a$ & Symmetry and nature of states    & \# states$^b$ \\
\noalign{\smallskip}\hline\noalign{\smallskip}
    (2230)   & $^1$A$_1$ (GS;$n_y\!\rightarrow\! 3p_y$,$3d_{yz}$;$\pi\!\rightarrow\!\pi^*$) & 4 \\
    (2200)   & $^1$B$_1$ ($\sigma\!\rightarrow\!\pi^*$) & 1\\
    (2211)   & $^1$B$_1$ ($n_y\!\rightarrow\! 3d_{xy}$) & 1 \\
    (6210)   & $^1$B$_2$ ($n_y\!\rightarrow\! 3s$, $3p_z$, $3d_{x^2\!-\!y^2}$, $3d_{z^2}$) & 4 \\
    (2410)   & $^1$A$_2$ ($n_y\!\rightarrow\!\pi^*$,$3p_x$,$3d_{xz}$) & 3 \\
\noalign{\smallskip}\hline
\end{tabular}\\
{\smallskip}

$^a$ number of molecular orbitals in the active space for 
the four irreducible representations ($a_1$,$b_1$,$b_2$,$a_2$)\\
$^b$ number of states used in the average procedure
% Or use
\end{table}

\newpage
%\begin{minipage}{\textwidth}
\begin{table*}[h]
\caption{Vertical excitation energies (eV) for the Rydberg states of
the acetone molecule}
\label{Tabaceeery}
\begin{tabular}{lccccccccc}
\hline\noalign{\smallskip}
Method &2 A$_1$ & 3 A$_1$ & 1 B$_2$ & 2 B$_2$ & 3 B$_2$ & 4 B$_2$ & 2 A$_2$ & 3 A$_2$ &  2 B$_1$ \\
 &(\nto{3p_y}) & (\nto{3d_{yz}}) & (\nto{3s}) & (\nto{3p_z}) & 
(\nto{3d_{x^2\!-\!y^2}}) & (\nto{3d_{z2}}) & (\nto{3p_x}) &
(\nto{3d_{xz}}) & (\nto{3d_{xy}})     \\
\noalign{\smallskip}\hline\noalign{\smallskip}
%CAS--SCF$^a$  &  7.00 & 7.55   & 5.12  & 5.85  & 6.40  & 6.49  & 6.99  & 7.70  & 6.48  \\
CAS--SCF$^a$  &  7.91  & 8.46  & 6.02  & 6.75  & 7.30  & 7.39  & 7.29  & 7.99  & 7.38  \\
%SC-NEVPT$^a$&  7.57  & 8.20  & 6.92  & 7.85 & 8.42  & 8.54  & 7.33  & 8.09  & 8.53  \\
SC-NEVPT$^a$&  7.40  & 8.03  & 6.75  & 7.67  & 8.25  & 8.37  & 7.48  & 8.24  & 8.36  \\
            & (0.182)&(0.179)&(0.159)&(0.155)&(0.154)&(0.153)&(0.168)&(0.166)&(0.154)\\
%PC-NEVPT$^a$&  7.60  & 8.24  & 7.03  & 7.97  & 8.54  & 8.67  & 7.47  & 8.24  & 8.67  \\
PC-NEVPT$^a$&  7.27  & 7.91  & 6.71  & 7.64  & 8.22  & 8.34  & 7.39  & 8.17  & 8.35  \\
            & (0.195)&(0.192)&(0.166)&(0.161)&(0.160)&(0.158)&(0.177)&(0.175)&(0.158)\\
CASPT2 \cite{Merch96}
            &  7.26  & 7.91  & 6.58 & 7.48   & 8.04  & 8.18  & 7.34  & 8.09  & 8.20  \\
EOM-CCSD \cite{gwalt95}
            &  7.45  & 8.23  & 6.39 & 7.51   & 7.95  & 8.48  & 7.41  & 8.44  & 8.43  \\
EOM-CCSD \cite{Wiberg02}
            &  7.41  & 8.02  & 6.42 & 7.39   & 7.82  & 8.10  & 7.31  & 8.04  & 8.11  \\
Exp \cite{Merch96}
            &        & 7.8   &      &        & 8.09  &       &       &       & 8.17  \\
Exp \cite{Philis93}
            &        &       & 6.35 &        &       &       &       &       &       \\
Exp \cite{Robin85}
            &        &       & 6.36 &        &       &       & 7.45  &       &       \\
Exp \cite{Mcdia88}
            &  7.41  &       &      & 7.45   &       &       & 7.36  &       &       \\
\noalign{\smallskip}\hline
\end{tabular}\\
{\smallskip}

$^a$  This work\\
The numbers in parenthesis are the squared norms of the first order
corrections to the wave function. The squared
norm for the ground state is 0.164 (NEV-PT SC) and 0.167 (NEV-PT PC).
\end{table*}

\clearpage
\newpage
\begin{table}[h]
\caption{Vertical excitation energies (eV) for the valence states of
the acetone molecule}
\label{Tabaceeeval}
\begin{tabular}{lccc}
\hline\noalign{\smallskip}
Method &4 A$_1$ & 1 A$_2$ & 1 B$_1$ \\
 & (\pipis) & ($n_y\rightarrow\pi^*$) & (\sipis) \\
\noalign{\smallskip}\hline\noalign{\smallskip}
%CAS--SCF$^a$             & 10.70 & 4.78  & 9.48 \\
CAS--SCF$^a$             & 11.60  & 5.57  &10.38 \\
%SC-NEVPT$^a$           &  9.77  & 4.36  & 9.46 \\
SC-NEVPT$^a$           &  9.60  & 4.42  & 9.29 \\
                       & (0.204)&(0.189)&(0.182)\\
%PC-NEVPT$^a$           &  9.33  & 4.46  & 9.55 \\
PC-NEVPT$^a$           &  9.01  & 4.22  & 9.23 \\
                       & (0.293)&(0.207)&(0.190)\\
CASPT2 \cite{Merch96}  &  9.16  & 4.18  & 9.10 \\
EOM-CC \cite{gwalt95}  &  9.15  & 4.48  & 9.30 \\
EOM-CC \cite{Wiberg02} &  8.52  & 4.47  & 8.87 \\
Exp \cite{Walzl87}     &        & 4.38  &      \\
Exp \cite{Robin85}     &        & 4.43  &      \\
\noalign{\smallskip}\hline
\end{tabular}\\
{\smallskip}

$^a$  This work\\
The numbers in parenthesis are the squared norms of the first order 
corrections to the wave function. The squared
norm for the ground state is 0.164 (NEV-PT SC) and 0.167 (NEV-PT PC).
\end{table}


\end{document}
