\begin{center}
\begin{threeparttable}
\footnotesize 
\begin{tabular*}{\textwidth}{l@{\hspace*{1mm}}ccccc}
\hline
Method & \snpi & \tnpi & $S_2$  & \tpipi & \tspi \\
\hline
CASSCF (CAS(A)) \tnote{a}      & 3.63 (3.56)& 3.36 (3.30)& 7.90 (7.82)& 4.36 (4.29) &  7.77 (7.71) \\
NEVPT2 SC (CAS(A)) \tnote{a}      & 3.52 (3.46)& 3.17 (3.10)& 7.67 (7.59)& 4.38 (4.31) &  7.28 (7.22) \\
NEVPT2 PC (CAS(A)) \tnote{a}      & 3.54 (3.47)& 3.17 (3.10)& 7.66 (7.58)& 4.41 (4.34) &  7.28 (7.21) \\
CASSCF (CAS(C)) \tnote{a}      & 3.92 (3.82)& 3.64 (3.54)& 8.25 (8.15)& 4.75 (4.67) &  7.89 (7.79) \\
NEVPT2 SC (CAS(C)) \tnote{a}      & 3.57 (3.48)& 3.21 (3.11)& 7.65 (7.55)& 4.48 (4.40) &  7.20 (7.10) \\
NEVPT2 PC (CAS(C)) \tnote{a}      & 3.57 (3.48)& 3.18 (3.08)& 7.62 (7.52)& 4.49 (4.42) &  7.17 (7.07) \\
CASSCF(10e/9o) \cite{jcp-105-5927-1996}   & 3.71 (3.80)&            &              &            &             \\
CIS \cite{jpc-97-4293-1993} &     4.54   &    3.63    &    8.65    &    3.64     &             \\
MRD-CI\cite{jpc-99-16576-1995} &            &    3.22    &            &    4.43     &    7.15     \\
MR-CISD+Q \cite{mp-100-1647-2002} &     3.42   &            &            &             &             \\
CASPT2 \cite{tca-92-227-1995} &     3.37   &            &    7.43    &             &             \\
MRD-CI \cite{jpc-99-8050-1995} &     3.64   &            &    7.95    &             &             \\
EOM-CCSD \cite{jpca-106-4192-2002} &     3.70   &            &    8.43    &             &             \\
CCSD(T)\tnote{b}&            & 3.17 (3.08)&            &             &             \\
MR-AQCC\tnote{c}&     3.60   &            &    7.45    &             &             \\
MR-CISD+Q\tnote{c} &     3.63   &            &    7.56    &             &             \\
Experimental           & 3.50\tnote{d}
                                    & 3.12\tnote{d}
				                 &            & 4.70\tnote{e},4.83\tnote{f}
						                            &             \\
\hline
\end{tabular*}
\caption{\footnotesize Adiabatic transition energies (eV) (all states computed at the CASSCF equilibrium
geometry) for the \npi, \pipi\ and \spi\   singlet and triplet excited
states of formaldehyde.  The values in parentheses are the ZPE corrected
values.  
\label{tbl:adiab_form}}
\begin{tablenotes}
\footnotesize
\item[a] This work.
\item[b] CCSD(T) value using the TZ2P(f,d) basis, Ref. \citen{jcp-108-5281-1998}.
\item[c] Extrapolated to basis set limit using cc-pVXZ basis sets (X=D, T and Q), indicated with
TQ in Ref. \citen{jcp-114-746-2001}.
\item[d] Optical spectroscopy results from Ref. \citen{arpc-34-31-1983,tcc-150-167-1989}.
\item[e] Estimated in Ref. \citen{jpc-99-16576-1995} from a polynomial fit of the electron-impact data for the 
$\nu_2$ progression in Ref. \citen{cp-70-291-1982}.
\item[f] Lowest observed peak in the $\nu_2$ progression assigned to $\nu^\prime=1$
in Ref. \citen{cp-70-291-1982}.
\end{tablenotes}
\end{threeparttable}
\end{center}
