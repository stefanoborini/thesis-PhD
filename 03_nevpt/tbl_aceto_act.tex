\begin{center}
\begin{threeparttable}
\begin{tabular*}{0.80\textwidth}{clc}
\hline\noalign{\smallskip}
\# MOs \tnote{a} & Symmetry and nature of states    & \# states\tnote{b} \\
\noalign{\smallskip}\hline\noalign{\smallskip}
    (2230)   & $^1$A$_1$ (GS; $n_y\!\rightarrow\! 3p_y$,$3d_{yz}$; $\pi\!\rightarrow\!\pi^*$) & 4 \\
    (2200)   & $^1$B$_1$ ($\sigma\!\rightarrow\!\pi^*$) & 1\\
    (2211)   & $^1$B$_1$ ($n_y\!\rightarrow\! 3d_{xy}$) & 1 \\
    (6210)   & $^1$B$_2$ ($n_y\!\rightarrow\! 3s$, $3p_z$, $3d_{x^2\!-\!y^2}$, $3d_{z^2}$) & 4 \\
    (2410)   & $^1$A$_2$ ($n_y\!\rightarrow\!\pi^*$, $3p_x$, $3d_{xz}$) & 3 \\
\noalign{\smallskip}\hline
\end{tabular*}
\caption{\footnotesize Active spaces and number of states used in the average CASSCF
calculations for the acetone molecule (always 6 active electrons except in
the case of the $^1$B$_1$ ($\sigma\rightarrow\pi^*$) which is computed with
4 active electrons)}
\label{tbl:aceto_act}
\begin{tablenotes}
\footnotesize
\item[a] number of molecular orbitals in the active space for the four
irreducible representations ($a_1$,$b_1$,$b_2$,$a_2$)
\item[b] number of states used in the average procedure
\end{tablenotes}
\end{threeparttable}
\end{center}
