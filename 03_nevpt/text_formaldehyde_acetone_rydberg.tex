\section{Evaluation: Rydberg states}

\begin{center}
\textit{This section is based on the article at Ref. \citen{tca-111-352-2004}.}
\end{center}
{\ }\\
\vspace{-1mm}

This section focuses on the description of vertical transitions to valence and
Rydberg excited states with Single-State NEVPT on formaldehyde and acetone.
It aims at completing the detailed description of the excited states of
carbonyl. In the presented evaluation a state-average CASSCF has been
performed. This is different from the previous analysis, where a single state
CASSCF approach was used.

\subsection{Formaldehyde}

The vertical spectrum of the formaldehyde molecule was computed at the
ground state experimental geometry\cite{jms-18-344-1965} (R$_{\mbox{\tiny CO}}$=1.208 \AA,
R$_{\mbox{\tiny CH}}$=1.116 \AA\ and $\theta_{\mbox{\tiny
HCH}}$=116.5$^{\circ}$). As in the previous evaluation, the molecule belongs
to the C$_{2v}$ point group symmetry and lies in the $yz$ plane with the
C and O atoms on the $z$ axis.  The
ANO-1 basis set\cite{tca-77-291-1990} has been used with two different
contraction schemes: the smaller (indicated with ANO(S)) is C,O
[$4s3p1d$]/H[$2s1p$] and the larger (ANO(L)) is C,O
[$6s5p3d2f$]/H[$4s3p2d$].  These valence basis sets have been augmented with
diffuse functions using the procedure described in the previous section, in
order to properly describe the diffuse orbitals involved in the Rydberg
states. These basis functions have been obtained by contraction of a
set of $8s8p8d$ gaussian primitives. Two contraction schemes are considered:
[$1s1p1d$], (Ryd(S)) and [$3s3p3d$], (Ryd(L)).  For a direct comparison
between NEVPT, CASPT2\cite{tca-92-227-1995} and Size-Consistent
Self-Consistent CI ((SC)$^2$-CI)\cite{mp-101-483-2003}, we used in the
calculations the combinations ANO(S)-Ryd(S) and ANO(L)-Ryd(L).

The molecular orbitals are obtained from average CASSCF calculations which
involve the lowest states of a given symmetry. The active spaces, together
with the number and the nature of the states considered in the average
procedure are reported in Tab. \ref{tbl:form_act} and are taken from Ref.
\citen{tca-92-227-1995}.

\begin{center}
\begin{threeparttable}
\begin{tabular*}{0.80\textwidth}{clc}
\hline
\# MOs\tnote{a} & Symmetry and nature of states    & \# states\tnote{b} \\
\hline
    (0340)   & $^1$A$_1$ (GS; $n_y\!\rightarrow\!3p_y$,$3d_{yz}$; $\pi\!\rightarrow\!\pi^*$) & 4 \\
    (2200)   & $^1$B$_1$ ($\sigma\!\rightarrow\!\pi^*$) & 1\\
    (0211)   & $^1$B$_1$ ($n_y\!\rightarrow\! 3d_{xy}$) & 1 \\
    (4210)   & $^1$B$_2$ ($n_y\!\rightarrow\!3s$, $3p_z$, $3d_{x^2\!-\!y^2}$, $3d_{z^2}$) & 4 \\
    (0410)   & $^1$A$_2$ ($n_y\!\rightarrow\! \pi^*$, $3p_x$, $3d_{xz}$) & 3 \\
\hline
\end{tabular*}
\caption{\footnotesize Active spaces and number of states used in the average CASSCF calculations
for the formaldehyde molecule (in all cases 4 active electrons)}
\label{tbl:form_act}       
\begin{tablenotes}
\footnotesize
\item[a] number of molecular orbitals in the active space for the four
irreducible representations ($a_1$,$b_1$,$b_2$,$a_2$)
\item[b] number of states used in the average procedure
\end{tablenotes}
\end{threeparttable}
\end{center}


The number of orbitals for each irreducible representation
was chosen by the authors so that all the states of interest could be
correctly described. In some cases the active space was enlarged in order
to minimize the effect of the intruder states problem in the CASPT2
calculations.  Given that NEVPT is not affected by the intruder states
problem, in this evaluation a reduction of the active space should be
possible but the comparison of our results with those of Refs.
\citen{tca-92-227-1995} and \citen{mp-101-483-2003} would be in this case less
clear.

The energies of the states are computed in a state specific multireference
perturbation scheme.  The zero-order description of each state is obtained
from a CASCI calculation using the average CASSCF active orbitals. The
inactive orbitals have been canonized to diagonalize the state-specific Fock
matrixes.

A second-order correction to the energy is computed using the NEVPT2 Strongly
Contracted and Partially Contracted variants. All orbitals and
electrons are included in the perturbative treatment. The excitation
energies are computed with respect to the same ground state energy, which is
evaluated as the second order correction to the energy with the reference
energy and wavefunction obtained from a state-specific CASSCF calculation
with four electrons in two $b_1$ ($\pi$ + $\pi^*$) and two $b_2$ ($n_y$ +
virtual) orbitals. This approach for the calculation of the excitation
energies differs from the one used in the CASPT2 calculation, where a
different ground state energy was used for each irreducible representation.

\begin{center}
\begin{threeparttable}
\tiny
\begin{tabular}{lccccccccc}
\hline
Method &2 A$_1$ & 3 A$_1$ & 1 B$_2$ & 2 B$_2$ & 3 B$_2$ & 4 B$_2$ & 2 A$_2$ & 3 A$_2$ &  2 B$_1$ \\
 &($3p_y$) & ($3d_{yz}$) & ($3s$) & ($3p_z$) & 
($3d_{x^2\!-\!y^2}$) & ($3d_{z2}$) & ($3p_x$) &
($3d_{xz}$) & ($3d_{xy}$)     \\
\hline
CASSCF\tnote{a,b}  &   8.07 & 9.18  & 7.37  & 8.15  & 9.08  & 9.21  & 8.84  & 9.78  & 9.16 \\
CASSCF\tnote{a,c}  &   8.04 & 9.12  & 7.29  & 8.08  & 8.99  & 9.12  & 8.81  & 9.72  & 9.12 \\
NEVPT SC\tnote{a,b}&   8.27 & 9.42  & 7.28  & 8.11  & 9.13  & 9.30  & 8.33  & 9.34  & 9.26 \\
                & (0.077)&(0.076)&(0.092)&(0.090)&(0.083)&(0.082)&(0.080)&(0.079)&(0.079)\\
NEVPT SC\tnote{a,c}&   8.39 & 9.56  & 7.32  & 8.16  & 9.17  & 9.37  & 8.46  & 9.48  & 9.39 \\
                & (0.084)&(0.083)&(0.090)&(0.090)&(0.087)&(0.086)&(0.099)&(0.097)&(0.086)\\
NEVPT PC\tnote{a,b}&   8.20 & 9.34  & 7.28  & 8.12  & 9.14  & 9.31  & 8.33  & 9.34  & 9.27 \\
                & (0.084)&(0.084)&(0.095)&(0.093)&(0.085)&(0.084)&(0.082)&(0.081)&(0.081)\\
NEVPT PC\tnote{a,c}&   8.31 & 9.49  & 7.33  & 8.17  & 9.17  & 9.38  & 8.45  & 9.48  & 9.39 \\
                & (0.092)&(0.090)&(0.092)&(0.092)&(0.089)&(0.088)&(0.103)&(0.100)&(0.087)\\
CASPT2 \cite{tca-92-227-1995}
                &   8.12 & 9.24  & 7.30  & 8.09  & 9.13  & 9.31  & 8.32  & 9.31  & 9.23 \\
MC/BMP \cite{cp-205-323-1996}
                &   7.95 & 9.11  & 6.90  & 7.77  & 8.95  & 9.11  & 8.46  & 8.82  & 9.06 \\
(SC)$^2$ CAS+SD\tnote{b} \cite{mp-101-483-2003}
                &   8.14 & 9.26  & 7.17  & 7.96  & 9.00  & 9.19  & 8.30  & 9.28  & 9.12 \\
(SC)$^2$ MR+SD\tnote{c} \cite{mp-101-483-2003}
                &   8.27 & 9.31  & 7.12  & 7.95  & 8.96  & 9.18  & 8.36  & 9.34  & 9.36 \\
CCR(3)$^b$ \cite{mp-101-483-2003}
               &    8.01 & 9.16  & 7.11  & 7.91  & 8.99  & 9.21  & 8.25  & 9.26  & 9.12 \\
CCR(3)$^c$ \cite{mp-101-483-2003}
               &    8.14 & 9.27  & 7.16  & 7.99  & 9.04  & 9.27  & 8.38  & 9.40  & 9.25 \\
MRD-CI \cite{jpc-99-8050-1995}
                &   8.10 & 9.25  & 7.15  & 8.05  & 9.05  & 9.25  & 8.32  & 9.34  & 9.32 \\
MR-CISD + Q \cite{tca-106-369-2001}
                &   8.13 & 9.28  & 7.27  & 8.10  & 9.15  & 9.30  & 8.34  & 9.36  & 9.26 \\
MR-AQCC  \cite{tca-106-369-2001}
                &   8.24 & 9.38  & 7.21  & 8.03  & 9.09  & 9.24  & 8.46  & 9.49  & 9.37 \\
EOM-CCSD \cite{cpl-241-26-1995}
                &   7.99 &10.16  & 6.99  & 7.93  & 9.25  & 9.98  & 8.45  &10.67  & 9.84 \\
EOM-CCSD \cite{jpca-106-4192-2002}
                &   7.98 & 9.13  & 7.04  & 7.88  & 8.94  & 9.12  & 8.21  & 9.29  &10.89 \\
Exp. \cite{robin-hespm}
                &   7.97 &       & 7.11  & 8.14  & 8.88  &       & 8.37  &       &      \\
Exp. \cite{cp-70-291-1982}
                &        &       &       &       &       &       &       &       & 9.22 \\
Exp. \cite{jcsft-281-1643-1985,jcp-85-4228-1986}
                &        &       & 7.09  &       &       &       &       &       &      \\
\hline
\end{tabular}
\caption{\footnotesize Vertical excitation energies (eV) for the Rydberg states of
the formaldehyde molecule.  The numbers in parentheses are the squared norms
of the first order corrections to the wave function. The squared norm for
the ground state is 0.075 (NEVPT SC$^b$), 0.076 (NEVPT PC$^b$), 0.091
(NEVPT SC$^c$) and 0.097 (NEVPT PC$^c$).}
\label{tbl:form_exc_ryd}
\begin{tablenotes}
\footnotesize
\item[a] This work
\item[b] ANO basis set with contraction [$4s3p1d$/$2s1p$] + $1s1p1d$, ANO(S)+Ryd(S)
\item[c] ANO basis set with contraction [$6s5p3d2f$/$4s3p2d$] + $3s3p3d$, ANO(L)+Ryd(L)
\end{tablenotes}
\end{threeparttable}
\end{center}


The vertical excitation energies obtained in our calculations are reported
in Tab. \ref{tbl:form_exc_ryd} for the Rydberg states and in Tab.
\ref{tbl:form_exc_val} for the valence states, together with the previous
theoretical and experimental results.

In Tab. \ref{tbl:form_comparison_1} we show the comparison between our
results and those of Pitarch-Ruiz {\it et al} \cite{mp-101-483-2003}, which
can be considered a good reference since they involve the whole single plus
double excitations space on top of a CAS at a variational level with a
size-consistence correction. We remark that the mean absolute error (MAE) of
our results is always small with the worst case being represented by the
Strongly Contracted NEVPT2 in the ANO(L) + Ryd(L) basis (0.15 eV).  The small
errors appearing in Tab.~\ref{tbl:form_comparison_1} bear out the
reliability of NEVPT2 which can yield results of good accuracy, comparable
with much more refined calculations, but at a reduced computational cost.

For the case of the smaller basis (ANO(S) + Ryd(S)) the CASPT2 results are
also available. NEVPT2 and CASPT2 appear to be of the same quality, with the
former showing in all cases a small squared norm of the wavefunction
perturbation correction thus getting over the intruder states problem.

As to the comparison with the experimental data, beyond a satisfactory
general agreement with our theoretical results we can make the following
observations.

\begin{center}
\begin{threeparttable}
\footnotesize
\begin{tabular*}{0.80\textwidth}{l@{\hspace*{20mm}}ccc}
\hline
Method &4 A$_1$ & 1 A$_2$ & 1 B$_1$ \\
 & (\pipis) & ($n_y\!\!\rightarrow\!\!\pi^*$) & (\sipis) \\
\hline
CASSCF$^{a,b}$          & 10.59  & 5.28  & 9.89  \\
CASSCF$^{a,c}$          & 10.47  & 5.27  & 9.82  \\
NEVPT SC$^{a,b}$        & 10.09  & 4.04  & 9.53  \\
                        & (0.087)&(0.105)&(0.081)\\
NEVPT SC$^{a,c}$        &  9.97  & 3.93  & 9.37  \\
                        & (0.095)&(0.113)&(0.091)\\
NEVPT PC$^{a,b}$        &  9.94  & 4.03  & 9.45  \\
                        & (0.100)&(0.109)&(0.088)\\
NEVPT PC$^{a,c}$        &  9.80  & 3.91  & 9.28  \\
                        & (0.112)&(0.118)&(0.097)\\
CASPT2 \cite{tca-92-227-1995}   &  9.77  & 3.91  & 9.09  \\
MC/BMP \cite{cp-205-323-1996}
                        & 10.37  & 3.83  &13.69  \\ 
(SC)$^2$ CAS+SD$^b$ \cite{mp-101-483-2003}
                        &  9.89  & 4.15  & 9.35  \\
(SC)$^2$ MR+SD$^c$ \cite{mp-101-483-2003}
                        &  9.74  & 4.04  & 9.33  \\
CCR(3)$^b$ \cite{mp-101-483-2003}
                        &  9.80  & 4.01  & 9.29  \\
CCR(3)$^c$ \cite{mp-101-483-2003}
                        &  9.64  & 3.97  & 9.25  \\
MRD-CI \cite{jpc-99-8050-1995}
                        &  9.60  & 4.05  & 9.35  \\
MR-CISD + Q \cite{tca-106-369-2001}
                        &  9.80  & 4.07  & 9.40  \\
MR-AQCC  \cite{tca-106-369-2001}
                        &  9.84  & 4.04  & 9.37  \\
EOM-CCSD \cite{cpl-241-26-1995} &  9.47  & 3.98  & 9.33  \\
EOM-CCSD \cite{jpca-106-4192-2002}&  9.37  & 4.04  & 9.43  \\
Exp. \cite{robin-hespm}      &        & 4.07  &       \\
Exp. \cite{jcp-87-3796-1987}      &        & 3.79  &       \\
\hline
\end{tabular*}
\caption{\footnotesize Vertical excitation energies (eV) for the valence states of
the formaldehyde molecule.  The numbers in parentheses are the squared norms
of the first order corrections to the wavefunction. The squared norm for
the ground state is 0.075 (NEVPT SC$^b$), 0.076 (NEVPT PC$^b$), 0.091
(NEVPT SC$^c$) and 0.097 (NEVPT PC$^c$).}
\label{tbl:form_exc_val}
\begin{tablenotes}
\footnotesize
\item[a] This work
\item[b] ANO(S)+Ryd(S) basis (see text)
\item[c] ANO(L)+Ryd(L) basis (see text)
\end{tablenotes}
\end{threeparttable}
\end{center}


First, in accordance with most theoretical calculations our vertical
2~$^1$A$_1$  and 1~$^1$B$_2$ Rydberg transitions appear in inverted order
with respect to experimental adiabatic transitions \cite{robin-hespm}: a
more stringent comparison would require the calculation of adiabatic
transition with due allowance for the zero point energy correction

Second, the calculation of the $^1$A$_1$ Rydberg states with the larger basis
set introduces one more Rydberg state ($n_y\rightarrow\pi^*$) below the
valence $\pi\rightarrow\pi^*$. Such state has been ignored in the average
CASSCF since we were interested in transitions involving Rydberg orbitals
not exceeding the quantum number $n=3$

Finally, mixing between Rydberg and valence character may occur in both the
$^1$A$_1$ and $^1$B$_1$ transitions \cite{tca-92-227-1995,jpc-99-8050-1995}.
For a correct treatment of such states a quasi-degenerate treatment
is required \cite{acp-67-1-1987,cpl-288-299-1998}.

\begin{center}
\begin{threeparttable}
\footnotesize
\begin{tabular*}{0.80\textwidth}{l@{\hspace*{15mm}}ccc}
\hline
         & NEVPT PC\tnote{a}& NEVPT SC\tnote{a} & CASPT2\tnote{b}    \\
\hline
2 A$_1$  & $~$0.06 & $~$0.13  &    -0.02  \\
3 A$_1$  & $~$0.08 & $~$0.16  &    -0.02  \\
4 A$_1$  & $~$0.05 & $~$0.20  &    -0.12  \\
1 B$_2$  & $~$0.11 & $~$0.11  &  $~$0.13  \\
2 B$_2$  & $~$0.16 & $~$0.15  &  $~$0.13  \\
3 B$_2$  & $~$0.14 & $~$0.13  &  $~$0.13  \\
4 B$_2$  & $~$0.12 & $~$0.11  &  $~$0.12  \\
1 A$_2$  &   -0.12 &   -0.11  &    -0.24  \\
2 A$_2$  & $~$0.03 & $~$0.03  &  $~$0.02  \\
3 A$_2$  & $~$0.06 & $~$0.06  &  $~$0.03  \\
1 B$_1$  & $~$0.10 & $~$0.18  &    -0.26  \\
2 B$_1$  & $~$0.15 & $~$0.14  &  $~$0.11  \\
\hline
MAE\tnote{c}   & $~$0.10 & $~$0.13  &  $~$0.11  \\
\hline
\end{tabular*}
\caption{\footnotesize 
Energy differences (eV) between the perturbation and the (SC)$^2$ CAS+SD
results of Ref. \citen{mp-101-483-2003}
}
\label{tbl:form_comparison_1}
\begin{tablenotes}
\footnotesize
\item[a] This work
\item[b] Ref. \citen{tca-92-227-1995}
\item[c] Mean Absolute Error
\end{tablenotes}
\end{threeparttable}
\end{center}





\subsection{Acetone}

The computational strategy used for acetone closely follows the one applied
to formaldehyde. The vertical spectrum has been computed at the ground
state experimental geometry \cite{jms-18-344-1965}. The molecule belongs to
the C$_{2v}$ point group symmetry with the OCCC skeleton in the $yz$
plane (C and O atoms on the $z$ axis and with an orientation of the CH$_3$ 
groups that place the two H atoms lying in the $yz$ plane as far as
possible). For acetone, we only consider the C,O[$4s3p1d$]/H[$2s1p$]
contraction of the ANO basis set. The Rydberg states are described using a
set of $8s8p8d$ diffuse functions contracted to $1s1p1d$. 

As in formaldehyde, average CASSCF calculations provide the molecular
orbitals: the active spaces and the number and the nature of the states
considered in the average procedure are reported in Tab. \ref{tbl:aceto_act}
and are taken from Ref. \citen{jcp-104-1791-1996}. 

With respect to formaldehyde, the active space has been modified adding the
two CO $\sigma$ and $\sigma^*$ orbitals and the two CO $\sigma$ electrons,
except for the A$_1$ symmetry where three virtual orbitals (one of $b_1$ and
two of $b_2$ symmetry) have been removed. In Ref. \citen{jcp-104-1791-1996}
the CO $\sigma$ and $\sigma^*$ orbitals have been added to the active space
to correctly describe the adiabatic electronic transitions for the
valence states, in which an elongation of the CO bond is observed.
% due to the
% promotion of an electron to the $\pi^{*}$ orbital.

Given that we present here only results on the vertical transitions, also in
the case of acetone a reduction of the active space used in Ref.
\citen{jcp-104-1791-1996} would have been possible, but we have chosen to
maintain the same active space in order to have a meaningful comparison with
the CASPT2 data.

\begin{center}
\begin{threeparttable}
\begin{tabular*}{0.80\textwidth}{clc}
\hline\noalign{\smallskip}
\# MOs \tnote{a} & Symmetry and nature of states    & \# states\tnote{b} \\
\noalign{\smallskip}\hline\noalign{\smallskip}
    (2230)   & $^1$A$_1$ (GS; $n_y\!\rightarrow\! 3p_y$,$3d_{yz}$; $\pi\!\rightarrow\!\pi^*$) & 4 \\
    (2200)   & $^1$B$_1$ ($\sigma\!\rightarrow\!\pi^*$) & 1\\
    (2211)   & $^1$B$_1$ ($n_y\!\rightarrow\! 3d_{xy}$) & 1 \\
    (6210)   & $^1$B$_2$ ($n_y\!\rightarrow\! 3s$, $3p_z$, $3d_{x^2\!-\!y^2}$, $3d_{z^2}$) & 4 \\
    (2410)   & $^1$A$_2$ ($n_y\!\rightarrow\!\pi^*$, $3p_x$, $3d_{xz}$) & 3 \\
\noalign{\smallskip}\hline
\end{tabular*}
\caption{\footnotesize Active spaces and number of states used in the average CASSCF
calculations for the acetone molecule (always 6 active electrons except in
the case of the $^1$B$_1$ ($\sigma\rightarrow\pi^*$) which is computed with
4 active electrons)}
\label{tbl:aceto_act}
\begin{tablenotes}
\footnotesize
\item[a] number of molecular orbitals in the active space for the four
irreducible representations ($a_1$,$b_1$,$b_2$,$a_2$)
\item[b] number of states used in the average procedure
\end{tablenotes}
\end{threeparttable}
\end{center}


The energies of the states are computed following the strategy outlined for
formaldehyde and the transition energies are reported in Tab.
\ref{tbl:aceto_exc_ryd} for the Rydberg states and in Tab.
\ref{tbl:aceto_exc_val} for the valence states, together with the results of
other theoretical calculations and with some experimental results.

We note that our Partially Contracted NEVPT2 results compare very well with
the CASPT2 ones.  We also remark that in two of the valence transitions
(4~$^1$A$_1$, $\pi\rightarrow\pi^*$ and 1~$^1$A$_2$, $n_y\rightarrow\pi^*$)
the difference between Strongly and Partially Contracted NEVPT2 appears to be
unusually sizable (0.59 eV and 0.20 eV, respectively). We think that this is
symptom for the zero-order wavefunction to necessitate significant
improvement.

\begin{center}
\begin{threeparttable}
\tiny
\begin{tabular}{lccccccccc}
\hline
Method &2 A$_1$ & 3 A$_1$ & 1 B$_2$ & 2 B$_2$ & 3 B$_2$ & 4 B$_2$ & 2 A$_2$ & 3 A$_2$ &  2 B$_1$ \\
 &($3p_y$) & ($3d_{yz}$) & ($3s$) & ($3p_z$) & 
($3d_{x^2\!-\!y^2}$) & ($3d_{z2}$) & ($3p_x$) &
($3d_{xz}$) & ($3d_{xy}$)     \\
\hline
CASSCF\tnote{a}  &  7.91  & 8.46  & 6.02  & 6.75  & 7.30  & 7.39  & 7.29  & 7.99  & 7.38  \\
NEVPT2 SC\tnote{a}&  7.40  & 8.03  & 6.75  & 7.67  & 8.25  & 8.37  & 7.48  & 8.24  & 8.36  \\
            & (0.182)&(0.179)&(0.159)&(0.155)&(0.154)&(0.153)&(0.168)&(0.166)&(0.154)\\
NEVPT2 PC\tnote{a}&  7.27  & 7.91  & 6.71  & 7.64  & 8.22  & 8.34  & 7.39  & 8.17  & 8.35  \\
            & (0.195)&(0.192)&(0.166)&(0.161)&(0.160)&(0.158)&(0.177)&(0.175)&(0.158)\\
CASPT2 \cite{jcp-104-1791-1996}
            &  7.26  & 7.91  & 6.58 & 7.48   & 8.04  & 8.18  & 7.34  & 8.09  & 8.20  \\
EOM-CCSD \cite{cpl-241-26-1995}
            &  7.45  & 8.23  & 6.39 & 7.51   & 7.95  & 8.48  & 7.41  & 8.44  & 8.43  \\
EOM-CCSD \cite{jpca-106-4192-2002}
            &  7.41  & 8.02  & 6.42 & 7.39   & 7.82  & 8.10  & 7.31  & 8.04  & 8.11  \\
Exp.\cite{jcp-104-1791-1996}
            &        & 7.8   &      &        & 8.09  &       &       &       & 8.17  \\
Exp.\cite{jcp-98-3795-1993}
            &        &       & 6.35 &        &       &       &       &       &       \\
Exp.\cite{robin-hespm}
            &        &       & 6.36 &        &       &       & 7.45  &       &       \\
Exp.\cite{jcp-89-6086-1988}
            &  7.41  &       &      & 7.45   &       &       & 7.36  &       &       \\
\hline
\end{tabular}
\caption{\footnotesize Vertical excitation energies (eV) for the Rydberg states of
the acetone molecule}
\label{tbl:aceto_exc_ryd}
\begin{tablenotes}
\footnotesize
\item[a] This work. The numbers in parentheses are the squared norms of the
first order corrections to the wavefunction. The squared norm for the
ground state is 0.164 (NEVPT2 SC) and 0.167 (NEVPT2 PC).
\end{tablenotes}
\end{threeparttable}
\end{center}


\begin{center}
\begin{threeparttable}
\footnotesize
\begin{tabular*}{0.80\textwidth}{l@{\hspace*{20mm}}ccc}
\hline
Method &4 A$_1$ & 1 A$_2$ & 1 B$_1$ \\
 & (\pipis) & ($n_y\!\!\rightarrow\!\!\pi^*$) & (\sipis) \\
\noalign{\smallskip}\hline\noalign{\smallskip}
CASSCF\tnote{a}             & 11.60  & 5.57  &10.38 \\
NEVPT2 SC\tnote{a}           &  9.60  & 4.42  & 9.29 \\
                       & (0.204)&(0.189)&(0.182)\\
NEVPT2 PC\tnote{a}           &  9.01  & 4.22  & 9.23 \\
                       & (0.293)&(0.207)&(0.190)\\
CASPT2 \cite{jcp-104-1791-1996}  &  9.16  & 4.18  & 9.10 \\
EOM-CC \cite{cpl-241-26-1995}  &  9.15  & 4.48  & 9.30 \\
EOM-CC \cite{jpca-106-4192-2002} &  8.52  & 4.47  & 8.87 \\
Exp. \cite{jcp-87-3796-1987}     &        & 4.38  &      \\
Exp. \cite{robin-hespm}     &        & 4.43  &      \\
\hline
\end{tabular*}
\caption{\footnotesize Vertical excitation energies (eV) for the valence states of
the acetone molecule}
\label{tbl:aceto_exc_val}
\begin{tablenotes}
\footnotesize
\item[a] This work.  The numbers in parentheses are the squared norms of the
first-order corrections to the wavefunction. The squared norm for the
ground state is 0.164 (NEVPT2 SC) and 0.167 (NEVPT2 PC).
\end{tablenotes}
\end{threeparttable}
\end{center}



\subsection*{Final remarks}

Among the formal requirements satisfied by NEVPT2, the absence of intruder
states appears particularly interesting for the application to the
calculation of electronically excited states. The results shown in the
precedent section for the vertical transitions of formaldehyde and acetone
are of good quality and exhibit good agreements with the best calculations so
far performed as well as with the existing experimental data. 

Our calculations have been carried out starting from rather modest size
CASSCF wavefunctions. The computational overhead involved by the two forms
of NEVPT2 (Strongly and Partially Contracted) amounts to only a small
fraction of the CASSCF calculation for such small active orbital spaces and
this favorable situation is not expected to drastically change when passing
on to larger molecules, provided that the active space can be kept within
manageable dimensions.

No evidence of divergences or misbehaviors in the perturbation summation
has been found in the calculation of the Rydberg states, which are
particularly prone to exhibiting the appearance of intruder states. 
