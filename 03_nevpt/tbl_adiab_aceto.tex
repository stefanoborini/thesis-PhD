\begin{center}
\begin{threeparttable}
\footnotesize
\begin{tabular*}{\textwidth}{l@{\hspace*{1mm}}ccccc}
\hline
Method & \snpi & \tnpi & $S_2$\tnote{a} & \tpipi &       \tspi \\
\hline
CASSCF (CAS(A))\tnote{b}
                      & 3.86 (3.81)& 3.66 (3.61)& 7.28 (7.17)& 4.47 (4.42) &  7.84 (7.77) \\
NEVPT2 SC (CAS(A))\tnote{b}
                      & 3.74 (3.70)& 3.50 (3.45)& 6.63 (6.52)& 4.46 (4.41) &  7.25 (7.18) \\
NEVPT2 PC (CAS(A))\tnote{b}
                      & 3.75 (3.70)& 3.49 (3.44)& 6.61 (6.51)& 4.48 (4.42) &  7.20 (7.14) \\
CASPT2\tnote{c}
                      & 3.64       &            & 7.19       &             &              \\
MRCI \cite{jcp-111-205-1999}
                      &      (3.73)&            &            &             &              \\
MRMP \cite{jms_theo-461-145-1999}
                      & 3.54 (3.47)& 3.27 (3.21)&            &             &              \\
CASPT2 \cite{cpl-325-86-2000}
                      & 4.40 (4.35)& 4.05 (4.01)&            &             &              \\
	CASSCF\tnote{d}& 4.15 (4.11)& 3.93 (3.89)&            &             &              \\
TDDFT/B3P86 \cite{cpc-2-273-2001}
                      & 4.11 (4.07)& 3.69 (3.65)&            &             &              \\
TDDFT/BLYP\cite{jcp-108-4060-1998}
                      & 3.41       &            &            &             &              \\

Experimental          & \mbox{3.75\tnote{e}, 3.77\tnote{f}}              &  3.47\tnote{h}                            & &    5.15\tnote{e}, 5.3\tnote{g}&             \\
Experimental          &                                     3.8\tnote{g} &                \mbox{3.44--3.54\tnote{i}} & &                               &             \\
\hline
\end{tabular*}
\caption{\footnotesize Adiabatic transition energies (eV) (all states computed at the CASSCF equilibrium
geometry) for the \npi, \pipi\ and \spi\   singlet and triplet excited
states of acetone.  The values in parentheses are the ZPE corrected values.
\label{tbl:adiab_aceto}}
\begin{tablenotes}
\footnotesize
\item[a] The \pipi\ and \spi\ electronic configurations are mixed. No minima have been found on
the $S_3$ energy surface.
\item[b] This work.
\item[c] Theoretical values from Ref. \citen{jcp-104-1791-1996}. For \snpi\ the value is an
estimation based on the hypothesis that the lowering passing from the vertical to adiabatic transition
is equal to the one found in Ref. \citen{tca-92-227-1995} for formaldehyde. For the \spipi\ state the
molecular skeleton is maintained planar and the CH$_3$ groups have the GS geometry.
\item[d] CASSCF with 10 electrons in 8 orbitals using the 6-31G(d) basis set,
Ref. \citen{cpc-2-273-2001}.
\item[e] Electron-impact data from Ref. \citen{jcp-87-3796-1987}.
\item[f] Fluorescence excitation spectra in Ar supersonic nozzle beam, Ref.
\citen{jcp-82-3938-1985}.
\item[g] Electron-energy-loss spectroscopy in condensed phase at 18 K, Ref.
\citen{jcp-112-6707-2000}.
\item[h] Experimental determination of Refs. \citen{cp-155-149-1991},\citen{jcp-102-4447-1995},
as reported in Ref. \citen{pps-3-6-2004}.
\item[i] Experimental estimation of the 0-0 band, Ref. \citen{jcp-44-945-1966}.
\end{tablenotes}
\end{threeparttable}
\end{center}

