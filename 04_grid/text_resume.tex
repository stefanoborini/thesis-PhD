\pagestyle{empty}
\begin{center}
{\Huge \textbf Resum\'e chapitre \ref{chp:grid} \\ Integration de reseau}
\end{center}

La puissance computationnelle et l'espace disque ont connu une incroyable
augmentation depuis quelques ans. La vitesse d'interconnexion r\'eseau est
aussi augment\'ee, et cet incr\'ement ouvre des nouvelles possibilit\'es pour obtenir
puissance computationnelle et espace disque utilisant plusieurs ordinateurs
peu employ\'e, \`a cr\'eer une m\'eta-plateforme commune: le r\'eseau \`a grid.
Avec une r\'eseau \`a grid, il est possible obtenir
\begin{itemize}
\item ressources d'infrastructure physique, comme espace disque, temp de
calcul, espace RAM, agr\'egation et archiviation des donn\'ees, utilisation
condivise d'instrumentation.
\item ressources logiques, comme exp\'erimentation, analyse des donn\'ees,
mod\'elisation et simulation.
\item ressources humaines, comme condivision des comp\'etences et communication
\end{itemize}

Le r\'esultat est un syst\`eme h\'et\'erog\`ene distribu\'e, qui donne une structure
virtuelle \`a l'utilisateur. L'interaction avec les services est
standardis\'ee et uniforme, sans probl\`emes d'architecture de CPU ou de
localisation g\'eographique de l'utilisateur.

Le r\'eseau \`a grid le plus connu est probablement le web. Le WWW est un
r\'eseau pour le stockage des donn\'ees. Ces donn\'ees sont consult\'ees
avec une interface standard, le navigateur web, avec une protocole \`a bas
niveau, l'HTTP. Cette abstraction est suffisante pour lire et \'ecrire
fichiers ou ex\'ecuter un script.

En ce moment, trois types d'applications peuvent \^etre d\'ecrit pour le
r\'eseau
\begin{itemize}
\item \textit{Applications \`a communication r\'eduite}: o\`u le probl\`eme est
d\'ecompos\'e en morceaux presque ind\'ependant, et le calcul est
effectu\'e sur chaque morceau. Seti@Home\cite{seti-site} ou le calculs
m\'et\'eorologiques sont des exemples.
\item \textit{Applications \`a pas}: o\`u la proc\'edure est d\'ecompos\'e en pas
s\'equentiels, chacun effectu\'e dans l'environnement de le reseau \`a grid.
Cette approche est utilis\'ee pour partager l'acc\'es \`a des instruments et
l'\'elaboration des r\'esultats, la distribution du co\^ut d'administration
et l'acces aux logiciels locaux.
\item \textit{Acc\'es \`a ressources}: pour partager ressources comme
bas des donn\'ees, instrumentation, espace de stockage.
\end{itemize}

Pour les processus de chimie quantique, de l'exp\'erience a \'et\'e faite avec la
parall\'elisation des algorithmes avec PVM et MPI, mais les \'evaluations
\textit{ab initio} sont principalement s\'equentiels: normalement, on a l'\'evaluation des
integrales, l'optimisation d'un fonction d'onde, et l'application d'autres
techniques. L'architecture de r\'ef\'erence est donc l'Applications \`a pas.

Le reseau Metachem est un type de reseau d\'evelopp\'e dans le cadre d'une action
COST de la communaut\'e europ\'eenne. L'objectif est l'int\'egration
de logiciels de chimie quantique mis \`a disposition par plusieurs
laboratoires europ\'eens.

Pendant ce travail de th\`ese, deux probl\`emes pour l'int\'egration ont \'et\'e
\'etudi\'e et r\'esolus. 
Le premier probl\`eme est le d\'eveloppement d'un format de fichier commun
pour \'echanger information binaire. 
Le formats de fichier dej\`a d\'evelopp\'es ont \'et\'e analis\'es, et 
leur d\'efects pour l'ex\'ecution en grid ont \'et\'e evalu\'es. Une reseau
\`a grid n\'ecessite de consid\'erations additionnelles: le format ne doit
pas \^etre platform-dependent, parce que le reseau peut \^etre construit
avec ordinateurs ayant des architectures diff\'erentes.  Le format id\'eal
pour ce travail est HDF5, mais malheureusement ce format est complexe \`a
utiliser. Autres consid\'erations sont la simplicit\'e d'utilise de la
biblioth\`eque d'acces, exactitude des messages d'erreur pendant l'acces et
extensibilit\'e du format.

La recherche effectu\'ee a conduit au d\'eveloppement d'un mod\`ele des
donn\'ees qui a \'et\'e impl\'ement\'e dans la biblioth\`eque Q5Cost. Cette
biblioth\`eque donner l'acces \`a des entit\'ees chimiques et simplifie
l'utilisation \`a des programmateurs avec une exp\'erience chimique.

Q5Cost a \'et\'e utilis\'ee pour l'integration des diff\'erents logiciels,
et ouvre des nouvelles possibilit\'es pour le partage d'information
quantistique et l'integration des m\'ethodes \textit{ab initio}.
La biblioth\`eque a aussi une testsuite pour contr\^oler l'interface et
chercher les probl\`emes dans l'implementation. 
Le r\'esultat est une biblioth\`eque tr\`es solide et extensible.

Le probl\`emes principaux qui restent \`a resoudre dans Q5Cost sont: le
stockage des integrals n'est pas encore efficient en terms d'espace
occup\'e, et l'implementation du mod\`ele des donn\'ees n'est pas complete,
mais l'impl\'ementation courante est suffisante pour le premiers
d\'eveloppements et implementations.

Le deuxi\`eme probl\`eme est le management de la s\'equence d'op\'erations
\`a effectuer et la conversion de ces instructions dans un format local pour
chaque logiciel. Cette conversion doit \^etre effectu\'ee par un logiciel wrapper
\`a l'entr\'ee et \`a la sortie, et donc est fortement li\'ee au logiciel
impliqu\'e.
La s\'equence des op\'erations sera probablement d\'ecrite avec le format
de fichier XML, et donc une biblioth\`eque de lecture et \'ecriture de ce
format avec le langage Fortran, F90xml, a \'et\'e d\'evelopp\'ee. Cette
biblioth\`eque interface libgdome2 (biblioth\`eque pour le langage C) avec
Fortran. 

Des exp\'eriences pr\'eliminaires de communication des donn\'ees ont \'et\'e
effectu\'ees, et les r\'esultats sont tr\`es encourageants.

